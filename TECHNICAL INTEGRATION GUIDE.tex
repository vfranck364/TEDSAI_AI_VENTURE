TECHNICAL INTEGRATION GUIDE
Vercel + Google Cloud Platform
E-Commerce & Web Application Platform for TEDSAI.COM
DOCUMENT OVERVIEW
This technical guide provides comprehensive implementation instructions for integrating Vercel (frontend) with Google Cloud Platform (backend) to create a high-performance e-commerce and web application platform. The guide includes detailed configuration steps, code examples, security best practices, and optimization strategies specific to the TEDSAI.com project.
Target Audience: Full-stack developers, DevOps engineers, and technical architects
Prerequisites: Knowledge of Next.js, Node.js, Docker, and cloud infrastructure concepts
Estimated Implementation Time: 8-10 weeks for complete deployment
 
ARCHITECTURE OVERVIEW
System Components
Layer	Technology	Purpose
Frontend	Vercel + Next.js 14	Static/dynamic pages, Edge rendering
API Gateway	Next.js API Routes	Request routing, authentication middleware
Backend Services	Cloud Run (Docker)	AI chatbots, business logic, processing
Database	Firestore / Cloud SQL	User data, products, orders, real-time chat
File Storage	Cloud Storage	Product images, videos, user uploads
Authentication	Firebase Auth	User registration, login, OAuth
AI Services	Vertex AI	Machine learning models, NLP, recommendations
CDN	Vercel Edge Network	Global content delivery, edge caching

Data Flow Architecture
Request Flow:
1.	User accesses tedsai.com → Vercel Edge Network (closest geographic location)
2.	Static content served directly from Edge (HTML, CSS, images cached)
3.	Dynamic requests → Next.js API Routes on Vercel
4.	API Routes authenticate & route to Cloud Run services in GCP
5.	Cloud Run processes business logic, queries Firestore/Cloud SQL
6.	Response cached where appropriate, returned to user
7.	Media assets loaded from Cloud Storage via public CDN URLs
 
GOOGLE CLOUD PLATFORM SETUP
Project Initialization
1. Create GCP Project
# Install Google Cloud SDK
curl https://sdk.cloud.google.com | bash
exec -l $SHELL
gcloud init

# Create new project
gcloud projects create tedsai-prod --name="TEDSAI Production"
gcloud config set project tedsai-prod

# Enable billing
gcloud beta billing projects link tedsai-prod --billing-account=YOUR_BILLING_ID
2. Enable Required APIs
# Enable all necessary GCP services
gcloud services enable run.googleapis.com
gcloud services enable firestore.googleapis.com
gcloud services enable storage.googleapis.com
gcloud services enable aiplatform.googleapis.com
gcloud services enable cloudbuild.googleapis.com
gcloud services enable secretmanager.googleapis.com
gcloud services enable sqladmin.googleapis.com
3. Configure Budget Alerts
Navigate to GCP Console → Billing → Budgets & alerts
•	Create budget: $150/month (GCP portion of total $300 budget)
•	Set alerts at: 50% ($75), 75% ($112.50), 90% ($135)
•	Configure email notifications to technical team
Firestore Database Setup
1. Initialize Firestore
# Create Firestore database in Native mode
gcloud firestore databases create --region=europe-west1

Region Selection Rationale:
•	europe-west1 (Belgium): Lowest latency to Cameroon from Europe
•	Alternative: europe-west3 (Frankfurt) or europe-west2 (London)
2. Database Schema Design
Collections Structure:
Collection	Document ID	Key Fields	Purpose
users	userId	email, name, created	User profiles
products	productId	name, price, stock	Product catalog
orders	orderId	userId, items, total	Order history
chatSessions	sessionId	userId, messages[]	AI chat history
inventory	productId	quantity, location	Stock tracking

 
3. Security Rules Configuration
Create file: firestore.rules
rules_version = '2';
service cloud.firestore {
  match /databases/{database}/documents {
    // Users can only read/write their own data
    match /users/{userId} {
      allow read, write: if request.auth != null && request.auth.uid == userId;
    }
    // Products are public for reading
    match /products/{productId} {
      allow read: if true;
      allow write: if request.auth != null && request.auth.token.admin == true;
    }
    // Orders belong to users
    match /orders/{orderId} {
      allow read: if request.auth != null && resource.data.userId == request.auth.uid;
      allow create: if request.auth != null && request.resource.data.userId == request.auth.uid;
    }
  }
}
# Deploy security rules
firebase deploy --only firestore:rules
Cloud Storage Configuration
1. Create Storage Bucket
# Create bucket for product images
gsutil mb -c STANDARD -l europe-west1 gs://tedsai-media

# Make bucket publicly readable
gsutil iam ch allUsers:objectViewer gs://tedsai-media

# Set CORS for frontend access
echo '[{"origin":["https://tedsai.com","https://*.vercel.app"],"method":["GET"],"maxAgeSeconds":3600}]' > cors.json
gsutil cors set cors.json gs://tedsai-media
2. Folder Structure
•	products/ - Product images and videos
•	users/ - User profile pictures and uploads
•	blog/ - Blog post images
•	temp/ - Temporary files (7-day lifecycle)
3. Lifecycle Management
# Configure automatic deletion of temp files
echo '{"lifecycle":{"rule":[{"action":{"type":"Delete"},"condition":{"age":7,"matchesPrefix":["temp/"]}}]}}' > lifecycle.json
gsutil lifecycle set lifecycle.json gs://tedsai-media
 
CLOUD RUN SERVICES DEPLOYMENT
AI Chatbot Service
1. Create Dockerfile
Create: services/chatbot/Dockerfile
FROM node:18-slim
WORKDIR /app
COPY package*.json ./
RUN npm ci --only=production
COPY . .
ENV PORT=8080
EXPOSE 8080
CMD ["node", "index.js"]
2. Service Code Structure
Create: services/chatbot/index.js
const express = require('express');
const { Firestore } = require('@google-cloud/firestore');
const { VertexAI } = require('@google-cloud/aiplatform');

const app = express();
const firestore = new Firestore();
const vertexAI = new VertexAI({ project: 'tedsai-prod', location: 'europe-west1' });

app.use(express.json());

// Health check endpoint
app.get('/health', (req, res) => res.json({ status: 'healthy' }));

// Chat endpoint
app.post('/chat', async (req, res) => {
  try {
    const { userId, message, sessionId } = req.body;
    
    // Get chat history
    const sessionDoc = await firestore.doc(`chatSessions/${sessionId}`).get();
    const history = sessionDoc.exists ? sessionDoc.data().messages : [];
    
    // Call Vertex AI
    const response = await vertexAI.predict({
      instances: [{ content: message, context: history }]
    });
    
    // Save to Firestore
    await firestore.doc(`chatSessions/${sessionId}`).set({
      userId,
      messages: [...history, { user: message, ai: response, timestamp: new Date() }]
    }, { merge: true });
    
    res.json({ response });
  } catch (error) {
    res.status(500).json({ error: error.message });
  }
});

const port = process.env.PORT || 8080;
app.listen(port, () => console.log(`Chatbot running on port ${port}`));
3. Deploy to Cloud Run
# Build and push container image
gcloud builds submit --tag gcr.io/tedsai-prod/chatbot ./services/chatbot

# Deploy with scale-to-zero configuration
gcloud run deploy chatbot \
  --image gcr.io/tedsai-prod/chatbot \
  --platform managed \
  --region europe-west1 \
  --allow-unauthenticated \
  --min-instances 0 \
  --max-instances 10 \
  --memory 512Mi \
  --cpu 1
CRITICAL: --min-instances 0 ensures scale-to-zero. Never set this > 0 in production unless absolutely necessary for cold-start performance.

 
VERCEL FRONTEND SETUP
Project Initialization
1. Create Next.js Project
# Create Next.js 14 app
npx create-next-app@latest tedsai-frontend --typescript --app --tailwind
cd tedsai-frontend

2. Project Structure
tedsai-frontend/
├── app/
│   ├── (pages)/
│   │   ├── page.tsx          # Homepage (SSG)
│   │   ├── products/
│   │   │   └── page.tsx      # Product catalog (ISR)
│   │   ├── cart/
│   │   │   └── page.tsx      # Shopping cart (SSR)
│   │   └── account/
│   │       └── page.tsx      # User account (SSR)
│   ├── api/
│   │   ├── chat/route.ts     # Chatbot proxy
│   │   ├── products/route.ts # Product API
│   │   └── orders/route.ts   # Order processing
├── components/
│   ├── ChatWidget.tsx
│   ├── ProductCard.tsx
│   └── Navigation.tsx
├── lib/
│   ├── firebase.ts          # Firebase client config
│   └── api-client.ts        # GCP API client
└── next.config.js           # Next.js configuration

3. Configure Next.js
Edit: next.config.js
module.exports = {
  images: {
    domains: ['storage.googleapis.com'],
    formats: ['image/webp'],
  },
  // Enable ISR
  experimental: {
    isrMemoryCacheSize: 0,  // Disable in-memory cache for ISR
  },
  // Environment variables
  env: {
    NEXT_PUBLIC_GCP_PROJECT_ID: process.env.NEXT_PUBLIC_GCP_PROJECT_ID,
    NEXT_PUBLIC_FIREBASE_API_KEY: process.env.NEXT_PUBLIC_FIREBASE_API_KEY,
    CLOUD_RUN_CHATBOT_URL: process.env.CLOUD_RUN_CHATBOT_URL,
  },
};
Firebase Client Integration
Create: lib/firebase.ts
import { initializeApp } from 'firebase/app';
import { getAuth } from 'firebase/auth';
import { getFirestore } from 'firebase/firestore';

const firebaseConfig = {
  apiKey: process.env.NEXT_PUBLIC_FIREBASE_API_KEY,
  authDomain: 'tedsai-prod.firebaseapp.com',
  projectId: 'tedsai-prod',
  storageBucket: 'tedsai-prod.appspot.com',
};

const app = initializeApp(firebaseConfig);
export const auth = getAuth(app);
export const db = getFirestore(app);
 
API Routes Configuration
Chatbot API Route
Create: app/api/chat/route.ts
import { NextResponse } from 'next/server';

export async function POST(request: Request) {
  try {
    const { message, userId, sessionId } = await request.json();
    
    // Forward to Cloud Run chatbot service
    const response = await fetch(`${process.env.CLOUD_RUN_CHATBOT_URL}/chat`, {
      method: 'POST',
      headers: { 'Content-Type': 'application/json' },
      body: JSON.stringify({ message, userId, sessionId }),
    });
    
    const data = await response.json();
    return NextResponse.json(data);
  } catch (error) {
    return NextResponse.json({ error: 'Chat service unavailable' }, { status: 500 });
  }
}
Deploy to Vercel
1. Initialize Git Repository
git init
git add .
git commit -m "Initial commit"
git remote add origin https://github.com/yourusername/tedsai-frontend.git
2. Connect to Vercel
1.	Visit vercel.com and sign in with GitHub
2.	Click "Add New Project" and import your repository
3.	Configure environment variables (see next section)
4.	Deploy automatically on push to main branch
3. Configure Environment Variables
In Vercel Dashboard → Settings → Environment Variables:
Variable Name	Value / Source
NEXT_PUBLIC_GCP_PROJECT_ID	tedsai-prod
NEXT_PUBLIC_FIREBASE_API_KEY	From Firebase Console → Project Settings
CLOUD_RUN_CHATBOT_URL	From Cloud Run service URL (e.g., https://chatbot-xxx.run.app)
STORAGE_BUCKET_URL	https://storage.googleapis.com/tedsai-media

 
SECURITY & AUTHENTICATION
Firebase Authentication Setup
1. Enable Authentication Methods
In Firebase Console → Authentication → Sign-in method:
•	Enable Email/Password authentication
•	Enable Google OAuth (for social login)
•	Configure authorized domains: tedsai.com, *.vercel.app
2. Authentication Component
Create: components/AuthProvider.tsx
import { createContext, useContext, useEffect, useState } from 'react';
import { onAuthStateChanged, User } from 'firebase/auth';
import { auth } from '@/lib/firebase';

const AuthContext = createContext<{ user: User | null }>({ user: null });

export function AuthProvider({ children }: { children: React.ReactNode }) {
  const [user, setUser] = useState<User | null>(null);

  useEffect(() => {
    const unsubscribe = onAuthStateChanged(auth, (user) => setUser(user));
    return unsubscribe;
  }, []);

  return <AuthContext.Provider value={{ user }}>{children}</AuthContext.Provider>;
}

export const useAuth = () => useContext(AuthContext);
Cloud Run Access Control
1. Service Account Configuration
# Create service account for Vercel access
gcloud iam service-accounts create vercel-frontend \
  --display-name="Vercel Frontend Service Account"

# Grant invoker role to service account
gcloud run services add-iam-policy-binding chatbot \
  --region=europe-west1 \
  --member="serviceAccount:vercel-frontend@tedsai-prod.iam.gserviceaccount.com" \
  --role="roles/run.invoker"
2. API Key Authentication
Add authentication header to all Cloud Run requests:
// In API routes
const response = await fetch(cloudRunUrl, {
  headers: {
    'Content-Type': 'application/json',
    'X-API-Key': process.env.GCP_API_KEY,
  },
});
 
PERFORMANCE OPTIMIZATION
Image Optimization Strategy
1. Convert Images to WebP
# Batch convert images to WebP
for img in products/*.{jpg,png}; do
  cwebp -q 80 "$img" -o "${img%.*}.webp"
done
2. Upload to Cloud Storage
# Upload with caching headers
gsutil -m -h "Cache-Control:public, max-age=31536000" cp products/*.webp gs://tedsai-media/products/

3. Next.js Image Component
import Image from 'next/image';

<Image
  src="https://storage.googleapis.com/tedsai-media/products/item1.webp"
  alt="Product"
  width={800}
  height={600}
  loading="lazy"
  quality={80}
/>
Caching Configuration
Static Site Generation (SSG)
For: Homepage, About page, Blog posts
// app/page.tsx
export default async function Home() {
  // This page is statically generated at build time
  return <div>Welcome to TEDSAI</div>;
}
Incremental Static Regeneration (ISR)
For: Product catalog, frequently updated content
// app/products/page.tsx
export const revalidate = 3600; // Revalidate every hour

export default async function Products() {
  const products = await fetchProducts();
  return <ProductGrid products={products} />;
}
 
MONITORING & MAINTENANCE
Cost Monitoring Dashboard
GCP Budget Alerts
1.	Navigate to GCP Console → Billing → Budgets & alerts
2.	Create monthly budget: $150
3.	Configure threshold alerts:
◦	50% ($75) - Warning email to dev team
◦	75% ($112.50) - Critical email + Slack notification
◦	90% ($135) - Emergency alert + review Cloud Run configs
Vercel Usage Monitoring
•	Dashboard → Usage → Set bandwidth alert at 80GB (before overage)
•	Monitor function invocations weekly
•	Review build minutes (Pro plan: 6,000/month included)
Performance Monitoring
Enable Cloud Monitoring
# Enable Cloud Monitoring API
gcloud services enable monitoring.googleapis.com

Key Metrics to Monitor:
•	Cloud Run: Request count, latency, error rate, instance count
•	Firestore: Read/write operations, document count
•	Cloud Storage: Bandwidth usage, request count
•	Vercel: Page load times, Core Web Vitals (LCP, FID, CLS)
Maintenance Schedule
Frequency	Task	Purpose
Daily	Review error logs	Identify and fix issues quickly
Weekly	Cost review (first 3 months)	Ensure budget compliance
Monthly	Cloud Run config audit	Verify scale-to-zero settings
Monthly	Security updates	Update dependencies, patch vulnerabilities
Quarterly	Performance optimization	Analyze and improve Core Web Vitals
Quarterly	Database optimization	Review indexes, clean old data

 
E-COMMERCE SPECIFIC IMPLEMENTATION
Payment Integration
Recommended Payment Gateway: Stripe
Why Stripe:
•	Supports Cameroon with international card processing
•	PCI compliance handled by Stripe (reduces security burden)
•	Excellent Next.js integration libraries
Setup Payment Flow
# Install Stripe SDK
npm install @stripe/stripe-js stripe

Create: app/api/checkout/route.ts
import Stripe from 'stripe';
import { NextResponse } from 'next/server';

const stripe = new Stripe(process.env.STRIPE_SECRET_KEY!);

export async function POST(request: Request) {
  const { items, userId } = await request.json();
  
  // Create Stripe checkout session
  const session = await stripe.checkout.sessions.create({
    payment_method_types: ['card'],
    line_items: items.map(item => ({
      price_data: {
        currency: 'xaf', // Central African Franc
        product_data: { name: item.name },
        unit_amount: item.price,
      },
      quantity: item.quantity,
    })),
    success_url: `${process.env.NEXT_PUBLIC_URL}/success?session_id={CHECKOUT_SESSION_ID}`,
    cancel_url: `${process.env.NEXT_PUBLIC_URL}/cart`,
    metadata: { userId },
  });
  
  return NextResponse.json({ sessionId: session.id });
}
Inventory Management
Real-time Stock Updates with Firestore
Create: lib/inventory.ts
import { db } from './firebase';
import { doc, runTransaction } from 'firebase/firestore';

export async function reserveStock(productId: string, quantity: number) {
  const inventoryRef = doc(db, 'inventory', productId);
  
  return runTransaction(db, async (transaction) => {
    const inventoryDoc = await transaction.get(inventoryRef);
    
    if (!inventoryDoc.exists()) throw new Error('Product not found');
    
    const currentStock = inventoryDoc.data().quantity;
    if (currentStock < quantity) throw new Error('Insufficient stock');
    
    transaction.update(inventoryRef, { quantity: currentStock - quantity });
    return true;
  });
}
 
TESTING & QUALITY ASSURANCE
Load Testing
1. Install Testing Tools
npm install -g artillery

2. Create Load Test Script
Create: tests/load-test.yml
config:
  target: 'https://tedsai.com'
  phases:
    - duration: 60
      arrivalRate: 10
    - duration: 120
      arrivalRate: 50
scenarios:
  - flow:
      - get:
          url: '/'
      - get:
          url: '/products'

# Run load test
artillery run tests/load-test.yml
Security Testing
Security Checklist
•	Enable HTTPS only (Vercel automatic, GCP Cloud Run configured)
•	Configure CORS properly on Cloud Storage and Cloud Run
•	Implement rate limiting on API endpoints
•	Validate all user inputs (frontend and backend)
•	Use environment variables for all secrets (never hardcode)
•	Enable Firestore security rules (production-ready rules)
•	Regular dependency updates (npm audit, dependabot)
 
