SITE WEB TEDSAI COMPLEX
Document de Spécifications Complètes & Optimisées
 
📋 TABLE DES MATIÈRES
1.	Vision & Philosophie Générale
2.	Architecture de l'Information
3.	Identité Visuelle & Design System
4.	Description Détaillée des Pages
5.	L'Assistant TED - Spécifications
6.	Fonctionnalités Transversales
7.	Spécifications Techniques
8.	Stratégie SEO & Contenu
9.	Aspects Légaux & Conformité
10.	Plan de Déploiement
 
1. VISION & PHILOSOPHIE GÉNÉRALE
1.1 Concept Central
Le site TEDSAI Complex n'est pas une simple vitrine digitale, mais le système nerveux numérique d'un écosystème intelligent. Il incarne physiquement la synergie entre trois univers complémentaires :
•	🧠 TEDSAI IA : L'intelligence artificielle au service des entreprises
•	🍽️ viTEDia : Le restaurant gastronomique durable
•	🌱 SelecTED Garden : Le jardin urbain productif et traçable
1.2 Principes Directeurs
Unité dans la Diversité : Chaque compartiment possède sa propre page dédiée avec son identité visuelle distinctive, tout en restant clairement rattaché à l'écosystème TEDSAI.
Intelligence Contextuelle : Le site s'adapte au profil du visiteur (B2B, B2C, partenaire, presse) et à son comportement de navigation.
Démonstration par l'Exemple : Le site lui-même est une vitrine technologique de ce que TEDSAI peut accomplir en matière d'IA conversationnelle, de personnalisation et d'expérience utilisateur.
Traçabilité & Transparence : Chaque élément de l'écosystème peut être exploré en profondeur, de la donnée à l'action, du jardin à l'assiette.
1.3 Objectifs Mesurables
Pour TEDSAI IA (B2B) :
•	Générer 50+ leads qualifiés/mois dans les 6 mois
•	Taux de conversion formulaire → démonstration : 25%
•	30% des visiteurs interagissent avec TED
Pour viTEDia (Restaurant) :
•	40% des réservations via le site web
•	Panier moyen Click & Collect : 35€
•	Taux de rebond < 40% sur la page menu
Pour SelecTED Garden (Traçabilité) :
•	60% des clients restaurant scannent un QR code traçabilité
•	20% partagent leur expérience sur les réseaux sociaux
•	Génération de 10+ demandes de partenariat/trimestre
 
2. ARCHITECTURE DE L'INFORMATION
2.1 Structure de Navigation Complète
🏠 ACCUEIL
│
├── 🔷 L'ÉCOSYSTÈME TEDSAI
│   ├── Notre Vision
│   ├── La Boucle Vertueuse (Infographie Interactive)
│   ├── Valeurs & Engagements
│   └── Impact Social & Environnemental
│
├── 🧠 SOLUTIONS IA POUR ENTREPRISES
│   ├── Vue d'Ensemble des Solutions
│   ├── Par Secteur d'Activité
│   │   ├── Commerce & Retail
│   │   ├── Restauration & Hôtellerie
│   │   ├── Agriculture & Agroalimentaire
│   │   ├── Santé & Services
│   │   └── PME & Artisanat
│   ├── Par Problématique
│   │   ├── Automatisation de la Facturation
│   │   ├── Gestion Intelligente des Stocks
│   │   ├── Service Client 360° & Chatbots
│   │   ├── Analyse Prédictive & Business Intelligence
│   │   └── Optimisation des Opérations
│   ├── IA Playground (Démo Interactive)
│   ├── Cas Clients & Témoignages
│   ├── Méthodologie & Processus
│   ├── Tarification & Packages
│   └── Demander une Démo Personnalisée
│
├── 🍽️ viTEDia RESTAURANT
│   ├── Découvrir viTEDia
│   ├── Notre Philosophie Culinaire
│   ├── Menu du Jour (Dynamique & Live)
│   ├── Carte des Vins & Boissons
│   ├── Galerie Photos & Ambiance
│   ├── Réserver une Table
│   ├── Click & Collect / Livraison
│   ├── Événements Privés & Séminaires
│   ├── Avis Clients & Presse
│   └── Traçabilité de Votre Assiette
│
├── 🌱 SelecTED GARDEN
│   ├── Notre Production Agricole
│   ├── Visite Virtuelle 360°
│   ├── Techniques & Innovations
│   ├── Traçabilité en Direct
│   │   ├── Scanner un QR Code
│   │   └── Rechercher un Produit
│   ├── Le Projet Épices (En Développement)
│   ├── Agriculture Urbaine & Durabilité
│   ├── Partenariats & Approvisionnement
│   └── Visites Guidées & Ateliers
│
├── 📚 LE LAB TEDSAI (Blog)
│   ├── Tous les Articles
│   ├── Catégorie : Intelligence Artificielle
│   ├── Catégorie : Agriculture Urbaine
│   ├── Catégorie : Gastronomie Durable
│   ├── Catégorie : Actualités & Événements
│   └── S'abonner à la Newsletter
│
├── 👥 À PROPOS
│   ├── Notre Histoire
│   ├── L'Équipe
│   ├── Nos Partenaires
│   ├── Presse & Médias
│   ├── Nous Rejoindre (Recrutement)
│   └── Responsabilité Sociétale
│
└── 📞 CONTACT
    ├── Formulaire de Contact Intelligent
    ├── Coordonnées & Localisation
    ├── Horaires d'Ouverture
    └── FAQ par Compartiment
2.2 Parcours Utilisateurs Optimisés
Parcours #1 : Dirigeant PME cherchant une solution IA
1.	Accueil → Attire l'attention via bannière "Automatisez votre facturation en 48h"
2.	Clic CTA → Page Solutions IA > Automatisation Facturation
3.	Lecture contenu + visionnage vidéo explicative (90s)
4.	TED s'active : "Quel est votre volume mensuel de factures ?"
5.	Qualification rapide (5 questions)
6.	TED : "Parfait ! Vous pourriez économiser ~15h/mois. Testez notre IA Playground ?"
7.	Démo interactive (upload facture test → analyse instantanée)
8.	CTA final : "Planifier un audit gratuit" → Calendrier Calendly
9.	Confirmation + email de suivi automatique
Parcours #2 : Client local cherchant un restaurant
1.	Recherche Google "restaurant bio Yaoundé" → Atterrit sur viTEDia
2.	Page viTEDia > Menu du jour affiché immédiatement
3.	Survol d'un plat → Badge "🌱 Tomate du jardin" cliquable
4.	Clic badge → Modal traçabilité avec photo de la parcelle
5.	TED : "Ce plat vous tente ? Je peux réserver pour ce soir ?"
6.	Conversation naturelle : "Oui, 2 personnes, 19h30"
7.	TED : "Terrasse ou intérieur ?" → "Terrasse"
8.	Confirmation instantanée + SMS de rappel
Parcours #3 : Curieux découvrant l'écosystème
1.	Accueil → Vidéo héro captivante (20s en autoplay muet)
2.	Scroll → Section "Comment ça fonctionne ?"
3.	Clic sur infographie interactive écosystème
4.	Navigation libre entre les 3 pôles (IA/Restaurant/Jardin)
5.	Intrigue sur la traçabilité → Clic "Voir un exemple"
6.	Page Traçabilité avec démo vidéo du scan QR code
7.	CTA : "Venez tester chez viTEDia !" → Réservation
 
3. IDENTITÉ VISUELLE & DESIGN SYSTEM
3.1 Palette de Couleurs
🔷 TEDSAI Core (Identité Mère)
•	Primaire : Bleu Marine #0A2463 - Confiance, technologie, professionnalisme
•	Secondaire : Bleu Ciel #5AA9E6 - Innovation, accessibilité, clarté
•	Neutre Foncé : #1A1A2E - Textes, contrastes
•	Neutre Clair : #F5F5F7 - Backgrounds, espaces respirants
•	Blanc : #FFFFFF - Pureté, lisibilité
🍽️ viTEDia (Accents Gastronomiques)
•	Accent Principal : Rouge Bordeaux #8B1E3F - Élégance, passion culinaire
•	Accent Or : Or Champagne #D4AF37 - Raffinement, qualité premium
•	Neutre Chaud : Beige Crème #F4EDE3 - Chaleur, convivialité
🌱 SelecTED Garden (Tonalités Naturelles)
•	Accent Principal : Vert Forêt #2D5A27 - Nature, croissance, santé
•	Accent Terre : Terre d'Ombre #B68D40 - Authenticité, enracinement
•	Neutre Organique : Vert Sauge #C9D5B5 - Fraîcheur, douceur
3.2 Typographie
Titres & Headlines : Inter (Google Fonts)
•	Weights : 700 (Bold), 600 (SemiBold)
•	Usage : H1, H2, Navigation, Boutons
•	Raison : Moderne, lisible, professionnelle
Corps de Texte : Open Sans (Google Fonts)
•	Weights : 400 (Regular), 600 (SemiBold)
•	Usage : Paragraphes, descriptions, formulaires
•	Raison : Excellente lisibilité écran, universelle
Accent viTEDia : Playfair Display (Google Fonts)
•	Weights : 400 (Regular), 700 (Bold)
•	Usage : Titres de plats, citations chefs
•	Raison : Élégance gastronomique, sérif raffiné
3.3 Système de Grille & Espacements
Grille Responsive
•	Desktop (>1200px) : 12 colonnes, gutter 24px
•	Tablet (768-1199px) : 8 colonnes, gutter 20px
•	Mobile (<767px) : 4 colonnes, gutter 16px
Système d'Espacement (Base 8px)
--space-xs: 8px;    /* Marges internes boutons, badges */
--space-sm: 16px;   /* Espacement entre éléments proches */
--space-md: 24px;   /* Espacement standard entre sections */
--space-lg: 48px;   /* Séparation de blocs importants */
--space-xl: 96px;   /* Respiration entre grandes sections */
3.4 Composants UI Clés
Boutons
•	Primaire : Fond bleu #0A2463, texte blanc, hover #5AA9E6
•	Secondaire : Bordure bleue, texte bleu, hover fond #5AA9E6
•	viTEDia CTA : Fond bordeaux #8B1E3F, texte blanc, hover or #D4AF37
•	Garden CTA : Fond vert #2D5A27, texte blanc, hover terre #B68D40
Cartes (Cards)
•	Fond blanc, ombre légère 0 2px 8px rgba(0,0,0,0.08)
•	Bordure arrondie 8px
•	Hover : Translation Y -4px + ombre accentuée
•	Padding interne : --space-md
Badges / Tags
•	Traçabilité : Fond vert pâle, texte vert foncé, icône feuille
•	Nouveau : Fond bleu, texte blanc
•	Promo : Fond or, texte bordeaux
•	Taille : --space-xs vertical, --space-sm horizontal
Formulaires
•	Champs : Bordure #D1D5DB, radius 4px, focus #5AA9E6
•	Labels : Police Open Sans 600, couleur #1A1A2E
•	Erreurs : Bordure rouge, message en dessous, icône alerte
3.5 Animations & Transitions
Principes d'Animation
•	Subtilité : Jamais distrayant, toujours fonctionnel
•	Rapidité : 200-400ms max (perception instantanée)
•	Fluidité : Cubic-bezier pour accélérations naturelles
Animations Signature
/* Transition Standard */
transition: all 300ms cubic-bezier(0.4, 0, 0.2, 1);

/* Hover Cards (Élévation 3D) */
transform: translateY(-4px) scale(1.02);
box-shadow: 0 12px 24px rgba(0,0,0,0.15);

/* Apparition Progressive (Scroll) */
@keyframes fadeInUp {
  from {
    opacity: 0;
    transform: translateY(30px);
  }
  to {
    opacity: 1;
    transform: translateY(0);
  }
}

/* Pulse Boutons Importants */
@keyframes pulse {
  0%, 100% { transform: scale(1); }
  50% { transform: scale(1.05); }
}
Micro-interactions Spécifiques
•	Menu Navigation : Soulignement qui glisse de gauche à droite au hover
•	Badges Traçabilité : "Pop" avec légère rotation au clic
•	Infographie Écosystème : Lignes de connexion animées (flow de données)
•	TED Chatbot : Icône qui pulse doucement toutes les 3s pour attirer l'attention
•	Boutons CTA Principaux : Gradient background animé au hover
 
4. DESCRIPTION DÉTAILLÉE DES PAGES
4.1 PAGE D'ACCUEIL
URL : https://tedsai.com/ (FR) / https://tedsai.com/en/ (EN)
Objectif Principal : Capter l'attention en 5 secondes, orienter vers le bon parcours (B2B/B2C), démontrer l'unicité de l'écosystème.
Structure de la Page
Section 1 : Héro Section (Above the Fold)
•	Contenu Visuel : Vidéo background 20 secondes en boucle, autoplay sans son 
o	Séquence : Dashboard IA (3s) → Drone survolant le jardin (4s) → Chef préparant un plat (5s) → Client souriant (3s) → Vue d'ensemble du complexe (5s)
o	Format : 1920x1080, optimisé WebM + MP4 fallback, < 3MB
•	Overlay Gradient : Du bleu marine opaque (haut) au transparent (bas)
•	Texte Principal : 
•	H1 (60px, Inter Bold, Blanc) :"De la Data à l'Assiette,Un Écosystème Intelligent"Sous-titre (24px, Open Sans Regular, Blanc 90%) :"IA pour entreprises • Restaurant durable • Jardin urbain"
•	Double CTA : 
o	Bouton Primaire : "Découvrir nos Solutions IA" → Page Solutions IA
o	Bouton Secondaire : "Réserver chez viTEDia" → Page Restaurant
•	Élément Interactif : TED s'ouvre automatiquement après 3 secondes en bas à droite : 
•	💬 TED : "Bonjour ! Je suis TED, votre assistant TEDSAI.Cherchez-vous une solution IA pour votre entreprise ousouhaitez-vous réserver une table chez viTEDia ?"[Solutions IA] [Réserver Table] [Découvrir l'écosystème]
Section 2 : Les Trois Piliers de l'Écosystème
•	Layout : 3 cartes côte à côte (grid 1fr 1fr 1fr), gap 32px
•	Carte 1 : TEDSAI IA (Accent bleu) 
o	Icône : 🧠 (ou picto cerveau stylisé)
o	Titre : "Solutions IA Sur Mesure"
o	Texte : "Automatisez vos processus, optimisez vos stocks, propulsez votre service client avec l'intelligence artificielle pragmatique."
o	Stats : "250+ heures économisées/mois pour nos clients"
o	CTA : "Explorer les solutions" → /solutions-ia
o	Hover : Carte s'élève, fond passe à bleu ciel léger
•	Carte 2 : viTEDia Restaurant (Accent bordeaux) 
o	Icône : 🍽️ (ou picto assiette élégante)
o	Titre : "Gastronomie Traçable"
o	Texte : "Une cuisine raffinée où chaque ingrédient raconte son histoire, de notre jardin à votre assiette."
o	Stats : "95% des produits issus de notre jardin"
o	CTA : "Voir le menu du jour" → /vitedia
o	Hover : Carte s'élève, fond passe à or léger
•	Carte 3 : SelecTED Garden (Accent vert) 
o	Icône : 🌱 (ou picto feuille stylisée)
o	Titre : "Agriculture Urbaine Innovante"
o	Texte : "Un jardin productif au cœur de la ville, utilisant l'IoT et l'IA pour une traçabilité totale."
o	Stats : "Zéro pesticide, 100% traçable"
o	CTA : "Découvrir le jardin" → /selected-garden
o	Hover : Carte s'élève, fond passe à vert sauge
Section 3 : La Boucle Vertueuse (Infographie Animée)
•	Titre Section : "Comment Ça Fonctionne ? Un Écosystème en Synergie"
•	Visuel : Infographie circulaire animée (SVG ou Lottie) 
o	Nœud 1 : 🧠 IA → "Analyse les données de production et de consommation"
o	Flèche animée (flow de data) vers...
o	Nœud 2 : 🌱 Jardin → "Optimise les cultures selon la demande restaurant"
o	Flèche animée vers...
o	Nœud 3 : 🍽️ Restaurant → "Sert des plats frais et génère nouvelles données"
o	Flèche animée retour vers IA (boucle complète)
•	Interaction : Au clic sur chaque nœud, une infobulle détaillée apparaît 
o	Exemple clic sur IA : "Nos algorithmes prédisent les pics de demande pour éviter le gaspillage et maximiser la fraîcheur."
•	CTA : "Plonger dans notre vision" → /ecosysteme
Section 4 : Chiffres Clés & Impact
•	Layout : 4 colonnes de stats animées (compteurs qui s'incrémentent au scroll) 
o	Stat 1 : "1 500+ heures économisées" (pour clients IA)
o	Stat 2 : "95% produits locaux" (viTEDia)
o	Stat 3 : "0 pesticide" (Garden)
o	Stat 4 : "3 secteurs unifiés" (Écosystème)
•	Design : Fond bleu marine, texte blanc, chiffres en grande taille (72px)
Section 5 : Témoignages Vidéo
•	Titre : "Ils Ont Choisi TEDSAI"
•	Contenu : Carousel de 3 vidéos courtes (30-45s) 
o	Vidéo 1 : CEO PME locale → "L'IA de TEDSAI a divisé par 3 nos erreurs de facturation"
o	Vidéo 2 : Client restaurant → "Savoir d'où vient ma tomate change tout"
o	Vidéo 3 : Journaliste agro → "Un modèle d'agriculture urbaine à répliquer"
•	Format : Vignettes avec play button, modal vidéo au clic
Section 6 : Call-to-Action Final
•	Fond : Gradient bleu marine → bleu ciel
•	Texte : "Prêt à Transformer Votre Activité ou Savourer l'Innovation ?"
•	Double CTA : 
o	"Demander une démo IA gratuite" (Bouton large, animation pulse)
o	"Réserver ma table" (Bouton secondaire)
Section 7 : Footer Riche
•	Colonne 1 : Logo TEDSAI + Slogan + Réseaux sociaux
•	Colonne 2 : Liens Solutions IA (4 problématiques principales)
•	Colonne 3 : Liens viTEDia (Menu, Réserver, Traçabilité)
•	Colonne 4 : Liens SelecTED Garden (Production, Visite, Partenariats)
•	Colonne 5 : Contact, Presse, Recrutement
•	Bas de page : Mentions légales, CGU/CGV, RGPD, © 2024 TEDSAI Complex
 
4.2 PAGE L'ÉCOSYSTÈME TEDSAI
URL : /ecosysteme (FR) / /en/ecosystem (EN)
Objectif : Raconter l'histoire, la vision, les valeurs. Créer une connexion émotionnelle et inspirer confiance.
Structure de la Page
Section 1 : Héro Narratif
•	Visuel : Image pleine largeur du complexe en vue aérienne (drone)
•	Texte Overlay : 
•	H1 : "Un Écosystème Où l'Intelligence ArtificielleRencontre la Nature et la Gastronomie"Chapô (400 caractères) :"TEDSAI Complex n'est pas une simple entreprise, c'est unlaboratoire vivant où trois mondes fusionnent : la technologiede pointe, l'agriculture responsable et l'excellence culinaire.Découvrez comment nous réinventons la chaîne de valeur."
Section 2 : Notre Histoire
•	Layout : Timeline verticale (scroll storytelling) 
o	2021 : "L'idée naît : et si l'IA pouvait servir le concret ?"
o	2022 : "Lancement de TEDSAI IA, premiers clients PME"
o	2023 : "Création de SelecTED Garden, autonomie alimentaire"
o	2024 : "Ouverture de viTEDia, la boucle est bouclée"
o	2025 : "Expansion : Projet Épices & Nouvelles Solutions IA"
•	Visuel : Photos d'archive à chaque étape
Section 3 : La Boucle Vertueuse (Version Détaillée)
•	Infographie Interactive Avancée (développée en D3.js ou Three.js) 
o	Représentation 3D de l'écosystème avec connexions animées
o	Clic sur "IA" → Zoom sur les algorithmes utilisés (prédiction, optimisation)
o	Clic sur "Jardin" → Détails des cultures, capteurs IoT, rendements
o	Clic sur "Restaurant" → Métriques de satisfaction, zéro déchet, etc.
•	Données en Temps Réel (si faisable) : 
o	"Actuellement : 142 tomates prêtes à récolter"
o	"Plat le plus commandé cette semaine : Risotto aux cèpes du jardin"
o	"Temps moyen d'automatisation facturation : 47 secondes"
Section 4 : Nos Valeurs
•	Layout : 5 blocs icônographiques 
1.	Innovation Pragmatique : "L'IA doit résoudre de vrais problèmes, pas impressionner"
2.	Transparence Totale : "Chaque produit, chaque donnée, chaque processus est traçable"
3.	Durabilité Systémique : "Zéro déchet, circuits courts, économie circulaire"
4.	Excellence Humaine : "La technologie au service de l'humain, pas l'inverse"
5.	Collaboration Ouverte : "Partager nos apprentissages pour un impact collectif"
Section 5 : Impact Mesurable
•	Infographies Chiffrées : 
o	🌍 Environnement : "8 tonnes de CO2 évitées vs approvisionnement classique"
o	💼 Économie : "25 emplois locaux créés"
o	🤝 Social : "15 partenariats avec producteurs locaux"
o	🔬 Innovation : "3 brevets IA déposés"
Section 6 : Vision 2025-2030
•	Roadmap Visuelle : Carte avec expansion géographique projetée 
o	2025 : Doubler capacité jardin, lancer projet épices
o	2027 : Ouverture 2e restaurant, franchise IA solutions
o	2030 : Réplication du modèle dans 5 villes africaines
CTA Final : "Rejoindre l'Aventure" → Page Recrutement ou Contact Partenariat
 
4.3 PAGE SOLUTIONS IA POUR ENTREPRISES
URL : /solutions-ia (FR) / /en/ai-solutions (EN)
Objectif : Convertir les visiteurs B2B en leads qualifiés via démonstration de valeur, social proof et appel à l'action clair.
Structure de la Page
Section 1 : Héro B2B
•	Titre : "Transformez Votre PME avec l'Intelligence Artificielle"
•	Sous-titre : "Des solutions pragmatiques qui automatisent, optimisent et propulsent votre croissance. Pas de jargon, que des résultats."
•	USP (Unique Selling Proposition) : 
•	✓ Implémentation en 15 jours (pas 6 mois)✓ ROI mesurable dès le 1er mois✓ Support humain 24/7 (pas que des bots)✓ Adapté aux PME locales (pas que les grands groupes)
•	CTA Primaire : "Tester l'IA Playground" → Scroll vers section demo
•	CTA Secondaire : "Parler à un expert" → Calendly
Section 2 : Navigation par Problématique (Onglets Interactifs)
•	Design : Tabs horizontaux (mobile : accordion) 
o	Tab 1 : 🧾 Automatisation Facturation
o	Tab 2 : 📦 Gestion Intelligente Stocks
o	Tab 3 : 💬 Service Client 360°
o	Tab 4 : 📊 Analyse Prédictive & BI
o	Tab 5 : ⚙️ Optimisation Opérations
Contenu de Chaque Tab (Template Unifié) :
Exemple : Tab "Automatisation Facturation"
1.	Le Problème (Relatable)
2.	"Vos équipes passent 15h/semaine à saisir des factures manuellement.
3.	Les erreurs de saisie coûtent 3-7% du CA en litiges et retards de paiement.
4.	Les relances clients sont chronophages et désorganisées."
5.	Notre Solution (Comment ça Marche)
6.	L'IA TEDSAI scanne, extrait, catégorise et intègre automatiquement
7.	vos factures dans votre système comptable. Fini la double saisie.
8.	
9.	[Schéma Visuel : Facture PDF → IA TEDSAI → ERP/Comptabilité]
10.	
11.	Fonctionnalités :
12.	• Reconnaissance OCR avancée (99,8% précision)
13.	• Extraction automatique (fournisseur, montant, TVA, date)
14.	• Validation intelligente (détection anomalies)
15.	• Intégration directe ERP (Sage, QuickBooks, Odoo...)
16.	• Tableau de bord temps réel
17.	Résultats Clients (Social Proof)
18.	📊 Cas Client : Supermarché Local (120 factures/mois)
19.	
20.	Avant TEDSAI :
21.	❌ 16h/semaine de saisie manuelle
22.	❌ 12% d'erreurs de saisie
23.	❌ 45 jours délai moyen de paiement
24.	
25.	Après TEDSAI :
26.	✅ 2h/semaine de validation seulement
27.	✅ 0,3% d'erreurs résiduelles
28.	✅ 22 jours délai moyen de paiement
29.	
30.	ROI : 8 500€ économisés/an
31.	CTA Section
o	Bouton 1 : "Tester avec mes factures" → Upload demo
o	Bouton 2 : "Calculer mon ROI" → Calculateur interactif
o	Bouton 3 : "Demander une démo" → Formulaire court
Section 3 : Navigation par Secteur d'Activité
•	Layout : 5 cartes secteurs (grid 3x2) 
o	Commerce & Retail 🛒
o	Restauration & Hôtellerie 🏨
o	Agriculture & Agroalimentaire 🌾
o	Santé & Services 💊
o	PME & Artisanat 🔧
Chaque Carte Secteur :
•	Icône distinctive
•	3 défis typiques du secteur
•	3 solutions TEDSAI adaptées
•	Lien "Voir cas client [Secteur]" → Page dédiée témoignage
Section 4 : IA Playground (Démo Interactive)
•	Titre : "Testez Notre IA en 5 Minutes (Sans Engagement)"
•	Interface : Simulateur interactif 
•	Étape 1 : Choisissez votre besoin[Automatisation Facturation] [Prédiction Stocks] [Chatbot Client]Étape 2 : Uploadez un échantillon[Zone de drop fichier] "Glissez votre facture PDF/Excel"Étape 3 : Observez la magie→ Animation de traitement (3-5s)→ Résultat affiché : Données extraites, catégorisées→ Rapport PDF généré "Votre Potentiel d'Automatisation"Étape 4 : Aller plus loin[Planifier un audit gratuit] [Télécharger le rapport]
Section 5 : Méthodologie & Processus
•	Titre : "Comment Nous Travaillons : Simple et Efficace"
•	Timeline 5 Étapes : 
1.	Audit Gratuit (1h) : Analyse de vos processus actuels
2.	Proposition Sur Mesure (48h) : Devis détaillé + roadmap
3.	Développement Agile (2-4 semaines) : Sprints hebdomadaires
4.	Formation Équipe (1 journée) : Prise en main accompagnée
5.	Support Continu (24/7) : TED + équipe dédiée
Section 6 : Tarification & Packages
•	Approche : Transparence totale (rare dans l'IA B2B)
•	3 Packages :
Package Starter (PME 1-10 employés)
À partir de 499€/mois
✓ 1 solution IA (au choix)
✓ 100 opérations/mois incluses
✓ Support email
✓ Formation initiale

Idéal pour : Tester l'IA sans risque
Package Business (PME 10-50 employés)
À partir de 1 299€/mois
✓ 3 solutions IA combinées
✓ 500 opérations/mois incluses
✓ Support prioritaire 24/7
✓ Intégrations ERP avancées
✓ Tableaux de bord personnalisés

Idéal pour : Automatisation multi-services
Package Enterprise (>50 employés)
Sur devis
✓ Solutions illimitées sur mesure
✓ Opérations illimitées
✓ Gestionnaire de compte dédié
✓ SLA 99,9% garanti
✓ API complète
✓ White label possible

Idéal pour : Transformation digitale complète
Section 7 : FAQ B2B
•	20 questions/réponses accordéon : 
o	"L'IA va-t-elle remplacer mes employés ?"
o	"Combien de temps pour voir des résultats ?"
o	"Mes données sont-elles sécurisées ?"
o	"Puis-je intégrer avec mon ERP actuel ?"
o	"Que se passe-t-il si je veux arrêter ?"
o	etc.
Section 8 : Ressources Téléchargeables
•	Lead Magnets (contre email) : 
o	📄 Livre Blanc : "L'IA pour les PME Locales : Guide 2025"
o	📊 Checklist : "10 Processus à Automatiser en Priorité"
o	🎥 Webinar : "1h pour Comprendre l'IA Pratique"
o	📈 Calculateur ROI : "Estimez Vos Économies en 2 Minutes"
CTA Final Section
•	Fond : Gradient bleu
•	Texte : "Prêt à Rejoindre les 150+ PME Qui Ont Choisi TEDSAI ?"
•	CTA : "Planifier Mon Audit Gratuit (30 min)" → Calendly
•	Trust Badge : "🔒 Sans engagement • 🎁 Audit offert • ⚡ Réponse en 24h"
 
4.4 PAGE viTEDia RESTAURANT
URL : /vitedia (FR) / /en/vitedia (EN)
Objectif : Séduire les clients, faciliter la réservation/commande, démontrer la qualité et la traçabilité.
Structure de la Page
Section 1 : Héro Gourmand
•	Visuel : Carousel automatique de 5 photos HD 
o	Photo 1 : Plat signature en gros plan (bokeh professionnel)
o	Photo 2 : Salle restaurant ambiance chaleureuse
o	Photo 3 : Chef en cuisine concentré
o	Photo 4 : Terrasse avec vue jardin
o	Photo 5 : Assiette colorée de légumes frais
•	Texte Overlay : 
•	H1 : "viTEDia : De Notre Jardin à Votre Assiette"Sous-titre : "Une cuisine raffinée où 95% des produits viennentde notre jardin urbain. Chaque ingrédient a une histoire."
•	Widget Réservation Flottant (Sticky top right) : 
•	[Date] [Heure] [Personnes] [Réserver]
Section 2 : Notre Philosophie Culinaire
•	Layout : Texte + Image côte à côte
•	Texte : 
•	"Chez viTEDia, nous croyons que la grande cuisine commenceau jardin. Notre chef [Nom] travaille main dans la main avecnos maraîchers pour créer des menus qui célèbrent la saison,le terroir et la traçabilité.Zéro compromis sur la qualité. Zéro distance entre la terreet l'assiette. Zéro secret sur l'origine."
•	Photo : Portrait du chef avec légumes du jardin
Section 3 : Menu du Jour (Dynamique)
•	Mise à Jour : Automatique chaque matin (connexion base de données)
•	Design : Tableau élégant avec typographie Playfair Display
┌─────────────────────────────────────────────────────────┐
│          MENU DU JOUR - Mercredi 13 Décembre            │
├─────────────────────────────────────────────────────────┤
│ ENTRÉES                                                  │
│ • Velouté de Potimarron du Jardin 🌱                    │
│   Crème de coco, graines torréfiées        12€          │
│   [Badge : Parcelle B3 • Récolte 10/12]    [Tracer]    │
│                                                          │
│ • Tartare de Tomates Anciennes 🌱                       │
│   Basilic, huile d'olive, pain maison      14€          │
│   [Badge : 5 variétés du jardin]           [Tracer]    │
├─────────────────────────────────────────────────────────┤
│ PLATS                                                    │
│ • Risotto Crémeux aux Cèpes 🌱                          │
│   Parmesan affiné, herbes fraîches        24€          │
│   [Badge : Champignons parcelle F1]        [Tracer]    │
│                                                          │
│ • Poisson du Bassin Piscicole 🌱                        │
│   Légumes de saison rôtis                  28€          │
│   [Badge : Tilapia - Bassin #2]            [Tracer]    │
├─────────────────────────────────────────────────────────┤
│ DESSERTS                                                 │
│ • Tarte Tatin Pommes du Jardin 🌱                       │
│   Glace vanille artisanale                 9€           │
│   [Badge : Pommes Reine des Reinettes]     [Tracer]    │
└─────────────────────────────────────────────────────────┘

[CTA : Commander en Click & Collect]  [Réserver une Table]
Interaction Traçabilité :
•	Clic sur badge 🌱 → Modal popup avec : 
o	Photo de la parcelle/bassin
o	Date de plantation/mise en eau
o	Date de récolte
o	Techniques de culture (bio, sans pesticide, etc.)
o	"Voir le parcours complet" → Page Traçabilité
Section 4 : Carte des Vins & Boissons
•	Onglets : Vins / Bières / Softs / Cocktails
•	Chaque Vin : Photo bouteille, origine, notes de dégustation, prix
•	Sélection : "Vins locaux & bio prioritaires"
Section 5 : Galerie Immersive
•	Grid Photos : 12 photos HD (4x3) 
o	Plats, ambiance, équipe, détails décoration
•	Lightbox : Clic → Plein écran avec navigation fléchée
Section 6 : Réservation Facile
•	Intégration : Widget Calendly ou développement custom
•	Formulaire : 
•	Nom : [___________]Email : [___________]Téléphone : [___________]Date : [Calendrier popup]Heure : [Dropdown : 18h00 → 22h00 par tranches 15min]Nombre de personnes : [Dropdown 1-10+]Préférences : [Zone texte] "Allergie, occasion spéciale, etc."[Terrasse] [Intérieur] [Pas de préférence][Valider ma Réservation]
•	Confirmation : Email automatique + SMS rappel J-1
•	TED Integration : "Besoin d'aide ? TED peut vous assister en direct"
Section 7 : Click & Collect / Livraison
•	Activation : Toggle en haut de page "Manger sur place / À emporter"
•	Mode À Emporter : 
o	Menu affiché avec checkbox pour sélection
o	Panier en sticky sidebar
o	Heure de retrait choisie (30min minimum délai)
o	Paiement en ligne (Stripe/PayPal)
o	Notification SMS "Votre commande est prête"
Section 8 : Événements Privés
•	Titre : "Privatisez viTEDia pour Vos Événements"
•	Capacité : Jusqu'à 50 personnes (salle complète)
•	Formules : Cocktail / Assis / Buffet
•	Cas d'Usage : Anniversaires, séminaires d'entreprise, mariages intimes
•	CTA : "Demander un Devis Personnalisé" → Formulaire dédié
Section 9 : Avis Clients
•	Source : Google Reviews API (5 derniers avis 5★)
•	Format : Cartes avec note, texte court, nom client, date
•	CTA : "Laisser un Avis" → Google Business Profile
Section 10 : Presse & Médias
•	Logos Médias : Journaux locaux, magazines culinaires
•	Citations : 3-4 extraits d'articles élogieux
•	CTA : "Espace Presse" → Page dédiée avec communiqués, photos HD
Footer viTEDia (Spécifique)
•	Horaires d'ouverture détaillés
•	Adresse + Google Maps embed
•	Téléphone direct
•	Email contact
•	Liens réseaux sociaux (Instagram focus)
 
4.5 PAGE SelecTED GARDEN
URL : /selected-garden (FR) / /en/selected-garden (EN)
Objectif : Éduquer sur l'agriculture urbaine, démontrer la traçabilité, inspirer confiance, générer partenariats.
Structure de la Page
Section 1 : Héro Nature
•	Visuel : Vidéo drone survolant le jardin (30s)
•	Texte Overlay : 
•	H1 : "SelecTED Garden : L'Agriculture Urbaine Réinventée"Sous-titre : "2 hectares au cœur de Yaoundé. Zéro pesticide.100% traçable. De la graine à l'assiette en quelques mètres."
•	CTA : "Découvrir en 360°" → Visite virtuelle
Section 2 : Notre Production
•	Layout : Grid 3x2 de catégories 
o	🥬 Légumes de Saison (25 variétés)
o	🍎 Fruits (10 espèces)
o	🌿 Herbes Aromatiques (15 variétés)
o	🥚 Œufs Frais (100 poules)
o	🐟 Poissons (Aquaponie tilapia)
o	🌶️ Épices (En développement)
Chaque Catégorie : Clic → Page dédiée avec :
•	Liste exhaustive des variétés
•	Photos haute résolution
•	Saisonnalité (calendrier visuel)
•	Quantité produite/mois
•	Utilisations culinaires (liens recettes viTEDia)
Section 3 : Techniques & Innovations
•	Titre : "Comment Nous Cultivons : Entre Tradition et Technologie"
•	Blocs Explicatifs :
1.	Agriculture Biologique Certifiée
o	Compostage sur site (zéro déchet organique)
o	Lutte biologique (coccinelles, insectes auxiliaires)
o	Rotation des cultures (3 ans minimum)
2.	IoT & Capteurs Intelligents
o	50 capteurs de sol (humidité, pH, température)
o	Stations météo connectées
o	Irrigation automatisée pilotée par IA
o	Dashboard temps réel (accessible au public)
3.	Aquaponie Circulaire
o	Bassins piscicoles (tilapia)
o	Eau des poissons fertilise les plantes
o	Plantes filtrent l'eau des poissons
o	Cycle 100% naturel, zéro intrant chimique
4.	Serres Intelligentes
o	Contrôle climatique automatisé
o	LED horticoles optimisés
o	Rendement 3x supérieur vs plein champ
Section 4 : Visite Virtuelle 360°
•	Technologie : Photos sphériques interactives (Matterport ou Kuula)
•	7 Points de Vue : 
1.	Entrée du jardin (panneau signalétique)
2.	Serres maraîchères (tomates, salades)
3.	Parcelles extérieures (carottes, choux)
4.	Poulailler (poules en liberté)
5.	Bassins aquaponiques (tilapias)
6.	Zone compostage (économie circulaire)
7.	Local de stockage (récoltes du jour)
•	Interactions : Hotspots cliquables → Infobulles explicatives
•	CTA : "Réserver une Visite Guidée Physique" → Formulaire
Section 5 : Traçabilité en Direct
•	Titre : "Suivez le Parcours de Votre Ingrédient"
Option 1 : Scanner QR Code
[Icône Grand QR Code]
"Scannez le QR code sur votre ticket viTEDia
ou sur l'étiquette produit"
•	Utilise caméra smartphone
•	Redirige vers page traçabilité produit unique
Option 2 : Recherche par Code
[Champ de saisie : "Entrez le code produit"]
Exemple : TOM-131224-B3

[Bouton : Tracer]
Résultat Traçabilité (Page Dédiée Dynamique) :
┌───────────────────────────────────────────────────────┐
│   🍅 Tomate Cœur de Bœuf Bio                          │
│   Code : TOM-131224-B3                                 │
├───────────────────────────────────────────────────────┤
│ 📍 ORIGINE                                             │
│ Parcelle : Serre #2, Section B3                       │
│ [Photo de la serre]                                    │
│                                                        │
│ 🌱 PLANTATION                                          │
│ Date : 15 septembre 2024                               │
│ Variété : Cœur de Bœuf Russe (bio certifié)          │
│ Semences : Fournisseur local BioDom                   │
│                                                        │
│ 🌦️ CONDITIONS DE CULTURE                               │
│ Température moyenne : 24°C                             │
│ Irrigation : Goutte-à-goutte automatisée               │
│ Traitements : Aucun pesticide (lutte biologique)      │
│ [Graphique des données capteurs IoT]                  │
│                                                        │
│ ✂️ RÉCOLTE                                             │
│ Date : 13 décembre 2024                                │
│ Récoltée par : Jean K. (maraîcher)                    │
│ Poids : 320g                                           │
│                                                        │
│ 🚚 LIVRAISON                                           │
│ À viTEDia : 13 décembre 2024, 11h00                   │
│ Distance : 50 mètres (jardin → cuisine)               │
│ Temps : 5 minutes                                      │
│                                                        │
│ 🍽️ UTILISATION CULINAIRE                              │
│ Plat : Tartare de Tomates Anciennes                   │
│ Chef : Marie L.                                        │
│ Menu du : 13 décembre 2024                             │
└───────────────────────────────────────────────────────┘

[Partager sur Facebook] [Partager sur Instagram]
[Télécharger le Certificat Traçabilité PDF]
Section 6 : Le Projet Épices (Teaser)
•	Titre : "Bientôt : Nos Propres Épices Bio"
•	Visuel : Mockup de sachets d'épices design
•	Texte : 
•	"Nous expérimentons actuellement la culture de 8 épices(curcuma, gingembre, piment, basilic, etc.). Lancementprévu : T2 2025."
•	CTA : "Être Informé du Lancement" → Formulaire email
Section 7 : Impact & Durabilité
•	Infographies Chiffrées : 
o	💧 "8 000L d'eau économisés/mois vs agriculture classique"
o	🌍 "12 tonnes CO2 évitées/an (circuits courts)"
o	♻️ "100% déchets organiques compostés"
o	🐝 "15 ruches accueillies (pollinisation naturelle)"
•	Certifications : Logos Bio, Label Local, Zéro Déchet
Section 8 : Partenariats & Approvisionnement
•	Texte : 
•	"Nous fournissons actuellement viTEDia (95% des besoins).Nous sommes ouverts aux partenariats avec :- Restaurants engagés en cuisine durable- Épiceries fines locales- Cantines scolaires/entreprises- Programmes CSR d'entreprises"
•	CTA : "Proposer un Partenariat" → Formulaire B2B
Section 9 : Visites Guidées & Ateliers
•	Titre : "Venez Nous Rencontrer"
•	Offres : 
o	Visite Guidée Gratuite : Samedi 10h-12h (réservation obligatoire)
o	Atelier Compostage : 1er dimanche du mois (25€/pers)
o	Atelier Semis Enfants : Vacances scolaires (15€/enfant)
o	Team Building Entreprise : Sur devis
•	Formulaire Réservation : Date, type activité, nombre participants
Footer SelecTED Garden
•	Adresse jardin (si différente de viTEDia)
•	Contact partenariats
•	Boutique en ligne (future) : Paniers hebdomadaires
 
4.6 PAGE LE LAB TEDSAI (Blog)
URL : /blog ou /le-lab (FR) / /en/the-lab (EN)
Objectif : Établir l'autorité, générer trafic organique SEO, nourrir leads B2B et B2C.
Structure de la Page
Section 1 : Header Blog
•	Titre : "Le Lab TEDSAI : Penser, Expérimenter, Partager"
•	Baseline : "Notre laboratoire d'idées sur l'IA pragmatique, l'agriculture urbaine et la gastronomie durable."
•	Barre de Recherche : "Rechercher un article..."
Section 2 : Catégories (Filtres)
•	Layout : Pills horizontales cliquables 
o	🧠 Intelligence Artificielle (Tag bleu)
o	🌱 Agriculture Urbaine (Tag vert)
o	🍽️ Gastronomie Durable (Tag bordeaux)
o	📰 Actualités TEDSAI (Tag bleu ciel)
o	🎓 Tutoriels & Guides (Tag gris)
Section 3 : Article Vedette
•	Format : Grande carte featured (33% page width) 
o	Image HD pleine largeur
o	Tag catégorie
o	Titre (H2, 60 caractères max)
o	Extrait (150 caractères)
o	Meta : Auteur, date, temps lecture
o	CTA : "Lire l'article"
Section 4 : Grille Articles Récents
•	Layout : Grid 3 colonnes (responsive 1 colonne mobile)
•	Chaque Carte : 
o	Thumbnail image (16:9)
o	Tag catégorie
o	Titre (H3)
o	Extrait court
o	Meta : Date, lecture
o	Lien "Lire"
Section 5 : Pagination
•	Navigation : [← Précédent] [1] [2] [3] ... [Suivant →]
•	Infinite Scroll (optionnel) : Articles suivants se chargent au scroll
Section 6 : Sidebar (Desktop)
•	Newsletter : 
•	📧 Restez Informé"Recevez nos articles & actus 2x/mois"[Email] [S'abonner]
•	Articles Populaires : Top 5 avec miniatures
•	Tags Nuage : Mots-clés cliquables
Template Article Individuel :
Header Article
•	Breadcrumb : Accueil > Blog > [Catégorie] > [Titre]
•	Tag catégorie
•	Titre H1 (70 caractères idéal)
•	Meta : Auteur (avec photo), date publication, temps lecture
•	Image featured (1200x630, optimisée social sharing)
Corps Article
•	Typographie : Open Sans 18px, line-height 1.7
•	Largeur : 700px max (lisibilité optimale)
•	Éléments enrichis : 
o	Sous-titres H2/H3 (structure claire)
o	Listes à puces / numérotées
o	Citations en exergue (pull quotes)
o	Images inline avec légendes
o	Vidéos embed (YouTube/Vimeo)
o	Tableaux comparatifs
o	CTA mid-article ("Télécharger notre guide gratuit")
Footer Article
•	Tags Associés : Liens vers autres tags
•	Partage Social : Boutons LinkedIn, Twitter, Facebook, WhatsApp
•	Auteur Bio : Photo, nom, titre, mini-bio (100 mots), lien profil
•	Articles Similaires : 3 suggestions automatiques (IA ou manuelles)
•	Commentaires : Désactivés (ou modération stricte) pour éviter spam
Exemples d'Articles (Calendrier Editorial 3 Mois)
Intelligence Artificielle (B2B) :
1.	"5 Processus à Automatiser en Priorité dans une PME Locale"
2.	"ROI de l'IA : Combien Économise Vraiment une Petite Entreprise ?"
3.	"IA vs Logiciel Classique : Quand Franchir le Pas ?"
4.	"Chatbots 2025 : Pourquoi 70% Échouent (Et Comment Réussir)"
5.	"Cas Client : Comment [Restaurant X] a Réduit ses Coûts de 40%"
Agriculture Urbaine (B2C + Institutionnel) :
1.	"Agriculture Urbaine : 10 Chiffres Qui Vont Vous Surprendre"
2.	"Comment Nous Cultivons 200kg de Légumes/Semaine sur 2 Hectares"
3.	"Aquaponie vs Hydroponie : Notre Retour d'Expérience"
4.	"IoT au Jardin : Les Capteurs Qui Transforment Notre Production"
5.	"Zéro Pesticide, Zéro Compromis : Notre Protocole de Lutte Biologique"
Gastronomie Durable (B2C) :
1.	"Du Jardin à l'Assiette : Notre Menu de Décembre Décrypté"
2.	"Pourquoi la Traçabilité Alimentaire Va Devenir la Norme"
3.	"Recette Signature : Le Velouté de Potimarron de Notre Chef"
4.	"Manger de Saison : Le Calendrier des Légumes au Cameroun"
5.	"Réduire le Gaspillage : Comment viTEDia Atteint 98% de Valorisation"
 
4.7 PAGE À PROPOS
URL : /a-propos (FR) / /en/about (EN)
Objectif : Humaniser la marque, inspirer confiance, recruter talents, attirer partenaires/médias.
Structure de la Page
Section 1 : Notre Histoire
•	Titre : "L'Aventure TEDSAI : Relier l'IA au Réel"
•	Narrative : Récit storytelling (500 mots) 
o	Genèse : Pourquoi TEDSAI a été créé
o	Fondateurs : Vision, parcours, motivations
o	Évolution : Jalons clés 2021-2024
o	Aujourd'hui : État des lieux (clients, production, équipe)
o	Demain : Ambitions 2025-2030
•	Timeline Visuelle : Frise chronologique illustrée
Section 2 : L'Équipe
•	Layout : Grid de cartes membres (4 colonnes desktop)
•	Chaque Carte : 
o	Photo professionnelle (mais chaleureuse)
o	Nom + Fonction
o	Mini-bio (50 mots)
o	LinkedIn (optionnel)
Exemples :
👤 [Photo]
Dr. [Nom Fondateur]
CEO & Co-fondateur
"Convaincu que l
SITE WEB TEDSAI COMPLEX
Document de Spécifications Complètes & Optimisées (PARTIE 2)
 
4. DESCRIPTION DÉTAILLÉE DES PAGES (Suite)
4.7 PAGE À PROPOS (Suite)
Section 2 : L'Équipe (Suite)
Exemples de Profils :
👤 [Photo]
Dr. [Nom Fondateur]
CEO & Co-fondateur
"Convaincu que l'IA doit servir l'humain, pas l'inverse.
Ex-data scientist, reconverti en entrepreneur social."

👤 [Photo]
Chef [Nom]
Directeur Culinaire viTEDia
"15 ans en gastronomie, passionné par les circuits courts.
Mon jardin est ma plus belle carte des vins."

👤 [Photo]
[Nom]
Responsable SelecTED Garden
"Agronome de formation, geek dans l'âme. J'optimise
la nature avec des capteurs et beaucoup d'amour."

👤 [Photo]
[Nom]
Lead Developer IA
"Je code des algorithmes qui changent vraiment la vie
des PME. Ça, c'est gratifiant."
Section 3 : Nos Valeurs
•	Layout : 5 blocs illustrés avec icônes 
1.	🎯 Impact Réel : "Nous ne faisons pas de l'IA pour l'IA. Chaque solution doit résoudre un problème concret."
2.	🌍 Durabilité Systémique : "De l'électricité du serveur aux déchets du restaurant, tout est pensé en cycle vertueux."
3.	🤝 Transparence Radicale : "Nos prix, nos process, nos données : nous n'avons rien à cacher."
4.	💡 Innovation Accessible : "La technologie de pointe ne doit pas être réservée aux grandes entreprises."
5.	❤️ Humain d'Abord : "L'IA remplace les tâches ingrates, pas les personnes. Nous créons des emplois, pas du chômage."
Section 4 : Nos Partenaires
•	Catégories : 
o	Partenaires Technologiques : Cloud providers, fournisseurs API IA
o	Partenaires Agricoles : Semenciers bio, coopératives locales
o	Partenaires Institutionnels : Chambres de commerce, incubateurs
o	Partenaires Médias : Publications qui soutiennent notre mission
•	Affichage : Logos en grille avec liens vers leurs sites
Section 5 : Presse & Médias
•	Titre : "Ils Parlent de Nous"
•	Format : Carousel d'articles 
o	Logo média
o	Titre article
o	Extrait (100 caractères)
o	Date publication
o	Lien "Lire l'article"
•	Kit Presse : CTA "Télécharger notre Kit Presse" (PDF avec logos HD, photos, communiqués)
Section 6 : Responsabilité Sociétale
•	Titre : "Notre Engagement RSE"
•	4 Piliers : 
1.	Économique : 25 emplois locaux, formation continue, salaires équitables
2.	Environnemental : Bilan carbone neutre 2025, zéro plastique, compostage total
3.	Social : Ateliers gratuits écoles, insertion jeunes décrocheurs
4.	Gouvernance : Transparence financière, mixité équipe (50% femmes)
Section 7 : Nous Rejoindre
•	Titre : "Rejoignez l'Aventure TEDSAI"
•	Texte d'Accroche : 
•	"Nous cherchons des talents passionnés qui veulent queleur travail ait du sens. Développeurs IA, chefs cuisiniers,maraîchers, commerciaux : si notre mission vous parle,parlons-nous."
•	Postes Ouverts : Liste dynamique (connexion ATS ou manuel) 
o	Développeur Full-Stack Python/React
o	Commercial Solutions IA B2B
o	Sous-Chef de Cuisine
o	Maraîcher Spécialisé Serres
•	CTA : "Candidature Spontanée" → Formulaire + upload CV
 
4.8 PAGE CONTACT INTELLIGENT
URL : /contact (FR) / /en/contact (EN)
Objectif : Faciliter la prise de contact qualifiée, router vers le bon interlocuteur, offrir plusieurs canaux.
Structure de la Page
Section 1 : Header Contact
•	Titre : "Parlons de Votre Projet"
•	Sous-titre : "Que vous cherchiez une solution IA, une réservation ou un partenariat, nous sommes là."
Section 2 : Formulaire Intelligent (Principal)
Design : Formulaire adaptatif en 2 colonnes (1 colonne mobile)
┌──────────────────────────────────────────────────────┐
│ Je suis intéressé par : [Dropdown obligatoire]       │
│ ○ Solution IA pour mon entreprise                    │
│ ○ Réservation viTEDia                                │
│ ○ Partenariat SelecTED Garden                        │
│ ○ Opportunité emploi                                 │
│ ○ Demande presse                                     │
│ ○ Autre demande                                      │
└──────────────────────────────────────────────────────┘

[CHAMPS DYNAMIQUES SELON LE CHOIX]

Si "Solution IA" sélectionné :
┌──────────────────────────────────────────────────────┐
│ Votre entreprise                                     │
│ Nom entreprise : [___________]                       │
│ Secteur d'activité : [Dropdown]                      │
│ Nombre d'employés : [1-10] [11-50] [51+]            │
│ Problématique principale : [Dropdown]                │
│   • Automatisation facturation                       │
│   • Gestion stocks                                   │
│   • Service client                                   │
│   • Analyse de données                               │
│   • Autre (préciser)                                 │
│                                                      │
│ Vos coordonnées                                      │
│ Nom complet : [___________]                          │
│ Email professionnel : [___________]                  │
│ Téléphone : [___________]                            │
│                                                      │
│ Votre message (optionnel)                            │
│ [Zone de texte 500 caractères max]                  │
│                                                      │
│ Budget estimé : [Dropdown]                           │
│   • < 1000€/mois                                     │
│   • 1000-3000€/mois                                  │
│   • 3000-10000€/mois                                 │
│   • > 10000€/mois                                    │
│   • À discuter                                       │
│                                                      │
│ Échéance projet : [Dropdown]                         │
│   • Urgent (< 1 mois)                                │
│   • Court terme (1-3 mois)                           │
│   • Moyen terme (3-6 mois)                           │
│   • Long terme (> 6 mois)                            │
│                                                      │
│ [✓] J'accepte d'être contacté par TEDSAI            │
│ [✓] J'accepte la politique de confidentialité       │
│                                                      │
│ [Envoyer ma Demande]                                 │
└──────────────────────────────────────────────────────┘

Si "Réservation viTEDia" sélectionné :
→ Redirection automatique vers page viTEDia/réservation
Ou widget réservation intégré directement

Si "Partenariat Garden" sélectionné :
┌──────────────────────────────────────────────────────┐
│ Type de partenariat :                                │
│ ○ Approvisionnement restaurant/épicerie              │
│ ○ Projet éducatif/scolaire                          │
│ ○ RSE entreprise                                     │
│ ○ Collaboration recherche                            │
│                                                      │
│ Nom organisation : [___________]                     │
│ Votre nom : [___________]                            │
│ Email : [___________]                                │
│ Téléphone : [___________]                            │
│ Décrivez votre projet : [Zone texte]                │
│                                                      │
│ [Envoyer]                                            │
└──────────────────────────────────────────────────────┘
Fonctionnalités Intelligentes :
•	Validation en temps réel : Email format, téléphone, champs requis
•	Autocomplétion : Noms d'entreprises connues
•	Anti-spam : reCAPTCHA invisible
•	Sauvegarde automatique : Brouillon conservé si l'utilisateur quitte
•	Confirmation visuelle : Animation checkmark + message "Reçu ! Réponse sous 24h"
Section 3 : Autres Canaux de Contact
Layout : 3 cartes côte à côte
Carte 1 : Assistance TED
💬 Chatbot Instantané
"TED peut répondre à 80% de vos questions
en moins de 2 minutes. Essayez !"

[Parler à TED]
Carte 2 : Prise de Rendez-vous
📅 Planifier un Appel
"Choisissez un créneau pour discuter
avec un expert (15-30 min)"

[Voir les Disponibilités] → Calendly
Carte 3 : Contact Direct
📞 Nous Appeler
+237 XXX XXX XXX

📧 Email Direct
contact@tedsai.cm
(Réponse sous 24h ouvrées)
Section 4 : Coordonnées & Localisation
Layout : 2 colonnes
Colonne Gauche : Infos Pratiques
📍 ADRESSE
TEDSAI Complex
[Rue], [Quartier]
Yaoundé, Cameroun

⏰ HORAIRES

TEDSAI IA (Bureau)
Lun-Ven : 8h00 - 18h00
Sam : Sur rendez-vous

viTEDia Restaurant
Mar-Dim : 12h00 - 15h00 / 18h30 - 22h30
Fermé lundis

SelecTED Garden (Visites)
Samedi : 10h00 - 12h00 (sur réservation)
Colonne Droite : Carte Interactive
•	Google Maps Embed : Localisation précise
•	Itinéraire : Lien "Obtenir l'itinéraire"
•	Street View : Vue de la façade
Section 5 : FAQ Rapide
•	Format : Accordéon 10 questions fréquentes 
o	"Combien coûte une solution IA ?"
o	"Peut-on privatiser viTEDia ?"
o	"Le jardin est-il ouvert au public ?"
o	"Livrez-vous les produits du jardin ?"
o	"Comment devenir partenaire ?"
o	"Quels modes de paiement acceptez-vous ?"
o	etc.
Section 6 : Réseaux Sociaux
•	Titre : "Suivez Nos Actualités"
•	Icônes Large : LinkedIn, Instagram, Facebook, Twitter
•	Feed Instagram : Widget affichant 6 derniers posts
 
4.9 PAGES SOLUTIONS IA PAR PROBLÉMATIQUE (Pages Dédiées)
URLs Exemples :
•	/solutions-ia/automatisation-facturation
•	/solutions-ia/gestion-stocks
•	/solutions-ia/service-client
•	/solutions-ia/analyse-predictive
•	/solutions-ia/optimisation-operations
Template Unifié pour Chaque Page :
Structure Page Problématique
Section 1 : Héro Spécifique
[Icône Grande Taille]

H1 : "Automatisation Intelligente de la Facturation"

Sous-titre : "Libérez 80% du temps de votre équipe comptable.
Zéro erreur, traçabilité parfaite, intégration ERP instantanée."

Stats Clé : "Économie moyenne : 1 200 heures/an pour une PME de 50 employés"

[CTA : Tester avec Mes Factures] [CTA : Voir une Démo Vidéo]
Section 2 : Le Problème en Détail
•	Titre : "Vous Reconnaissez-vous ?"
•	3-4 Pain Points : 
•	❌ Votre équipe passe des heures à saisir manuellement les factures fournisseurs❌ Les erreurs de saisie créent des litiges et retards de paiement❌ Le rapprochement bancaire est un cauchemar mensuel❌ Impossible d'avoir une vue temps réel de vos dépenses
Section 3 : Comment Notre IA Résout Cela
•	Schéma Visuel : Workflow "Avant / Après"
AVANT TEDSAI (Processus Manuel) :
Facture papier/PDF reçue
    ↓
Saisie manuelle comptable (30 min/facture)
    ↓
Vérification manager (10 min)
    ↓
Saisie dans ERP
    ↓
Rapprochement bancaire mensuel (8h)
    ↓
TOTAL : 40 min/facture + 8h/mois
APRÈS TEDSAI (IA Automatisée) :
Facture reçue (email/scan)
    ↓
IA TEDSAI extrait automatiquement (5 secondes)
    ↓
Validation optionnelle humaine (2 min si besoin)
    ↓
Intégration ERP automatique
    ↓
Rapprochement bancaire auto (temps réel)
    ↓
TOTAL : 2 min/facture + 0h de rapprochement
Fonctionnalités Détaillées :
✅ OCR Avancé Multilingue
   • Scanne PDF, photos, emails
   • Reconnaissance 99,8% précise
   • Support 15+ langues

✅ Extraction Intelligente
   • Fournisseur (nom, SIRET, adresse)
   • Montants (HT, TVA, TTC)
   • Dates (émission, échéance)
   • Références (commande, devis)
   • Lignes de détail produits

✅ Validation Automatique
   • Détection doublons (factures déjà saisies)
   • Alerte incohérences (prix anormaux)
   • Vérification TVA réglementaire
   • Matching avec bons de commande

✅ Intégrations ERP Natives
   • Sage (50, 100, X3)
   • QuickBooks
   • Odoo
   • SAP Business One
   • Cegid
   • API personnalisée

✅ Tableau de Bord Analytics
   • Vue temps réel des dépenses
   • Prévisions de trésorerie IA
   • Alertes échéances paiement
   • Export Excel/PDF rapports
Section 4 : Cas Client Détaillé
┌────────────────────────────────────────────────────┐
│ 🏢 CAS CLIENT : Supermarché "Chez Nous"           │
│                                                    │
│ PROFIL                                             │
│ • Secteur : Commerce de détail alimentaire        │
│ • Taille : 45 employés                            │
│ • Volume : 180 factures fournisseurs/mois         │
│                                                    │
│ PROBLÈME INITIAL                                   │
│ "Notre comptable passait 3 jours par semaine      │
│ uniquement sur la saisie factures. Les erreurs    │
│ créaient des tensions avec les fournisseurs."     │
│ - Marie D., Directrice Administrative             │
│                                                    │
│ SOLUTION TEDSAI DÉPLOYÉE                          │
│ • Module Automatisation Facturation               │
│ • Intégration avec Sage 100                       │
│ • Formation équipe : 1 journée                    │
│ • Mise en production : 10 jours                   │
│                                                    │
│ RÉSULTATS APRÈS 6 MOIS                            │
│ ✅ 94% de factures traitées automatiquement       │
│ ✅ Temps de saisie : -87% (3 jours → 4h/semaine)  │
│ ✅ Taux d'erreur : -95% (12% → 0,6%)              │
│ ✅ Délai moyen paiement : -18 jours               │
│ ✅ ROI : 340% (solution amortie en 2 mois)        │
│                                                    │
│ TÉMOIGNAGE                                         │
│ "L'IA de TEDSAI a transformé notre comptabilité.  │
│ Mon équipe se concentre enfin sur l'analyse, pas  │
│ la saisie. Et les fournisseurs sont payés à       │
│ temps, ce qui améliore nos relations."            │
│                                                    │
│ [Vidéo Témoignage 2 min]                          │
└────────────────────────────────────────────────────┘
Section 5 : Démonstration Interactive
•	Titre : "Voyez Par Vous-Même : Testez Avec Votre Facture"
•	Widget Upload : 
•	[Zone de Drag & Drop]"Glissez une facture PDF ou prenez une photo"Formats acceptés : PDF, JPG, PNGTaille max : 10 MBConfidentialité : Données traitées localement, non stockées[Ou Cliquez pour Parcourir]
•	Résultat Instantané (après 3-5s de traitement) : 
•	✅ Facture Analysée avec Succès !📊 DONNÉES EXTRAITES :Fournisseur : ABC Distribution SARLSIRET : 123 456 789 00012Montant HT : 1 450,00 €TVA (20%) : 290,00 €Montant TTC : 1 740,00 €Date émission : 10/12/2024Date échéance : 25/01/2025Référence : FACT-2024-5678[Télécharger le Rapport Complet PDF][Planifier une Démo Personnalisée]
Section 6 : Tarification Transparente
•	Package Starter : 349€/mois 
o	Jusqu'à 50 factures/mois
o	1 utilisateur
o	Intégration 1 ERP
o	Support email
•	Package Business : 899€/mois 
o	Jusqu'à 200 factures/mois
o	5 utilisateurs
o	Intégrations illimitées
o	Support prioritaire
o	Analytics avancés
•	Package Enterprise : Sur devis 
o	Factures illimitées
o	Utilisateurs illimités
o	API complète
o	SLA 99,9%
o	Gestionnaire dédié
Section 7 : FAQ Spécifique
•	15 questions ciblées sur cette problématique : 
o	"Quels formats de factures sont supportés ?"
o	"Que se passe-t-il si l'IA se trompe ?"
o	"Combien de temps pour l'intégration avec mon ERP ?"
o	"Mes données sont-elles sécurisées ?"
o	"Puis-je annuler à tout moment ?"
o	etc.
Section 8 : Ressources Complémentaires
•	📄 Fiche Technique PDF (2 pages)
•	📊 Étude de Cas Complète PDF
•	🎥 Webinar Replay "Automatiser sa Facturation en 2024"
•	📘 Livre Blanc "Le Guide Complet de l'Automatisation Comptable"
CTA Final Puissant :
[Grand Bandeau Gradient Bleu]

"Prêt à Économiser 1 200 Heures/An ?"

[Démarrer Mon Essai Gratuit 14 Jours]
[Parler à un Expert Comptabilité]

✓ Sans carte bancaire  ✓ Sans engagement  ✓ Setup en 24h
 
4.10 PAGES SOLUTIONS IA PAR SECTEUR (Pages Dédiées)
URLs Exemples :
•	/solutions-ia/commerce-retail
•	/solutions-ia/restauration-hotellerie
•	/solutions-ia/agriculture-agroalimentaire
•	/solutions-ia/sante-services
Template Structure (Exemple : Restauration & Hôtellerie)
Page Solutions IA - Restauration & Hôtellerie
Section 1 : Héro Sectoriel
H1 : "IA Pour Restaurants & Hôtels :
De la Réservation à la Fidélisation"

Sous-titre : "Optimisez vos stocks, automatisez vos réservations,
personnalisez l'expérience client. L'IA qui comprend l'hospitalité."

Stats : "Nos clients restaurateurs économisent en moyenne
18h/semaine de gestion administrative"

[CTA : Voir Les Solutions] [CTA : Cas Client Restaurant]
Section 2 : Les 5 Défis du Secteur
🍽️ Défis Restaurants & Hôtels

1. GESTION DES STOCKS
   "40% du gaspillage vient d'une mauvaise prévision des besoins"

2. RÉSERVATIONS & NO-SHOWS
   "15-20% des réservations ne se présentent pas, impactant le CA"

3. SERVICE CLIENT
   "Répondre à 100+ demandes/jour (menu, allergies, horaires...)"

4. GESTION DU PERSONNEL
   "Planning complexe avec pics imprévisibles"

5. ANALYSE PERFORMANCES
   "Difficile d'identifier les plats rentables vs les boulets"
Section 3 : Nos Solutions Sectorielles
Solution 1 : Prédiction Intelligente des Stocks
🎯 PROBLÈME RÉSOLU
"Combien de poulets commander cette semaine ?"

💡 L'IA TEDSAI ANALYSE
• Historique de ventes (2 ans)
• Saisonnalité & météo
• Événements locaux (concerts, matchs)
• Réservations confirmées
• Tendances social media

📊 PRÉDICTION PRÉCISE
"Cette semaine : 45 kg poulet, 30 kg tomates, 12 kg basilic"
Précision : 94% (vs 67% estimation humaine)

💰 RÉSULTAT
• -35% gaspillage alimentaire
• -20% coût achats
• Fraîcheur maximale garantie
Solution 2 : Assistant Réservation IA
🤖 CHATBOT INTELLIGENT

Gère automatiquement :
✓ Réservations 24/7 (site web, Facebook, Instagram)
✓ Modifications & annulations
✓ Rappels automatiques (-80% no-shows)
✓ Suggestions upsell ("Champagne pour l'occasion ?")
✓ Liste d'attente intelligente

📱 MULTICANAL
• Site web
• WhatsApp Business
• Messenger
• Téléphone (vocal IA)

🎯 CONVERSION
Taux de réservation : +40% vs formulaire classique
Solution 3 : Service Client 360°
💬 IA CONVERSATIONNELLE

Répond instantanément à :
• "Avez-vous des plats sans gluten ?"
• "Quelle est l'origine de vos produits ?"
• "Pouvez-vous accommoder 12 personnes samedi ?"
• "Menu enfant disponible ?"
• "Parking à proximité ?"

🧠 PERSONNALISATION
Reconnaît les clients réguliers :
"Bonsoir M. Dupont ! Votre table habituelle en terrasse ?"

📈 LIBÈRE L'ÉQUIPE
70% des demandes traitées sans intervention humaine
Solution 4 : Analytics & Business Intelligence
📊 TABLEAUX DE BORD TEMPS RÉEL

KPIs Suivis :
• Taux de remplissage par créneau
• Panier moyen par client
• Plats les plus rentables
• Performance serveurs
• Satisfaction client (avis agrégés)

🔮 PRÉDICTIONS
• "Samedi prochain : affluence +30%, commandez plus"
• "Le risotto aux cèpes a un potentiel +15%, mettez-le en avant"
• "Période creuse 15h-18h : lancez une offre happy hour"

💡 RECOMMANDATIONS ACTIONABLES
"Votre coût matière sur la salade César est trop élevé (58%).
Suggestion : remplacer les crevettes par du poulet du jardin."
Section 4 : Cas Client Secteur
🏨 CAS CLIENT : Hôtel-Restaurant "Le Palmier d'Or"

PROFIL
• 25 chambres + Restaurant 60 couverts
• Équipe : 18 personnes
• Yaoundé, clientèle business + tourisme

DÉFIS
❌ Stocks : Gaspillage 30% sur produits frais
❌ Réservations : 40% par téléphone (chronophage)
❌ Service client : Manager débordé par les demandes
❌ Analyse : Aucune visibilité sur la rentabilité réelle

SOLUTIONS TEDSAI DÉPLOYÉES
✅ Module Prédiction Stocks
✅ Chatbot Réservations Multicanal
✅ Assistant Client IA
✅ Dashboard Analytics

RÉSULTATS APRÈS 1 AN
🎯 Gaspillage : -42% (de 30% à 17%)
🎯 Réservations online : +65%
🎯 No-shows : -60% (grâce aux rappels auto)
🎯 Temps service client : -12h/semaine libérées
🎯 Chiffre d'affaires : +18%
🎯 ROI : 520% (investissement amorti en 3 mois)

TÉMOIGNAGE
"TEDSAI a révolutionné notre gestion. On a enfin le temps
de se concentrer sur l'essentiel : l'expérience client."
- Patrick M., Directeur

[Lire le Cas Complet PDF]
Section 5 : Intégrations Sectorielles
🔌 COMPATIBLE AVEC VOS OUTILS

Systèmes de Caisse :
• Lightspeed
• Toaster
• Square
• Clover

Gestion Réservations :
• OpenTable
• Resy
• TheFork
• Google Réserve

Fournisseurs :
• Metro
• Sysco
• GrubMarket

Comptabilité :
• Sage
• QuickBooks
• Cegid
Section 6 : Tarification Restauration
•	Package adapté au secteur avec pricing spécifique
•	Essai gratuit 30 jours pour restaurants
•	Pas de frais d'installation
•	Formation équipe incluse
CTA Final :
"Transformez Votre Restaurant Avec l'IA"

[Démo Personnalisée Gratuite]
[Télécharger le Guide Restaurateur]

🍴 100+ restaurants nous font confiance
⭐ 4.9/5 satisfaction client
 
5. L'ASSISTANT TED - SPÉCIFICATIONS COMPLÈTES
5.1 Architecture Technique
Stack Technologique :
•	LLM Backend : GPT-4 Turbo (OpenAI) ou Claude 3 (Anthropic)
•	Framework : LangChain pour orchestration agents
•	Base Connaissances : Vector Database (Pinecone ou Weaviate)
•	APIs Intégrées : 
o	Base de données restaurant (menu, disponibilités)
o	Base traçabilité garden (produits, parcelles)
o	CRM (leads, historique clients)
o	Calendly (prise RDV)
o	Stripe (paiements)
5.2 Modes de Fonctionnement 
Mode 2 : Expert IA B2B (Pages Solutions IA)
TED : "Bonjour ! Je vois que vous consultez nos solutions IA.

Pour vous orienter au mieux, quelques questions rapides :

1. Quel est votre secteur d'activité ?
   [Commerce] [Restaurant] [Services] [Agriculture] [Autre]

2. Quelle est votre problématique principale ?
   [Facturation] [Stocks] [Service client] [Analyse données] [Autre]

3. Taille de votre entreprise ?
   [1-10] [11-50] [51-200] [200+]

Ou préférez-vous que je vous montre directement nos solutions ?
[Voir les solutions] [Parler à un expert]
Réponses Contextuelles Avancées :
Utilisateur : "Je perds beaucoup de temps sur la facturation"

TED : "Je comprends parfaitement. La facturation manuelle est 
effectivement chronophage.

📊 En moyenne, nos clients passaient 15h/semaine sur ce processus.

Notre solution d'automatisation peut :
✅ Scanner et extraire les données automatiquement (99,8% précision)
✅ Intégrer directement dans votre ERP
✅ Réduire de 87% le temps de traitement

Voulez-vous :
1. [Tester avec une de vos factures] (2 minutes)
2. [Voir un cas client similaire]
3. [Calculer votre ROI potentiel]
4. [Parler à un expert maintenant]
Mode 3 : Conciergerie Restaurant (Pages viTEDia)
TED : "Bienvenue chez viTEDia ! 😊

Je peux vous aider à :
🍽️ Réserver une table
📋 Découvrir le menu du jour
🌱 En savoir plus sur la traçabilité
🎁 Organiser un événement privé
📦 Commander en Click & Collect

Que souhaitez-vous faire ?"
Conversation Réservation Naturelle :
Utilisateur : "Je voudrais réserver pour ce soir"

TED : "Parfait ! Pour combien de personnes ?"

Utilisateur : "4 personnes"

TED : "Excellent. À quelle heure préférez-vous dîner ?
Nous avons des disponibilités à :
• 18h30 ✅
• 19h00 ✅
• 19h30 ✅
• 20h00 ✅
• 20h30 ⚠️ (dernière table)
• 21h00 ❌ Complet"

Utilisateur : "19h30"

TED : "Parfait pour 19h30. Préférez-vous :
🌿 La terrasse avec vue sur le jardin
🏠 L'intérieur cosy et climatisé"

Utilisateur : "Terrasse"

TED : "Excellent choix ! Pour finaliser :

📅 Mercredi 13 décembre
⏰ 19h30
👥 4 personnes
🌿 Terrasse

Pourriez-vous me donner :
• Votre nom
• Votre numéro de téléphone
• Votre email (pour la confirmation)"

[Après collecte des informations]

TED : "✅ Votre réservation est confirmée !

Récapitulatif :
• M./Mme [Nom]
• 4 personnes, terrasse
• Mercredi 13 déc. à 19h30

Vous recevrez :
📧 Un email de confirmation immédiatement
📱 Un SMS de rappel la veille

Des allergies ou préférences alimentaires à signaler ?

À très bientôt chez viTEDia ! 🍽️"
Mode 4 : Guide Traçabilité (Pages SelecTED Garden)
TED : "Bonjour ! Curieux de savoir d'où vient votre nourriture ? 🌱

Je peux vous montrer le parcours complet de n'importe quel 
produit de notre jardin.

Comment souhaitez-vous explorer ?
📱 [Scanner un QR code]
🔍 [Entrer un code produit]
🍅 [Choisir un produit]
🌿 [Visite virtuelle du jardin]
Exemple Conversation Traçabilité :
Utilisateur : "D'où viennent les tomates du menu ?"

TED : "Excellente question ! Nos tomates viennent directement 
de notre jardin SelecTED Garden, à 50 mètres du restaurant.

🍅 TOMATES DISPONIBLES AUJOURD'HUI :

1. Tomate Cœur de Bœuf (Serre 2, Parcelle B3)
   • Plantée : 15 septembre 2024
   • Récoltée : Ce matin à 8h00
   • 0 pesticide, culture biologique

2. Tomate Cerise Multicolore (Serre 1, Parcelle A5)
   • Plantée : 1er octobre 2024
   • Récoltée : Hier à 16h30
   • Variétés anciennes

Souhaitez-vous :
[Voir le parcours détaillé] [Photo de la parcelle] 
[Visite virtuelle] [Réserver avec ces tomates]
Mode 5 : Support Technique & FAQ
TED : "Une question ? Je suis là pour vous aider ! 

💡 Questions fréquentes :
• Prix et packages solutions IA
• Processus d'implémentation
• Sécurité des données
• Compatibilité avec votre système actuel
• Horaires et localisation
• Politique d'annulation

Ou posez-moi votre question directement."
5.3 Capacités Conversationnelles Avancées
Compréhension Contextuelle
Mémorisation de Session :
Conversation 1 (10h00) :
Utilisateur : "Combien coûte l'automatisation de facturation ?"
TED : [Donne les prix]

Conversation 2 (10h15, même utilisateur) :
Utilisateur : "Et pour la gestion de stocks ?"
TED : "Pour la gestion de stocks, c'est 899€/mois dans le package 
Business. Couplé avec l'automatisation facturation que vous 
regardiez tout à l'heure, vous pourriez bénéficier du package 
combiné à 1 299€/mois (au lieu de 1 248€ séparément).

Voulez-vous que je vous montre comment ces deux solutions 
fonctionnent ensemble ?"
Gestion des Objections
Utilisateur : "C'est trop cher"

TED : "Je comprends votre préoccupation sur le budget. 
Regardons ensemble le retour sur investissement :

📊 VOTRE SITUATION ACTUELLE (estimée)
• Temps facturation manuelle : 15h/semaine
• Coût horaire : 25€/h
• Coût mensuel actuel : 1 500€

💡 AVEC TEDSAI
• Investissement : 349€/mois
• Temps réduit à : 2h/semaine
• Nouveau coût : 200€ + 349€ = 549€/mois

💰 ÉCONOMIE NETTE : 951€/mois

Le système s'autofinance en moins d'1 mois. Après, c'est 
du bénéfice pur.

Souhaitez-vous un calcul personnalisé avec vos vrais chiffres ?
[Calculer mon ROI] [Parler à un conseiller] [Voir un cas similaire]"
Multi-langue Intelligent
Détection automatique de la langue :

User: "How much for the AI solution?"
TED: "Hello! I can help you in English. 

Our AI solutions start at 499€/month for the Starter package. 
Would you like me to explain the features in detail?

[View Packages] [Calculate ROI] [Schedule Demo]"

Utilisateur: "Préférez-vous français?"
TED: "Bien sûr ! Je passe en français. 😊"
5.4 Intégrations Backend
Connexions API Essentielles
1. Base de Données Réservations
# Vérification disponibilité en temps réel
def check_availability(date, time, guests):
    api_call = f"/api/reservations/check?date={date}&time={time}&guests={guests}"
    response = restaurant_api.get(api_call)
    return response.available_slots

# Création réservation
def create_reservation(details):
    api_call = "/api/reservations/create"
    response = restaurant_api.post(api_call, data=details)
    send_confirmation_email(details.email)
    send_sms_reminder(details.phone, details.date)
    return response.reservation_id
2. Base Traçabilité Jardin
# Récupération historique produit
def get_product_traceability(product_code):
    api_call = f"/api/traceability/{product_code}"
    response = garden_api.get(api_call)
    return {
        'origin': response.parcelle,
        'planting_date': response.date_plantation,
        'harvest_date': response.date_recolte,
        'treatments': response.traitements,
        'iot_data': response.donnees_capteurs,
        'photos': response.photos_parcelle
    }
3. CRM & Lead Management
# Qualification lead automatique
def qualify_lead(conversation_data):
    lead_score = calculate_score({
        'sector': conversation_data.sector,
        'budget': conversation_data.budget,
        'urgency': conversation_data.timeline,
        'company_size': conversation_data.employees
    })
    
    if lead_score >= 80:
        notify_sales_team("hot_lead", conversation_data)
        schedule_call_within_hours(2)
    elif lead_score >= 50:
        add_to_nurturing_campaign(conversation_data)
    
    save_to_crm(conversation_data, lead_score)
4. Calendly / Prise de RDV
# Proposition créneaux disponibles
def get_available_slots(service_type):
    if service_type == "demo_ia":
        calendar = "sales_team_calendar"
        duration = 30
    elif service_type == "audit_gratuit":
        calendar = "expert_calendar"
        duration = 60
    
    slots = calendly_api.get_slots(calendar, next_7_days, duration)
    return format_slots_for_user(slots)
5.5 Design Interface TED
Widget Chat (Position & Style)
Position Desktop :
.ted-widget {
    position: fixed;
    bottom: 24px;
    right: 24px;
    z-index: 9999;
}

.ted-bubble {
    width: 64px;
    height: 64px;
    border-radius: 50%;
    background: linear-gradient(135deg, #0A2463 0%, #5AA9E6 100%);
    box-shadow: 0 4px 16px rgba(10, 36, 99, 0.3);
    cursor: pointer;
    display: flex;
    align-items: center;
    justify-content: center;
    transition: all 300ms ease;
}

.ted-bubble:hover {
    transform: scale(1.1);
    box-shadow: 0 6px 24px rgba(10, 36, 99, 0.4);
}

.ted-bubble-icon {
    font-size: 32px;
}

/* Animation pulse pour attirer l'attention */
@keyframes pulse-ted {
    0%, 100% { box-shadow: 0 4px 16px rgba(10, 36, 99, 0.3); }
    50% { box-shadow: 0 4px 24px rgba(90, 169, 230, 0.6); }
}

.ted-bubble.pulsing {
    animation: pulse-ted 2s infinite;
}
Fenêtre Chat Ouverte :
.ted-chat-window {
    position: fixed;
    bottom: 100px;
    right: 24px;
    width: 400px;
    height: 600px;
    background: white;
    border-radius: 16px;
    box-shadow: 0 8px 32px rgba(0, 0, 0, 0.15);
    display: flex;
    flex-direction: column;
    overflow: hidden;
    animation: slideUp 300ms ease;
}

@keyframes slideUp {
    from {
        opacity: 0;
        transform: translateY(20px);
    }
    to {
        opacity: 1;
        transform: translateY(0);
    }
}

/* Header */
.ted-header {
    background: linear-gradient(135deg, #0A2463 0%, #5AA9E6 100%);
    color: white;
    padding: 16px;
    display: flex;
    align-items: center;
    justify-content: space-between;
}

.ted-avatar {
    width: 40px;
    height: 40px;
    border-radius: 50%;
    background: white;
    display: flex;
    align-items: center;
    justify-content: center;
    font-size: 24px;
}

.ted-info h3 {
    margin: 0;
    font-size: 16px;
    font-weight: 600;
}

.ted-status {
    font-size: 12px;
    opacity: 0.9;
}

.ted-status::before {
    content: "●";
    color: #4ade80;
    margin-right: 4px;
}

/* Zone messages */
.ted-messages {
    flex: 1;
    overflow-y: auto;
    padding: 16px;
    background: #F5F5F7;
}

.ted-message {
    margin-bottom: 12px;
    display: flex;
    animation: fadeIn 300ms ease;
}

@keyframes fadeIn {
    from { opacity: 0; transform: translateY(10px); }
    to { opacity: 1; transform: translateY(0); }
}

.ted-message.assistant {
    justify-content: flex-start;
}

.ted-message.user {
    justify-content: flex-end;
}

.message-bubble {
    max-width: 75%;
    padding: 12px 16px;
    border-radius: 16px;
    font-size: 14px;
    line-height: 1.5;
}

.message-bubble.assistant {
    background: white;
    color: #1A1A2E;
    border-bottom-left-radius: 4px;
}

.message-bubble.user {
    background: #0A2463;
    color: white;
    border-bottom-right-radius: 4px;
}

/* Boutons rapides */
.quick-actions {
    display: flex;
    flex-wrap: wrap;
    gap: 8px;
    margin-top: 8px;
}

.quick-action-btn {
    background: white;
    border: 1px solid #0A2463;
    color: #0A2463;
    padding: 8px 16px;
    border-radius: 20px;
    font-size: 13px;
    cursor: pointer;
    transition: all 200ms ease;
}

.quick-action-btn:hover {
    background: #0A2463;
    color: white;
}

/* Input zone */
.ted-input-area {
    padding: 16px;
    background: white;
    border-top: 1px solid #e5e7eb;
}

.input-wrapper {
    display: flex;
    gap: 8px;
    align-items: center;
}

.ted-input {
    flex: 1;
    border: 1px solid #D1D5DB;
    border-radius: 24px;
    padding: 10px 16px;
    font-size: 14px;
    outline: none;
    transition: border 200ms ease;
}

.ted-input:focus {
    border-color: #5AA9E6;
}

.send-btn {
    width: 40px;
    height: 40px;
    border-radius: 50%;
    background: #0A2463;
    color: white;
    border: none;
    cursor: pointer;
    display: flex;
    align-items: center;
    justify-content: center;
    transition: all 200ms ease;
}

.send-btn:hover {
    background: #5AA9E6;
    transform: scale(1.05);
}

.send-btn:disabled {
    background: #D1D5DB;
    cursor: not-allowed;
}
Version Mobile :
@media (max-width: 768px) {
    .ted-chat-window {
        bottom: 0;
        right: 0;
        left: 0;
        width: 100%;
        height: 100vh;
        border-radius: 0;
    }
    
    .ted-bubble {
        bottom: 16px;
        right: 16px;
        width: 56px;
        height: 56px;
    }
}
5.6 Scénarios d'Usage Complets
Scénario 1 : Conversion Lead B2B
[Utilisateur arrive sur /solutions-ia/automatisation-facturation]

TED (après 5 secondes) : "👋 Bonjour ! Je vois que vous 
vous intéressez à l'automatisation de la facturation.

Puis-je vous poser 2 questions rapides pour vous orienter 
au mieux ?"

Utilisateur : "Ok"

TED : "Parfait ! 

1️⃣ Combien de factures traitez-vous par mois environ ?
[0-50] [51-200] [201-500] [500+]"

Utilisateur : [Clic 51-200]

TED : "Merci ! 

2️⃣ Quel est votre principal défi actuellement ?
[Temps de saisie] [Taux d'erreur] [Suivi paiements] [Autre]"

Utilisateur : [Clic Temps de saisie]

TED : "Je comprends parfaitement. Avec 51-200 factures/mois, 
vous passez probablement 20-30 heures/mois sur la saisie. 💼

Bonne nouvelle : nos clients dans votre situation 
économisent en moyenne 85% de ce temps.

Voulez-vous :
1. [Voir comment ça marche] (vidéo 2 min)
2. [Tester avec une de vos factures] (démo 3 min)
3. [Calculer mes économies exactes] (calculateur)
4. [Parler à un expert maintenant] (RDV sous 24h)"

Utilisateur : [Clic Tester avec une de vos factures]

TED : "Excellent choix ! Vous allez voir la magie en direct. 🎩

📄 Glissez ici une facture PDF ou prenez-la en photo :
[Zone upload]

Confidentialité : Votre document est traité localement, 
jamais stocké sur nos serveurs."

[Utilisateur upload facture]

TED : "⏳ Analyse en cours... (3 secondes)

✅ Facture analysée avec succès !

📊 DONNÉES EXTRAITES AUTOMATIQUEMENT :
• Fournisseur : ABC Distribution SARL
• N° SIRET : 123 456 789 00012
• Montant HT : 2 450,00 €
• TVA (20%) : 490,00 €
• Total TTC : 2 940,00 €
• Date émission : 10/12/2024
• Échéance : 09/01/2025
• Référence : FACT-2024-8765

⚡ Temps de traitement : 3,2 secondes
(vs 15-20 minutes en saisie manuelle)

Ce résultat vous convainc ? 

💰 Pour 150 factures/mois comme la vôtre :
• Vous économisez : 37,5 heures/mois
• Soit : 450 heures/an
• Valeur : ~11 250€/an (à 25€/h)

Notre solution coûte 899€/mois.
ROI : Rentabilisée en moins de 1 mois ! 

Prochaine étape ?
[Démarrer essai gratuit 14j] [Planifier démo personnalisée] 
[Télécharger rapport complet]"

Utilisateur : [Clic Planifier démo]

TED : "Parfait ! Je vous connecte avec un de nos experts 
en automatisation comptable. 

📅 Choisissez un créneau qui vous convient :

CETTE SEMAINE :
• Jeudi 14 déc - 10h00 ✅
• Jeudi 14 déc - 14h30 ✅
• Vendredi 15 déc - 11h00 ✅
• Vendredi 15 déc - 16h00 ✅

SEMAINE PROCHAINE :
• Lundi 18 déc - 09h00 ✅
• Lundi 18 déc - 15h00 ✅
[Voir plus de créneaux]"

Utilisateur : [Sélectionne Jeudi 14 déc - 10h00]

TED : "Excellent ! Dernière étape, vos coordonnées :

Nom complet : [_________]
Email professionnel : [_________]
Téléphone : [_________]
Nom de votre entreprise : [_________]

[Confirmer mon RDV]"

[Après validation]

TED : "✅ Votre démonstration est confirmée !

📅 Jeudi 14 décembre à 10h00
👤 Avec Marc, Expert Automatisation
⏱️ Durée : 30 minutes

Vous allez recevoir :
📧 Email de confirmation (avec lien visio)
📱 SMS de rappel la veille
📄 Guide préparatoire pour profiter au max de la démo

En attendant, puis-je vous aider avec autre chose ?
[Explorer autres solutions] [Questions sur TEDSAI] [C'est bon merci]"

[Lead enregistré dans CRM avec score : 90/100 - HOT LEAD]
[Notification envoyée à l'équipe commerciale]
[Email nurturing programmé]
Scénario 2 : Réservation Restaurant Complexe
[Utilisateur sur page viTEDia]

TED : "Bienvenue chez viTEDia ! 🍽️
Puis-je vous aider à réserver une table ?"

Utilisateur : "Oui, pour samedi soir"

TED : "Parfait ! Samedi 16 décembre. 
Pour combien de personnes ?"

Utilisateur : "On sera 8"

TED : "Super ! Pour 8 personnes samedi soir. 

⚠️ Pour les groupes de 6+ personnes, je recommande de 
réserver entre 19h00 et 20h00 pour avoir l'espace optimal.

Quelle heure préférez-vous ?
[18h30] [19h00] [19h30] [20h00] [20h30]"

Utilisateur : "19h30 c'est bien"

TED : "Excellent choix ! 19h30 pour 8 personnes.

Pour cette taille de groupe, j'ai deux options :

🌿 TERRASSE (Vue jardin)
• 1 grande table de 8
• Ambiance plein air
• Chauffage extérieur disponible

🏠 SALON PRIVÉ (Intérieur)
• Espace semi-privé
• Climatisé
• Plus intime

Votre préférence ?"

Utilisateur : "Le salon privé me tente"

TED : "Excellent ! Le salon privé est parfait pour les 
groupes. Vous aurez un espace semi-isolé très convivial.

🎉 S'agit-il d'une occasion spéciale ?
[Anniversaire] [Repas d'affaires] [Famille] 
[Juste un dîner] [Autre]"

Utilisateur : [Clic Anniversaire]

TED : "Ah, un anniversaire ! 🎂

Nous adorons célébrer nos clients. Quelques petites attentions :

🎁 Ce que nous offrons :
• Dessert d'anniversaire surprise ✨
• Photo souvenir avec le chef
• Coin décoré avec ballons

💡 Ce que nous pouvons ajouter (en option) :
• Bouteille de champagne (45€)
• Bouquet de fleurs (25€)
• Gâteau personnalisé (35€)

Souhaitez-vous ajouter quelque chose ?
[Champagne + Fleurs] [Juste le champagne] 
[Juste les fleurs] [Non merci]"

Utilisateur : "Juste les fleurs"

TED : "Parfait ! Un joli bouquet sera sur votre table. 🌸

🍽️ Dernière question : Des allergies ou régimes particuliers 
dans le groupe ?
(végétarien, sans gluten, allergie fruits à coque, etc.)"

Utilisateur : "1 personne végétarienne et 1 sans gluten"

TED : "Noté ! Notre chef préparera des options adaptées. 
Nous avons d'excellents plats végétariens et toutes nos 
alternatives sans gluten.

📝 RÉCAPITULATIF :
• Samedi 16 décembre à 19h30
• 8 personnes
• Salon Privé (intérieur)
• Occasion : Anniversaire 🎂
• Option : Bouquet de fleurs (+25€)
• Régimes : 1 végétarien, 1 sans gluten

Prix bouquet : 25€ (à régler sur place)

Pour confirmer, vos coordonnées :

Nom : [_________]
Téléphone : [_________]
Email : [_________]

[Confirmer la Réservation]"

[Après validation]

TED : "✅ Réservation confirmée avec succès !

🎉 VOTRE SOIRÉE CHEZ viTEDia

📅 Samedi 16 décembre 2024
⏰ 19h30 (merci d'arriver 5-10 min avant)
👥 8 personnes
📍 Salon Privé
🌸 Bouquet inclus
🎂 Dessert d'anniversaire offert

📧 Email de confirmation envoyé
📱 SMS de rappel vendredi soir

💡 CONSEILS POUR PROFITER AU MAX :
• Arrivez 10 min avant pour profiter d'un apéritif
• Consultez notre menu du jour en ligne la veille
• Découvrez la traçabilité de chaque plat (QR codes)

Des questions en attendant ?
[Voir le menu] [Modifier réservation] 
[Politique annulation] [Tout est parfait !]"

Utilisateur : "Voir le menu"

TED : "Voici notre menu actuel (mis à jour quotidiennement) :
[Affichage menu]

⚠️ Note : Le menu de samedi sera mis à jour vendredi soir 
avec les produits frais du jardin récoltés dans la journée.

Je vous enverrai un email vendredi soir avec le menu exact !

Autre chose ? 😊"
Scénario 3 : Exploration Traçabilité
[Utilisateur scan QR code sur table restaurant depuis son smartphone]

TED : "Bonjour ! 👋

Vous venez de scanner le QR code de votre table. 
Que souhaitez-vous découvrir ?

🍅 [Traçabilité de mon plat]
🌱 [Visite virtuelle du jardin]
📊 [Origine de tous les ingrédients]
🎥 [Vidéo : du jardin à l'assiette]"

Utilisateur : [Clic Traçabilité de mon plat]

TED : "Parfait ! Quel plat souhaitez-vous tracer ?

🍽️ VOTRE COMMANDE AUJOURD'HUI :
1. Velouté de Potimarron
2. Risotto aux Cèpes
3. Tarte Tatin Pommes

[Ou entrez le code produit de votre ticket]"

Utilisateur : [Clic Risotto aux Cèpes]

TED : "Excellent choix ! Voici l'histoire complète de 
votre Risotto aux Cèpes 🍄

━━━━━━━━━━━━━━━━━━━━━━━━

🌾 RIZ CARNAROLI
📍 Origine : Fournisseur BioDom, Douala
🌱 Culture : Biologique certifiée
📅 Récolte : Octobre 2024
✈️ Distance : 180 km

🍄 CÈPES FRAIS
📍 Origine : SelecTED Garden, Parcelle F1
🌱 Culture : Serre climatisée #3
📅 Récolte : Hier, 11 décembre, 16h30
🚚 Distance : 50 mètres (jardin → cuisine)
📸 [Voir photo de la parcelle]

🧈 BEURRE & PARMESAN
📍 Origine : Ferme Laitière du Moungo
🌱 Vaches élevées en plein air
📅 Production : Cette semaine
✈️ Distance : 95 km

🌿 HERBES (Persil, Thym)
📍 Origine : SelecTED Garden, Parcelle H2
🌱 Culture : Plein air, zéro pesticide
📅 Cueillette : Ce matin, 7h00
🚚 Distance : 50 mètres

━━━━━━━━━━━━━━━━━━━━━━━━

📊 BILAN TRAÇABILITÉ :
✅ 60% du plat : Notre jardin (< 100m)
✅ 40% : Fournisseurs locaux certifiés bio
✅ Distance moyenne : 108 km
✅ CO2 économisé vs circuit classique : 2,3 kg

🧑‍🍳 PRÉPARÉ PAR :
Chef Marie L. - 15 ans d'expérience
Cuisson : Aujourd'hui, 12h15

━━━━━━━━━━━━━━━━━━━━━━━━

Impressionnant, non ? 

[📸 Télécharger le certificat]
[🌱 Voir la parcelle en 360°]
[📤 Partager sur Instagram]
[🍽️ Commander ce plat]"

Utilisateur : [Clic Voir la parcelle en 360°]

TED : "🌱 VISITE VIRTUELLE - Serre #3, Parcelle F1

[Viewer 360° interactif s'ouvre]

Vous êtes maintenant dans notre serre où ont poussé 
vos cèpes ! Utilisez votre doigt pour explorer.

💡 POINTS D'INTÉRÊT :
🔴 Zone de culture des cèpes
🔴 Capteurs IoT (température, humidité)
🔴 Système d'irrigation automatique
🔴 Notre maraîcher Jean au travail

📊 CONDITIONS ACTUELLES (en direct) :
• Température : 18°C
• Humidité : 75%
• Dernière irrigation : Il y a 3h

[Cliquez sur les points rouges pour en savoir plus]

Voulez-vous :
[🎥 Voir la récolte en vidéo]
[📅 Réserver une visite du jardin]
[🛒 Commander un panier hebdomadaire]
[↩️ Retour au risotto]"
 
6. FONCTIONNALITÉS TRANSVERSALES
6.1 Système de Recherche Global
Barre de Recherche Intelligente
Position : Header du site, visible sur toutes les pages
Design :
.search-bar {
    position: relative;
    max-width: 600px;
    margin: 0 auto;
}

.search-input {
    width: 100%;
    padding: 12px 48px 12px 20px;
    border: 2px solid #D1D5DB;
    border-radius: 24px;
    font-size: 15px;
    transition: all 300ms ease;
}

.search-input:focus {
    border-color: #5AA9E6;
    box-shadow: 0 4px 12px rgba(90, 169, 230, 0.2);
}

.search-icon {
    position: absolute;
    right: 16px;
    top: 50%;
    transform: translateY(-50%);
    color: #6B7280;
}
Fonctionnalités :
1.	Autocomplétion Intelligente
Utilisateur tape : "auto"

Suggestions affichées :
🔷 Solutions IA
   • Automatisation de la facturation
   • Automatisation des stocks

🍽️ viTEDia
   • Réservation automatique

🌱 SelecTED Garden
   • Irrigation automatisée

📚 Articles Blog
   • "5 processus à automatiser en priorité"
   • "Comment automatiser sa comptabilité"
2.	Recherche Multi-Critères
function smartSearch(query) {
    const results = {
        solutions: searchInSolutions(query),
        menu: searchInMenu(query),
        products: searchInGarden(query),
        blog: searchInBlog(query),
        faq: searchInFAQ(query)
    };
    
    // Scoring selon pertinence
    return rankResults(results, query);
}
3.	Recherche Vocale
<button class="voice-search-btn" onclick="startVoiceSearch()">
    🎤
</button>

<script>
function startVoiceSearch() {
    const recognition = new webkitSpeechRecognition();
    recognition.lang = 'fr-FR';
    recognition.onresult = (event) => {
        const query = event.results[0][0].transcript;
        performSearch(query);
    };
    recognition.start();
}
</script>
Page Résultats :
┌────────────────────────────────────────────────────┐
│ Résultats pour "facturation automatique"          │
│ 24 résultats trouvés                               │
├────────────────────────────────────────────────────┤
│                                                    │
│ 🔷 SOLUTIONS IA (5 résultats)                     │
│                                                    │
│ ⭐ Automatisation de la Facturation               │
│    Libérez 80% du temps de votre équipe...        │
│    [solutions-ia/automatisation-facturation]      │
│                                                    │
│    Gestion Intelligente des Stocks                │
│    Optimisez vos approvisionnements avec...       │
│    [solutions-ia/gestion-stocks]                  │
│                                                    │
├────────────────────────────────────────────────────┤
│ 📚 ARTICLES BLOG (8 résultats)                    │
│                                                    │
│    "5 Processus à Automatiser en Priorité"        │
│    Découvrez comment l'automatisation peut...     │
│    Publié le 5 déc 2024 • 8 min lecture          │
│                                                    │
│    "ROI de l'IA : Combien Économise..."          │
│    Une étude approfondie sur le retour...         │
│    Publié le 28 nov 2024 • 12 min lecture        │
│                                                    │
├────────────────────────────────────────────────────┤
│ ❓ FAQ (3 résultats)                               │
│                                                    │
│    Comment fonctionne l'automatisation ?          │
│    L'IA scanne vos documents et extrait...        │
│                                                    │
├────────────────────────────────────────────────────┤
│ 🎥 VIDÉOS (2 résultats)                           │
│                                                    │
│    [Thumbnail] Démo : Automatisation Facturation  │
│    Durée : 3:24 • 1 245 vues                      │
│                                                    │
└────────────────────────────────────────────────────┘

💬 TED : "Je vois que vous cherchez des infos sur 
l'automatisation de facturation. Puis-je vous aider 
à trouver la solution adaptée à votre situation ?"

[Oui, aide-moi] [Non merci]
6.2 Système de Personnalisation
Détection Automatique du Profil Visiteur
Mécanisme de Scoring :
function detectUserProfile() {
    const signals = {
        // Analyse comportementale
        visitedPages: getVisitedPages(),
        timeOnPages: getTimeSpent(),
        interactions: getInteractions(),
        
        // Données techniques
        referrer: document.referrer,
        utmParams: getUTMParameters(),
        device: detectDevice(),
        
        // Données explicites
        formData: getSavedFormData(),
        previousConversations: getTEDHistory()
    };
    
    const profiles = {
        b2b_decision_maker: calculateB2BScore(signals),
        b2c_restaurant_client: calculateB2CScore(signals),
        partner_potential: calculatePartnerScore(signals),
        media_press: calculateMediaScore(signals),
        job_seeker: calculateJobScore(signals)
    };
    
    return getHighestScoringProfile(profiles);
}
Adaptations par Profil :
PROFIL B2B (Dirigeant PME) :
Homepage adaptée :
• Hero CTA : "Automatisez 80% de vos tâches répétitives"
• Mise en avant ROI et cas clients
• TED s'ouvre avec : "Cherchez-vous une solution IA ?"
• Éléments sociaux : Logos clients, chiffres économies
• Menu : Solutions IA en premier
PROFIL B2C (Client Restaurant) :
Homepage adaptée :
• Hero CTA : "Réserver une table" en premier plan
• Visuels appétissants de plats
• TED propose : "Menu du jour ou réservation ?"
• Badge "Dernière récolte du jardin : ce matin"
• Menu : viTEDia en premier
PROFIL PARTENAIRE :
Homepage adaptée :
• Hero CTA : "Découvrir nos partenariats"
• Mise en avant impact social & environnemental
• TED : "Intéressé par un partenariat ?"
• Section dédiée partenaires existants
• Formulaire contact B2B facilement accessible
Système de Recommandation
Algorithme de Suggestions :
function generateRecommendations(user) {
    const recommendations = [];
    
    // Basé sur l'historique
    if (user.viewedSolutionIA('facturation')) {
        recommendations.push({
            type: 'cross_sell',
            title: 'Complétez avec la Gestion de Stocks',
            reason: 'Les clients qui automatisent la facturation optimisent aussi leurs stocks',
            cta: 'Découvrir',
            link: '/solutions-ia/gestion-stocks',
            discount: '15% si combiné'
        });
    }
    
    // Basé sur le comportement
    if (user.timeOnPage('/vitedia') > 120) {
        recommendations.push({
            type: 'engagement',
            title: 'Réservez Maintenant et Économisez 10%',
            reason: 'Offre spéciale première visite',
            cta: 'Réserver',
            link: '/vitedia/reservation'
        });
    }
    
    // Basé sur le profil
    if (user.profile === 'b2b' && user.sector === 'restauration') {
        recommendations.push({
            type: 'case_study',
            title: 'Comment un Restaurant a Réduit ses Coûts de 40%',
            reason: 'Cas similaire à votre activité',
            cta: 'Lire le cas',
            link: '/blog/cas-client-restaurant'
        });
    }
    
    return recommendations;
}
Affichage des Recommandations :
<div class="recommendations-panel">
    <h3>Recommandé Pour Vous</h3>
    
    <div class="recommendation-card">
        <span class="rec-badge">Basé sur votre profil</span>
        <h4>Complétez avec la Gestion de Stocks</h4>
        <p>Les clients qui automatisent la facturation 
           optimisent aussi leurs stocks</p>
        <div class="rec-benefit">
            💰 15% de réduction si combiné
        </div>
        <button class="rec-cta">Découvrir</button>
    </div>
</div>
6.3 Système Multi-langue
Langues Supportées
Priorité 1 (Lancement) :
•	🇫🇷 Français (défaut)
•	🇬🇧 Anglais
Priorité 2 (Phase 2) :
•	🇩🇪 Allemand
•	🇪🇸 Espagnol
Détection & Sélection Langue
Détection Automatique :
function detectLanguage() {
    // 1. Paramètre URL
    const urlLang = getURLParameter('lang');
    if (urlLang) return urlLang;
    
    // 2. Cookie préférence
    const cookieLang = getCookie('preferred_language');
    if (cookieLang) return cookieLang;
    
    // 3. Langue navigateur
    const browserLang = navigator.language.split('-')[0];
    const supportedLangs = ['fr', 'en', 'de', 'es'];
    if (supportedLangs.includes(browserLang)) {
        return browserLang;
    }
    
    // 4. Géolocalisation IP
    const geoLang = getLanguageFromIP();
    if (geoLang) return geoLang;
    
    // 5. Défaut
    return 'fr';
}
Sélecteur de Langue (Header) :
<div class="language-selector">
    <button class="current-lang">
        <img src="/flags/fr.svg" alt="FR"> FR
    </button>
    <div class="lang-dropdown">
        <a href="?lang=fr" class="lang-option">
            <img src="/flags/fr.svg"> Français
        </a>
        <a href="?lang=en" class="lang-option">
            <img src="/flags/gb.svg"> English
        </a>
        <a href="?lang=de" class="lang-option">
            <img src="/flags/de.svg"> Deutsch
        </a>
        <a href="?lang=es" class="lang-option">
            <img src="/flags/es.svg"> Español
        </a>
    </div>
</div>
Structure URLs Multi-langue
Approche Sous-dossiers :
https://tedsai.com/           (FR - défaut)
https://tedsai.com/en/        (EN)
https://tedsai.com/de/        (DE)
https://tedsai.com/es/        (ES)

https://tedsai.com/solutions-ia
https://tedsai.com/en/ai-solutions
https://tedsai.com/de/ki-losungen
https://tedsai.com/es/soluciones-ia
Balises hreflang (SEO) :
<link rel="alternate" hreflang="fr" href="https://tedsai.com/" />
<link rel="alternate" hreflang="en" href="https://tedsai.com/en/" />
<link rel="alternate" hreflang="de" href="https://tedsai.com/de/" />
<link rel="alternate" hreflang="es" href="https://tedsai.com/es/" />
<link rel="alternate" hreflang="x-default" href="https://tedsai.com/" />
Système de Traduction
Fichiers JSON par langue :
// fr.json
{
    "nav.home": "Accueil",
    "nav.solutions": "Solutions IA",
    "nav.restaurant": "viTEDia",
    "nav.garden": "SelecTED Garden",
    "nav.blog": "Le Lab",
    "nav.about": "À Propos",
    "nav.contact": "Contact",
    
    "home.hero.title": "De la Data à l'Assiette, Un Écosystème Intelligent",
    "home.hero.subtitle": "IA pour entreprises • Restaurant durable • Jardin urbain",
    "home.cta.discover": "Découvrir nos Solutions IA",
    "home.cta.book": "Réserver chez viTEDia",
    
    // etc...
}

// en.json
{
    "nav.home": "Home",
    "nav.solutions": "AI Solutions",
    "nav.restaurant": "viTEDia",
    "nav.garden": "SelecTED Garden",
    "nav.blog": "The Lab",
    "nav.about": "About",
    "nav.contact": "Contact",
    
    "home.hero.title": "From Data to Plate, An Intelligent Ecosystem",
    "home.hero.subtitle": "AI for businesses • Sustainable restaurant • Urban garden",
    "home.cta.discover": "Discover our AI Solutions",
    "home.cta.book": "Book at viTEDia",
    
    // etc...
}
Fonction de Traduction :
function translate(key, lang = currentLanguage) {
    const translations = loadTranslations(lang);
    return translations[key] || key;
}

// Usage dans le code
document.querySelector('.hero-title').textContent = 
    translate('home.hero.title');
TED Multi-langue
Détection Langue Conversation :
TED.onMessage((message) => {
    const detectedLang = detectLanguage(message.text);
    
    if (detectedLang !== TED.currentLanguage) {
        TED.switchLanguage(detectedLang);
        TED.respond(translate('ted.language_switch', detectedLang));
    } else {
        TED.processMessage(message);
    }
});

// Exemple
User: "Hello, I need help"
TED: "Hello! I've switched to English. How can I assist you today?"

User: "Actually, je préfère en français"
TED: "D'accord ! Je repasse en français. Comment puis-je vous aider ?"
6.4 Programme de Fidélité & Gamification
SelecTED Rewards Program
Concept : Système de points récompensant l'engagement avec l'écosystème TEDSAI
Moyens de Gagner des Points :
ACTION                           | POINTS | FRÉQUENCE
─────────────────────────────────┼────────┼──────────────
Réservation viTEDia             | 100    | Par visite
Scanner QR traçabilité          | 10     | Par scan
Partage social traçabilité      | 25     | Par partage
Avis Google/TripAdvisor         | 200    | Une fois
Parrainage client restaurant    | 300    | Par filleul
S'inscrire newsletter           | 50     | Une fois
Lire article blog complet       | 15     | Par article
Visite guidée jardin            | 150    | Par visite
Commande Click & Collect        | 75     | Par commande
Anniversaire                    | 100    | Annuel
Paliers & Récompenses :
🌱 NIVEAU SEMIS (0-500 points)
Avantages :
• Newsletter mensuelle exclusive
• Accès prioritaire nouveaux articles blog

🌿 NIVEAU POUSSE (501-1500 points)
Avantages :
• 5% réduction viTEDia
• Invitation événements privés
• Visite jardin gratuite

🌳 NIVEAU ARBRE (1501-3000 points)
Avantages :
• 10% réduction viTEDia
• 1 menu dégustation/an offert
• Panier hebdomadaire -20%
• Accès bêta nouvelles fonctionnalités IA

🏆 NIVEAU FORÊT (3000+ points)
Avantages :
• 15% réduction permanente
• Table réservée en priorité
• Participation aux récoltes VIP
• Nom sur le "Mur des Ambassadeurs"
• Consultation gratuite solution IA
Dashboard Utilisateur :
<div class="rewards-dashboard">
    <div class="rewards-header">
        <div class="user-level">
            <span class="level-icon">🌿</span>
            <div class="level-info">
                <h3>Niveau Pousse</h3>
                <p>1 247 points • Plus que 254 points pour Niveau Arbre</p>
            </div>
        </div>
        <div class="progress-bar">
            <div class="progress-fill" style="width: 83%"></div>
        </div>
    </div>
    
    <div class="rewards-grid">
        <div class="reward-stat">
            <span class="stat-icon">🍽️</span>
            <div>
                <strong>12</strong>
                <p>Visites restaurant</p>
            </div>
        </div>
        <div class="reward-stat">
            <span class="stat-icon">🌱</span>
            <div>
                <strong>47</strong>
                <p>Tracés scannés</p>
            </div>
        </div>
        <div class="reward-stat">
            <span class="stat-icon">📤</span>
            <div>
                <strong>8</strong>
                <p>Partages sociaux</p>
            </div>
        </div>
    </div>
    
    <div class="rewards-actions">
        <h4>Gagnez Plus de Points</h4>
        <div class="action-card">
            <span class="action-icon">📝</span>
            <div class="action-content">
                <strong>Laissez un avis Google</strong>
                <p>Gagnez 200 points</p>
            </div>
            <button class="action-btn">Faire →</button>
        </div>
        <div class="action-card">
            <span class="action-icon">👥</span>
            <div class="action-content">
                <strong>Parrainez un ami</strong>
                <p>300 points par filleul</p>
            </div>
            <button class="action-btn">Partager →</button>
        </div>
    </div>
    
    <div class="rewards-benefits">
        <h4>Vos Avantages Actuels</h4>
        <ul>
            <li>✅ 5% réduction viTEDia</li>
            <li>✅ Invitation événements privés</li>
            <li>✅ Visite jardin gratuite</li>
        </ul>
    </div>
</div>
Badges & Achievements
Système de Badges Collectibles :
CATÉGORIE EXPLORATION :
🔍 "Curieux" - Visité 5 pages différentes
🗺️ "Explorateur" - Visité les 3 compartiments
📚 "Érudit" - Lu 10 articles blog
🎓 "Expert" - Complété tous les parcours

CATÉGORIE ENGAGEMENT :
🍽️ "Habitué" - 5 réservations
🌟 "VIP" - 20 réservations
💎 "Légende" - 50 réservations
🎂 "Anniversaire" - Célébré un anniversaire chez nous

CATÉGORIE TRAÇABILITÉ :
🌱 "Traceur Novice" - 10 QR scannés
🔬 "Inspecteur" - 50 QR scannés
🏆 "Maître Traceur" - 100 QR scannés

CATÉGORIE SOCIAL :
📢 "Ambassadeur" - 10 partages sociaux
🌍 "Influenceur" - 50 partages sociaux
⭐ "Critique" - Laissé 3 avis positifs

CATÉGORIE JARDIN :
🌾 "Jardinier Apprenti" - 1 visite guidée
🌳 "Main Verte" - Participé à une récolte
🚜 "Agriculteur Urbain" - 5 visites jardin

CATÉGORIE EXCLUSIVE :
🥇 "Premier" - Dans les 100 premiers inscrits
🎖️ "Fondateur" - Client année 1
👑 "Pionnier" - Testé 3 features bêta
Affichage Profil Utilisateur :
<div class="user-profile">
    <div class="profile-header">
        <img src="avatar.jpg" class="avatar">
        <div class="profile-info">
            <h2>Marie Dupont</h2>
            <p>🌿 Niveau Pousse • 1 247 points</p>
            <p>Membre depuis mars 2024</p>
        </div>
    </div>
    
    <div class="badges-showcase">
        <h3>Badges Obtenus (12/45)</h3>
        <div class="badges-grid">
            <div class="badge earned">
                <span class="badge-icon">🍽️</span>
                <p>Habitué</p>
            </div>
            <div class="badge earned">
                <span class="badge-icon">🌱</span>
                <p>Traceur Novice</p>
            </div>
            <div class="badge earned">
                <span class="badge-icon">📢</span>
                <p>Ambassadeur</p>
            </div>
            <div class="badge locked">
                <span class="badge-icon opacity-30">🌟</span>
                <p>VIP</p>
                <small>8 visites restantes</small>
            </div>
            <!-- ... -->
        </div>
    </div>
</div>
6.5 Newsletter & Email Marketing
Inscription Newsletter
Formulaires d'Inscription (Multiples Emplacements) :
Footer Site :
<div class="newsletter-footer">
    <h3>Restez Connecté à l'Écosystème TEDSAI</h3>
    <p>Recevez nos actualités, recettes exclusives et conseils IA</p>
    <form class="newsletter-form">
        <input type="email" placeholder="votre@email.com" required>
        <button type="submit">S'abonner</button>
    </form>
    <small>📧 2 emails/mois maximum • Désinscription en 1 clic</small>
</div>
Pop-up Exit Intent :
// Détection intention de quitter
document.addEventListener('mouseout', (e) => {
    if (e.clientY < 50 && !hasSeenPopup) {
        showNewsletterPopup();
    }
});

function showNewsletterPopup() {
    // Modal
    `
    ✋ Attendez !
    
    Ne manquez pas nos prochains articles et offres exclusives.
    
    🎁 CADEAU DE BIENVENUE :
    • Livre Blanc "L'IA pour les PME" (PDF)
    • 10% sur votre première réservation viTEDia
    • Accès prioritaire aux événements
    
    [Email]
    [Je veux mon cadeau !]
    
    [Non merci, je préfère passer à côté]
    `
}
Segments d'Audience
Segmentation Intelligente :
const segments = {
    'b2b_prospects': {
        criteria: {
            visited: ['/solutions-ia'],
            notConverted: true,
            profile: 'business'
        },
        content: 'Case studies, ROI, webinars B2B'
    },
    
    'restaurant_clients': {
        criteria: {
            hasBooked: true,
            lastVisit: 'last_30_days'
        },
        content: 'Menus, événements, traçabilité'
    },
    
    'potential_partners': {
        criteria: {
            visited: ['/selected-garden', '/ecosysteme'],
            timeSpent: '> 5min',
            downloaded: 'partnership_doc'
        },
        content: 'Impact, collaborations, innovations'
    },
    
    'blog_readers': {
        criteria: {
            readArticles: '>= 3',
            notCustomer: true
        },
        content: 'Deep dives techniques, études de cas'
    },
    
    'loyal_customers': {
        criteria: {
            visits: '>= 5',
            reviewLeft: true
        },
        content: 'Exclusivités, avant-premières, VIP'
    }
};
Campagnes Email Types
1. Campagne Bienvenue (Automation) :
EMAIL 1 (Immédiat) : Bienvenue + Cadeaux
Sujet : "🎁 Votre cadeau de bienvenue TEDSAI"

Contenu :
• Merci de votre inscription
• Lien téléchargement Livre Blanc
• Code promo 10% viTEDia (BIENVENUE10)
• Présentation rapide écosystème
• CTA : Découvrir l'écosystème

━━━━━━━━━━━━━━━━━━━━━━━━

EMAIL 2 (J+3) : Découverte Solutions IA
Sujet : "Comment l'IA peut transformer votre PME 🚀"

Contenu :
• Article : "5 processus à automatiser"
• Vidéo : Démo automatisation 2 min
• Témoignage client court
• CTA : Tester l'IA Playground

━━━━━━━━━━━━━━━━━━━━━━━━

EMAIL 3 (J+7) : Invitation Action
Sujet : "🍽️ Goûtez la différence chez viTEDia"

Contenu :
• Histoire traçabilité avec exemple concret
• Photos appétissantes menu
• Offre spéciale : Dessert offert première visite
• CTA : Réserver ma table

━━━━━━━━━━━━━━━━━━━━━━━━

EMAIL 4 (J+14) : Ressources & Engagement
Sujet : "📚 Nos meilleures ressources pour vous"

Contenu :
• Top 3 articles blog
• Invitation webinar prochain
• Témoignage vidéo client
• CTA multiples selon intérêt
2. Newsletter Mensuelle Segment B2B :
Sujet : "🧠 TEDSAI Insights | Janvier 2025 : Tendances IA PME"

┌──────────────────────────────────────────────┐
│ 📊 CHIFFRE DU MOIS                           │
│ Les PME qui automatisent leur facturation    │
│ économisent en moyenne 1 250 heures/an       │
└──────────────────────────────────────────────┘

🎯 ARTICLE VEDETTE
"2025 : L'Année de l'IA Accessible pour les PME"
[Lire l'article complet - 8 min]

💡 CAS CLIENT RÉCENT
Comment une boulangerie a réduit son gaspillage de 60%
[Voir le cas →]

🎓 WEBINAR GRATUIT
"Automatiser sans licencier : L'IA comme assistant"
Jeudi 25 janvier, 14h-15h
[Je m'inscris →]

📚 RESSOURCES
• Checklist : 10 signaux qu'il est temps d'automatiser
• Guide : Choisir sa première solution IA
• Template : Calculateur ROI personnalisable

🔔 NOUVEAUTÉS TEDSAI
• Nouveau module : Prédiction de trésorerie IA
• Intégration : Compatible avec Sage X3
• Offre : -20% sur package combiné ce mois

[Découvrir les nouveautés →]

━━━━━━━━━━━━━━━━━━━━━━━━
PS : Besoin d'un audit gratuit ? Répondez simplement
à cet email, notre équipe vous contactera sous 24h.
3. Newsletter Mensuelle Segment Restaurant :
Sujet : "🍽️ viTEDia | Février : Nouveaux Menus & Événements"

┌──────────────────────────────────────────────┐
│ 🌱 DU JARDIN CETTE SEMAINE                   │
│ Asperges vertes • Fraises gariguette         │
│ Roquette sauvage • Herbes fraîches           │
└──────────────────────────────────────────────┘

🍴 NOUVEAUX PLATS
Notre chef vous propose ce mois-ci :
• Risotto aux asperges et citron confit
• Tartare de fraises, basilic et poivre
• Velouté de petits pois, menthe fraîche
[Voir le menu complet →]

🎉 ÉVÉNEMENT SPÉCIAL
🗓️ Vendredi 16 février - 19h30
"Soirée Dégustation : Accords Mets & Vins Locaux"
5 plats • 5 vins • Rencontre avec les producteurs
Places limitées : 30 personnes
[Réserver ma place →]

📸 VOS PLUS BELLES PHOTOS
Merci pour vos partages Instagram !
Cette semaine, @marie_foodlover a gagné
un menu dégustation offert 🎁
[Participer au concours →]

🌿 TRAÇABILITÉ DU MOIS
4 237 QR codes scannés en janvier !
Le produit le plus tracé : Tomates anciennes
[Découvrir leur parcours →]

💳 OFFRE FIDÉLITÉ
+50 points bonus sur toute réservation
ce week-end (10-11 février)
[Profiter de l'offre →]

━━━━━━━━━━━━━━━━━━━━━━━━
📍 viTEDia Restaurant
[Adresse] • [Téléphone]
[Réserver] [Instagram] [Menu]
4. Campagne Réactivation (Clients Inactifs) :
CONDITION : Pas de visite depuis 60 jours

EMAIL 1 (J+60) : On vous a manqué ?
Sujet : "Ça fait longtemps ! 😊 Voici ce que vous avez manqué"

Contenu :
• Message personnalisé chaleureux
• Nouveautés depuis dernière visite
• Offre retour : 15% réduction
• CTA : Revenir nous voir

━━━━━━━━━━━━━━━━━━━━━━━━

EMAIL 2 (J+75 si pas de réaction) : Feedback
Sujet : "Une minute pour nous améliorer ?"

Contenu :
• Questionnaire court (3 questions)
• Pourquoi ne revenez-vous plus ?
• Que pourrions-nous améliorer ?
• Cadeau : 20% réduction si feedback donné

━━━━━━━━━━━━━━━━━━━━━━━━

EMAIL 3 (J+90 si toujours inactif) : Dernière chance
Sujet : "C'est peut-être un au revoir... 💔"

Contenu :
• Message émouvant mais pas culpabilisant
• Offre irrésistible : Menu complet -30%
• Validité courte : 7 jours
• Option : Se désabonner proprement
Métriques & Optimisation
KPIs à Suivre :
MÉTRIQUE              | OBJECTIF     | COMMENT AMÉLIORER
──────────────────────┼──────────────┼─────────────────────
Taux d'ouverture      | > 25%        | A/B test sujets
Taux de clic          | > 3%         | CTAs plus visibles
Taux de conversion    | > 5%         | Offres personnalisées
Taux désabonnement    | < 2%         | Segmentation fine
ROI email             | > 3:1        | Automation poussée
A/B Testing Systématique :
const emailTests = {
    subject_lines: [
        "🎁 Votre cadeau vous attend",
        "Découvrez notre secret...",
        "Marie, une surprise pour vous"
    ],
    
    cta_buttons: [
        {text: "Découvrir", color: "blue"},
        {text: "J'en profite", color: "green"},
        {text: "Voir l'offre →", color: "orange"}
    ],
    
    send_times: [
        "Tuesday 10:00",
        "Thursday 14:00",
        "Saturday 09:00"
    ]
};

function runABTest(segment, variations) {
    const groups = splitSegment(segment, variations.length);
    
    groups.forEach((group, index) => {
        sendEmail(group, variations[index]);
        trackMetrics(variations[index].id);
    });
    
    // Après 24h, analyser et garder le meilleur
    setTimeout(() => {
        const winner = analyzeResults();
        useForFutureEmails(winner);
    }, 24 * 3600 * 1000);
}
 
7. SPÉCIFICATIONS TECHNIQUES
7.1 Stack Technologique Recommandé
Frontend
Framework Principal :
•	Next.js 14+ (React) 
o	SSR/SSG pour SEO optimal
o	App Router pour routing moderne
o	API Routes pour backend léger
o	Image Optimization native
o	Internationalization intégrée
Styling :
•	Tailwind CSS 3+ 
o	Utility-first pour rapidité développement
o	Configuration custom (couleurs TEDSAI)
o	Plugin forms, typography, animations
UI Components :
•	Headless UI (composants accessibles)
•	Framer Motion (animations fluides)
•	React Hook Form (gestion formulaires)
•	Recharts / D3.js (dataviz)
State Management :
•	Zustand (léger, performant)
•	React Query (data fetching, cache)
Backend
API :
•	Node.js + Express ou Next.js API Routes
•	REST pour endpoints standards
•	GraphQL (optionnel) pour queries complexes
Base de Données :
•	PostgreSQL 15+ (données structurées) 
o	Tables : users, reservations, products, traceability
o	Extensions : PostGIS (géolocalisation)
•	Redis (cache, sessions)
•	MongoDB (logs, analytics optionnel)
CMS (Blog) :
•	Strapi ou Sanity.io 
o	Headless CMS pour flexibilité
o	API automatique
o	Interface admin conviviale
Infrastructure
Hébergement :
•	Vercel (Frontend Next.js) 
o	Edge Network global
o	Déploiement automatique GitHub
o	Previews pour chaque PR
•	AWS / DigitalOcean (Backend) 
o	EC2 / Droplets pour API
o	RDS pour PostgreSQL
o	S3 pour médias (images, vidéos)
CDN :
•	Cloudflare (images, assets statiques) 
o	Optimisation automatique images
o	Protection DDoS
o	Analytics
Monitoring :
•	Sentry (error tracking)
•	Google Analytics 4
•	Hotjar (heatmaps, recordings)
•	Uptime Robot (disponibilité)
7.2 Architecture Système
Diagramme Architecture Globale
┌─────────────────────────────────────────────────┐
│              UTILISATEURS                        │
│  Desktop • Mobile • Tablette                    │
└──────────────┬──────────────────────────────────┘
               │
               ▼
┌─────────────────────────────────────────────────┐
│              CLOUDFLARE CDN                      │
│  Cache • WAF • DDoS Protection                  │
└──────────────┬──────────────────────────────────┘
               │
       ┌───────┴────────┐
       ▼                ▼
┌─────────────┐   ┌──────────────┐
│   VERCEL    │   │     AWS      │
│  (Frontend) │   │   (Backend)  │
│             │   │              │
│ • Next.js   │◄─►│ • Node API   │
│ • SSR/SSG   │   │ • PostgreSQL │
│ • Edge Func │   │ • Redis      │
└──────┬──────┘   └──────┬───────┘
       │                 │
       │          ┌──────┴───────┐
       │          ▼              ▼
       │   ┌───────────┐  ┌─────────────┐
       │   │ STRIPE    │  │  CALENDLY   │
       │   │ Paiement  │  │  RDV        │
       │   └───────────┘  └─────────────┘
       │
       │          ┌──────────────┐
       └─────────►│  S3 BUCKET   │
                  │  Médias      │
                  └──────────────┘
                  
┌─────────────────────────────────────────────────┐
│           SERVICES EXTERNES                      │
│                                                  │
│ • OpenAI API (TED Intelligence)                 │
│ • SendGrid (Emails transactionnels)             │
│ • Twilio (SMS notifications)                    │
│ • Google Maps API (Localisation)                │
│ • Social APIs (Facebook, Instagram)             │
└─────────────────────────────────────────────────┘
Base de Données - Schéma Principal
-- USERS
CREATE TABLE users (
    id UUID PRIMARY KEY DEFAULT gen_random_uuid(),
    email VARCHAR(255) UNIQUE NOT NULL,
    password_hash VARCHAR(255),
    first_name VARCHAR(100),
    last_name VARCHAR(100),
    phone VARCHAR(20),
    language VARCHAR(2) DEFAULT 'fr',
    profile_type VARCHAR(20), -- b2b, b2c, partner
    loyalty_points INTEGER DEFAULT 0,
    loyalty_level VARCHAR(20) DEFAULT 'semis',
    created_at TIMESTAMP DEFAULT NOW(),
    updated_at TIMESTAMP DEFAULT NOW()
);

-- RESERVATIONS (viTEDia)
CREATE TABLE reservations (
    id UUID PRIMARY KEY DEFAULT gen_random_uuid(),
    user_id UUID REFERENCES users(id),
    reservation_date DATE NOT NULL,
    reservation_time TIME NOT NULL,
    guests_count INTEGER NOT NULL,
    location VARCHAR(50), -- terrasse, interieur, salon_prive
    occasion VARCHAR(100), -- anniversaire, affaires, etc.
    special_requests TEXT,
    status VARCHAR(20) DEFAULT 'confirmed',
    created_at TIMESTAMP DEFAULT NOW(),
    confirmed_at TIMESTAMP,
    cancelled_at TIMESTAMP
);

-- PRODUCTS (SelecTED Garden)
CREATE TABLE products (
    id UUID PRIMARY KEY DEFAULT gen_random_uuid(),
    code VARCHAR(50) UNIQUE NOT NULL,
    name VARCHAR(255) NOT NULL,
    category VARCHAR(100), -- legume, fruit, herbe, etc.
    variety VARCHAR(100),
    parcelle VARCHAR(50),
    planting_date DATE,
    harvest_date DATE,
    treatment_info TEXT,
    iot_data JSONB,
    photos TEXT[], -- URLs
    created_at TIMESTAMP DEFAULT NOW()
);

-- TRACEABILITY (Traçabilité complète)
CREATE TABLE traceability_events (
    id UUID PRIMARY KEY DEFAULT gen_random_uuid(),
    product_id UUID REFERENCES products(id),
    event_type VARCHAR(50), -- planting, irrigation, harvest, delivery
    event_date TIMESTAMP NOT NULL,
    details JSONB,
    user_responsible VARCHAR(100),
    location_lat DECIMAL(10, 8),
    location_lng DECIMAL(11, 8),
    created_at TIMESTAMP DEFAULT NOW()
);

-- LEADS (B2B)
CREATE TABLE leads (
    id UUID PRIMARY KEY DEFAULT gen_random_uuid(),
    company_name VARCHAR(255),
    contact_name VARCHAR(255),
    email VARCHAR(255),
    phone VARCHAR(20),
    sector VARCHAR(100),
    employee_count VARCHAR(20),
    problem_area VARCHAR(100),
    budget_range VARCHAR(50),
    urgency VARCHAR(50),
    lead_score INTEGER,
    status VARCHAR(50) DEFAULT 'new',
    assigned_to VARCHAR(100),
    source VARCHAR(100), -- website, referral, etc.
    ted_conversation_id VARCHAR(100),
    created_at TIMESTAMP DEFAULT NOW(),
    updated_at TIMESTAMP DEFAULT NOW()
);

-- BLOG POSTS
CREATE TABLE blog_posts (
    id UUID PRIMARY KEY DEFAULT gen_random_uuid(),
    slug VARCHAR(255) UNIQUE NOT NULL,
    title VARCHAR(500) NOT NULL,
    excerpt TEXT,
    content TEXT NOT NULL,
    category VARCHAR(100),
    tags TEXT[],
    featured_image VARCHAR(500),
    author_id UUID REFERENCES users(id),
    published BOOLEAN DEFAULT false,
    published_at TIMESTAMP,
    views_count INTEGER DEFAULT 0,
    reading_time INTEGER, -- minutes
    lang VARCHAR(2) DEFAULT 'fr',
    created_at TIMESTAMP DEFAULT NOW(),
    updated_at TIMESTAMP DEFAULT NOW()
);

-- LOYALTY ACTIONS
CREATE TABLE loyalty_actions (
    id UUID PRIMARY KEY DEFAULT gen_random_uuid(),
    user_id UUID REFERENCES users(id),
    action_type VARCHAR(100),
    points_earned INTEGER,
    description TEXT,
    created_at TIMESTAMP DEFAULT NOW()
);

-- NEWSLETTER SUBSCRIBERS
CREATE TABLE newsletter_subscribers (
    id UUID PRIMARY KEY DEFAULT gen_random_uuid(),
    email VARCHAR(255) UNIQUE NOT NULL,
    segments TEXT[], -- b2b, restaurant, blog
    status VARCHAR(20) DEFAULT 'active',
    source VARCHAR(100),
    subscribed_at TIMESTAMP DEFAULT NOW(),
    unsubscribed_at TIMESTAMP
);

-- TED CONVERSATIONS
CREATE TABLE ted_conversations (
    id UUID PRIMARY KEY DEFAULT gen_random_uuid(),
    user_id UUID REFERENCES users(id),
    session_id VARCHAR(100),
    messages JSONB,
    context JSONB, -- page, user profile, etc.
    lead_created BOOLEAN DEFAULT false,
    satisfaction_rating INTEGER,
    created_at TIMESTAMP DEFAULT NOW(),
    ended_at TIMESTAMP
);
7.3 Sécurité & Conformité
RGPD & Protection Données
Principes Appliqués :
1.	Minimisation des Données
// Ne collecter que le nécessaire
const collectUserData = {
    essential: ['email', 'name'], // Obligatoire
    optional: ['phone', 'birthdate'], // Avec consentement explicite
    forbidden: ['religion', 'political_views'] // Jamais collecté
};
2.	Consentement Explicite
<form class="signup-form">
    <input type="email" required>
    
    <label class="consent-checkbox">
        <input type="checkbox" required>
        J'accepte la <a href="/privacy">Politique de Confidentialité</a>
    </label>
    
    <label class="optional-consent">
        <input type="checkbox">
        J'accepte de recevoir des offres personnalisées par email
    </label>
    
    <button type="submit">S'inscrire</button>
</form>
3.	Droit d'Accès & Suppression
// API endpoints RGPD
app.get('/api/user/data-export', authenticate, async (req, res) => {
    const userData = await exportAllUserData(req.user.id);
    res.json(userData); // Format JSON complet
});

app.delete('/api/user/account', authenticate, async (req, res) => {
    await anonymizeUserData(req.user.id); // Anonymisation
    await deleteAccount(req.user.id);
    res.json({message: 'Compte supprimé'});
});
4.	Bannière Cookies
<div class="cookie-banner" id="cookieBanner">
    <div class="cookie-content">
        <p>Nous utilisons des cookies pour améliorer votre expérience.</p>
        <div class="cookie-actions">
            <button onclick="acceptAll()">Accepter Tout</button>
            <button onclick="customizePreferences()">Personnaliser</button>
            <button onclick="rejectOptional()">Refuser Non-Essentiels</button>
        </div>
        <a href="/cookies-policy">En savoir plus</a>
    </div>
</div>

<script>
function acceptAll() {
    setCookie('analytics', 'true');
    setCookie('marketing', 'true');
    hideBanner();
}

function customizePreferences() {
    showPreferencesModal();
}

function rejectOptional() {
    setCookie('analytics', 'false');
    setCookie('marketing', 'false');
    setCookie('essential', 'true'); // Toujours actifs
    hideBanner();
}
</script>
Sécurité Technique
1. HTTPS Obligatoire
# Nginx config - Redirect HTTP to HTTPS
server {
    listen 80;
    server_name tedsai.com www.tedsai.com;
    return 301 https://$server_name$request_uri;
}

server {
    listen 443 ssl http2;
    server_name tedsai.com www.tedsai.com;
    
    ssl_certificate /path/to/cert.pem;
    ssl_certificate_key /path/to/key.pem;
    ssl_protocols TLSv1.2 TLSv1.3;
}
2. Protection XSS & CSRF
// Helmet.js (Express)
const helmet = require('helmet');
app.use(helmet({
    contentSecurityPolicy: {
        directives: {
            defaultSrc: ["'self'"],
            styleSrc: ["'self'", "'unsafe-inline'", 'fonts.googleapis.com'],
            scriptSrc: ["'self'", 'cdnjs.cloudflare.com'],
            imgSrc: ["'self'", 'data:', 'https:'],
            fontSrc: ["'self'", 'fonts.gstatic.com']
        }
    }
}));

// CSRF Token
const csrf = require('csurf');
app.use(csrf({ cookie: true }));
3. Rate Limiting
const rateLimit = require('express-rate-limit');

// Limite globale
const globalLimiter = rateLimit({
    windowMs: 15 * 60 * 1000, // 15 minutes
    max: 100 // 100 requêtes max
});

// Limite formulaires
const formLimiter = rateLimit({
    windowMs: 60 * 60 * 1000, // 1 heure
    max: 5, // 5 soumissions max
    message: 'Trop de tentatives, réessayez dans 1 heure'
});

app.use('/api/', globalLimiter);
app.post('/api/contact', formLimiter, handleContact);
4. Sanitization Inputs
const sanitizeHtml = require('sanitize-html');
const validator = require('validator');

function sanitizeUserInput(input) {
    // Nettoyer HTML
    const clean = sanitizeHtml(input, {
        allowedTags: [], // Aucun tag HTML autorisé
        allowedAttributes: {}
    });
    
    // Valider email
    if (validator.isEmail(clean)) {
        return validator.normalizeEmail(clean);
    }
    
    // Escape SQL
    return escapeSQL(clean);
}
5. Authentification Sécurisée
const bcrypt = require('bcrypt');
const jwt = require('jsonwebtoken');

// Hashing mot de passe
async function hashPassword(password) {
    const salt = await bcrypt.genSalt(12);
    return bcrypt.hash(password, salt);
}

// Vérification
async function verifyPassword(password, hash) {
    return bcrypt.compare(password, hash);
}

// JWT Token
function generateToken(userId) {
    return jwt.sign(
        { userId },
        process.env.JWT_SECRET,
        { expiresIn: '7d' }
    );
}
7.4 Performance & Optimisation
Temps de Chargement
Objectifs :
•	First Contentful Paint (FCP) : < 1.5s
•	Largest Contentful Paint (LCP) : < 2.5s
•	Time to Interactive (TTI) : < 3.5s
•	Cumulative Layout Shift (CLS) : < 0.1
Techniques d'Optimisation :
1. Images
// Next.js Image Component
import Image from 'next/image';

<Image
    src="/hero-image.jpg"
    alt="TEDSAI Complex"
    width={1920}
    height={1080}
    priority // Pour images above-the-fold
    quality={85} // Compression optimale
    placeholder="blur" // Placeholder flou
    blurDataURL="/hero-image-blur.jpg"
/>

// Formats modernes
// Servir WebP avec fallback JPEG
<picture>
    <source srcset="image.webp" type="image/webp">
    <source srcset="image.jpg" type="image/jpeg">
    <img src="image.jpg" alt="Description">
</picture>
2. Lazy Loading
// Lazy load sections non-critiques
import dynamic from 'next/dynamic';

const TestimonialsSection = dynamic(
    () => import('../components/Testimonials'),
    { loading: () => <Skeleton /> }
);

// Lazy load images hors viewport
<img
    src="product.jpg"
    loading="lazy"
    alt="Produit"
/>
3. Code Splitting
// Routes séparées
// pages/solutions-ia.js charge uniquement son code
// pages/vitedia.js charge uniquement son code

// Dynamic imports
const HeavyComponent = dynamic(() => import('./HeavyComponent'), {
    ssr: false, // Pas de SSR si pas nécessaire
    loading: () => <Spinner />
});
4. Caching Agressif
// Cache API responses
import { useQuery } from 'react-query';

const { data } = useQuery(
    ['menu', date],
    () => fetchMenu(date),
    {
        staleTime: 5 * 60 * 1000, // 5 minutes
        cacheTime: 30 * 60 * 1000 // 30 minutes
    }
);

// Service Worker pour offline
// sw.js
self.addEventListener('fetch', (event) => {
    event.respondWith(
        caches.match(event.request).then((response) => {
            return response || fetch(event.request);
        })
    );
});
5. Minification & Compression
// next.config.js
module.exports = {
    compress: true, // Gzip automatique
    
    webpack: (config) => {
        config.optimization.minimize = true;
        return config;
    },
    
    // Tree shaking automatique
    // Dead code elimination
};
SEO Technique
1. Métadonnées Optimales
// components/SEOHead.jsx
import Head from 'next/head';

export default function SEOHead({ page }) {
    return (
        <Head>
            {/* Title optimisé */}
            <title>{page.title} | TEDSAI Complex</title>
            <meta name="description" content={page.description} />
            
            {/* Open Graph (Facebook, LinkedIn) */}
            <meta property="og:title" content={page.title} />
            <meta property="og:description" content={page.description} />
            <meta property="og:image" content={page.image} />
            <meta property="og:url" content={page.url} />
            <meta property="og:type" content="website" />
            
            {/* Twitter Card */}
            <meta name="twitter:card" content="summary_large_image" />
            <meta name="twitter:title" content={page.title} />
            <meta name="twitter:description" content={page.description} />
            <meta name="twitter:image" content={page.image} />
            
            {/* Canonical */}
            <link rel="canonical" href={page.canonical} />
            
            {/* Hreflang pour multi-langue */}
            <link rel="alternate" hreflang="fr" href={page.url_fr} />
            <link rel="alternate" hreflang="en" href={page.url_en} />
            
            {/* Structured Data (JSON-LD) */}
            <script type="application/ld+json">
                {JSON.stringify(page.structuredData)}
            </script>
        </Head>
    );
}
2. Structured Data
// Organization Schema
const organizationSchema = {
    "@context": "https://schema.org",
    "@type": "Organization",
    "name": "TEDSAI Complex",
    "url": "https://tedsai.com",
    "logo": "https://tedsai.com/logo.png",
    "description": "Écosystème intelligent : IA pour entreprises, restaurant durable, jardin urbain",
    "address": {
        "@type": "PostalAddress",
        "streetAddress": "[Rue]",
        "addressLocality": "Yaoundé",
        "addressCountry": "CM"
    },
    "contactPoint": {
        "@type": "ContactPoint",
        "telephone": "+237-XXX-XXX-XXX",
        "contactType": "customer service",
        "availableLanguage": ["French", "English"]
    },
    "sameAs": [
        "https://facebook.com/tedsai",
        "https://linkedin.com/company/tedsai",
        "https://instagram.com/tedsai"
    ]
};

// Restaurant Schema
const restaurantSchema = {
    "@context": "https://schema.org",
    "@type": "Restaurant",
    "name": "viTEDia",
    "image": "https://tedsai.com/vitedia-hero.jpg",
    "servesCuisine": "Gastronomie durable",
    "priceRange": "€€",
    "address": {
        "@type": "PostalAddress",
        "streetAddress": "[Rue]",
        "addressLocality": "Yaoundé"
    },
    "menu": "https://tedsai.com/vitedia/menu",
    "acceptsReservations": "True"
};
3. Sitemap XML Dynamique
// pages/sitemap.xml.js
export async function getServerSideProps({ res }) {
    // Récupérer toutes les URLs
    const staticPages = [
        '',
        'ecosysteme',
        'solutions-ia',
        'vitedia',
        'selected-garden',
        'blog',
        'contact'
    ];
    
    const blogPosts = await fetchAllBlogPosts();
    const solutions = await fetchAllSolutions();
    
    const sitemap = `<?xml version="1.0" encoding="UTF-8"?>
    <urlset xmlns="http://www.sitemaps.org/schemas/sitemap/0.9">
        ${staticPages.map(page => `
            <url>
                <loc>https://tedsai.com/${page}</loc>
                <lastmod>${new Date().toISOString()}</lastmod>
                <changefreq>weekly</changefreq>
                <priority>${page === '' ? '1.0' : '0.8'}</priority>
            </url>
        `).join('')}
        
        ${blogPosts.map(post => `
            <url>
                <loc>https://tedsai.com/blog/${post.slug}</loc>
                <lastmod>${post.updated_at}</lastmod>
                <changefreq>monthly</changefreq>
                <priority>0.6</priority>
            </url>
        `).join('')}
    </urlset>`;
    
    res.setHeader('Content-Type', 'text/xml');
    res.write(sitemap);
    res.end();
    
    return { props: {} };
}
4. Robots.txt
# robots.txt
User-agent: *
Allow: /

# Bloquer pages admin
Disallow: /admin/
Disallow: /api/

# Bloquer pages de test
Disallow: /test/
Disallow: /_next/

# Sitemap
Sitemap: https://tedsai.com/sitemap.xml
Sitemap: https://tedsai.com/sitemap-blog.xml
Sitemap: https://tedsai.com/sitemap-solutions.xml

# Crawl-delay pour bots agressifs
User-agent: AhrefsBot
Crawl-delay: 10

User-agent: SemrushBot
Crawl-delay: 10
7.5 Analytics & Tracking
Configuration Google Analytics 4
// lib/gtag.js
export const GA_TRACKING_ID = 'G-XXXXXXXXXX';

// Initialisation
export const pageview = (url) => {
    window.gtag('config', GA_TRACKING_ID, {
        page_path: url,
    });
};

// Événements personnalisés
export const event = ({ action, category, label, value }) => {
    window.gtag('event', action, {
        event_category: category,
        event_label: label,
        value: value,
    });
};

// Événements spécifiques TEDSAI
export const trackLeadGeneration = (leadData) => {
    event({
        action: 'generate_lead',
        category: 'Lead',
        label: leadData.source,
        value: leadData.score
    });
};

export const trackReservation = (reservationData) => {
    event({
        action: 'make_reservation',
        category: 'Restaurant',
        label: 'viTEDia',
        value: reservationData.guests
    });
};

export const trackTraceability = (productCode) => {
    event({
        action: 'scan_qr_code',
        category: 'Traceability',
        label: productCode
    });
};

export const trackTEDInteraction = (conversation) => {
    event({
        action: 'chatbot_interaction',
        category: 'TED',
        label: conversation.intent,
        value: conversation.messages_count
    });
};
Implémentation dans Next.js :
// pages/_app.js
import { useEffect } from 'react';
import { useRouter } from 'next/router';
import * as gtag from '../lib/gtag';

function MyApp({ Component, pageProps }) {
    const router = useRouter();
    
    useEffect(() => {
        const handleRouteChange = (url) => {
            gtag.pageview(url);
        };
        
        router.events.on('routeChangeComplete', handleRouteChange);
        return () => {
            router.events.off('routeChangeComplete', handleRouteChange);
        };
    }, [router.events]);
    
    return <Component {...pageProps} />;
}
Dashboard Analytics Custom
KPIs à Suivre en Temps Réel :
// Dashboard Admin
const dashboardMetrics = {
    // B2B Metrics
    b2b: {
        visits_solutions_page: 0,
        demo_requests: 0,
        playground_tests: 0,
        leads_generated: 0,
        conversion_rate: 0,
        average_lead_score: 0
    },
    
    // B2C Metrics (Restaurant)
    restaurant: {
        menu_page_views: 0,
        reservations_made: 0,
        click_collect_orders: 0,
        traceability_scans: 0,
        social_shares: 0,
        average_guests_per_reservation: 0
    },
    
    // Garden Metrics
    garden: {
        virtual_tour_views: 0,
        partnership_inquiries: 0,
        visit_bookings: 0,
        product_traces: 0
    },
    
    // TED Metrics
    ted: {
        total_conversations: 0,
        avg_messages_per_conversation: 0,
        resolution_rate: 0,
        satisfaction_rating: 0,
        leads_from_ted: 0
    },
    
    // Global Metrics
    global: {
        unique_visitors: 0,
        returning_visitors: 0,
        avg_session_duration: 0,
        bounce_rate: 0,
        pages_per_session: 0
    }
};
Visualisation Dashboard :
// components/AdminDashboard.jsx
import { LineChart, BarChart, PieChart } from 'recharts';

export default function AdminDashboard({ metrics }) {
    return (
        <div className="dashboard-grid">
            {/* KPIs Principaux */}
            <div className="kpi-cards">
                <KPICard
                    title="Leads Générés"
                    value={metrics.b2b.leads_generated}
                    trend="+23%"
                    icon="🎯"
                />
                <KPICard
                    title="Réservations"
                    value={metrics.restaurant.reservations_made}
                    trend="+12%"
                    icon="🍽️"
                />
                <KPICard
                    title="QR Scannés"
                    value={metrics.restaurant.traceability_scans}
                    trend="+45%"
                    icon="🌱"
                />
                <KPICard
                    title="Taux Conversion"
                    value={`${metrics.b2b.conversion_rate}%`}
                    trend="+3.2%"
                    icon="📈"
                />
            </div>
            
            {/* Graphique Trafic */}
            <div className="chart-card">
                <h3>Trafic des 7 Derniers Jours</h3>
                <LineChart data={metrics.traffic_weekly}>
                    {/* Configuration graphique */}
                </LineChart>
            </div>
            
            {/* Sources de Trafic */}
            <div className="chart-card">
                <h3>Sources de Trafic</h3>
                <PieChart data={metrics.traffic_sources} />
            </div>
            
            {/* Parcours Utilisateurs */}
            <div className="funnel-card">
                <h3>Funnel Conversion B2B</h3>
                <FunnelChart stages={[
                    { name: 'Visite Solutions IA', value: 1000 },
                    { name: 'Test Playground', value: 350 },
                    { name: 'Demande Démo', value: 120 },
                    { name: 'Lead Qualifié', value: 85 },
                    { name: 'Client', value: 28 }
                ]} />
            </div>
            
            {/* TED Performance */}
            <div className="ted-stats">
                <h3>Performance TED</h3>
                <div className="ted-metrics">
                    <Metric label="Conversations" value={metrics.ted.total_conversations} />
                    <Metric label="Taux Résolution" value={`${metrics.ted.resolution_rate}%`} />
                    <Metric label="Satisfaction" value={`${metrics.ted.satisfaction_rating}/5`} />
                </div>
            </div>
            
            {/* Alerts & Actions */}
            <div className="alerts-card">
                <h3>Alertes & Actions</h3>
                <Alert type="success">
                    🎉 25 nouveaux leads cette semaine (+40%)
                </Alert>
                <Alert type="warning">
                    ⚠️ Taux de rebond page Solutions élevé (65%)
                </Alert>
                <Alert type="info">
                    💡 65% des conversions viennent de TED
                </Alert>
            </div>
        </div>
    );
}
 
8. STRATÉGIE SEO & CONTENU
8.1 Stratégie de Mots-Clés
Recherche & Ciblage
Mots-Clés Prioritaires par Compartiment :
TEDSAI IA (B2B) :
HAUTE PRIORITÉ (Volume moyen, Concurrence faible)
├─ "automatisation facturation cameroun" (30/mois, Difficulté: 25)
├─ "solution ia pme afrique" (50/mois, Difficulté: 30)
├─ "chatbot entreprise cameroun" (40/mois, Difficulté: 28)
├─ "gestion stock intelligente" (70/mois, Difficulté: 35)
└─ "optimisation processus pme" (60/mois, Difficulté: 32)

MOYENNE PRIORITÉ (Volume élevé, Concurrence moyenne)
├─ "intelligence artificielle entreprise" (200/mois, Difficulté: 55)
├─ "automatisation comptabilité" (150/mois, Difficulté: 50)
├─ "logiciel gestion restaurant" (180/mois, Difficulté: 52)
└─ "analyse prédictive business" (90/mois, Difficulté: 48)

LONGUE TRAÎNE (Volume faible, Conversion élevée)
├─ "comment automatiser facturation sage" (15/mois, Difficulté: 20)
├─ "ia pour petit restaurant yaoundé" (8/mois, Difficulté: 15)
├─ "solution gestion stock sans gaspillage" (12/mois, Difficulté: 18)
└─ "chatbot service client français cameroun" (10/mois, Difficulté: 16)
viTEDia Restaurant (B2C) :
HAUTE PRIORITÉ
├─ "restaurant bio yaoundé" (120/mois, Difficulté: 40)
├─ "restaurant produits locaux cameroun" (80/mois, Difficulté: 35)
├─ "restaurant traçabilité yaoundé" (25/mois, Difficulté: 20)
├─ "meilleur restaurant yaoundé" (400/mois, Difficulté: 60)
└─ "restaurant terrasse yaoundé" (90/mois, Difficulté: 42)

LONGUE TRAÎNE
├─ "restaurant légumes frais yaoundé" (18/mois, Difficulté: 25)
├─ "où manger bio cameroun" (22/mois, Difficulté: 28)
├─ "restaurant jardin urbain" (12/mois, Difficulté: 15)
└─ "réserver restaurant yaoundé en ligne" (35/mois, Difficulté: 30)
SelecTED Garden (B2B/B2C) :
HAUTE PRIORITÉ
├─ "agriculture urbaine cameroun" (60/mois, Difficulté: 38)
├─ "jardin urbain yaoundé" (40/mois, Difficulté: 30)
├─ "légumes bio yaoundé" (110/mois, Difficulté: 45)
├─ "aquaponie cameroun" (25/mois, Difficulté: 22)
└─ "traçabilité alimentaire afrique" (30/mois, Difficulté: 35)

LONGUE TRAÎNE
├─ "acheter légumes bio yaoundé" (45/mois, Difficulté: 32)
├─ "visite ferme urbaine cameroun" (8/mois, Difficulté: 15)
├─ "agriculture sans pesticide yaoundé" (12/mois, Difficulté: 18)
└─ "panier légumes bio livraison yaoundé" (28/mois, Difficulté: 28)
Stratégie de Contenu SEO
Calendrier Editorial 6 Mois (Exemples) :
Mois 1-2 : Fondations
SEMAINE 1
📄 "Guide Complet : L'IA pour les PME Camerounaises en 2025"
   Mot-clé : "intelligence artificielle pme cameroun"
   Format : Guide 3000 mots, infographies
   
📄 "Top 10 Restaurants Bio à Yaoundé : Notre Sélection"
   Mot-clé : "restaurant bio yaoundé"
   Format : Liste, photos HD

SEMAINE 2
📄 "Automatisation Facturation : ROI et Mise en Place"
   Mot-clé : "automatisation facturation"
   Format : Étude de cas, calculateur ROI
   
📄 "Agriculture Urbaine : 5 Techniques Innovantes"
   Mot-clé : "agriculture urbaine"
   Format : Tutorial, vidéos

SEMAINE 3
📄 "Chatbot vs Service Client Traditionnel : Comparatif"
   Mot-clé : "chatbot entreprise"
   Format : Tableau comparatif, chiffres
   
📄 "Traçabilité Alimentaire : Pourquoi C'est l'Avenir"
   Mot-clé : "traçabilité alimentaire"
   Format : Article argumenté, exemples

SEMAINE 4
📄 "Gestion de Stock Restaurant : Les Erreurs à Éviter"
   Mot-clé : "gestion stock restaurant"
   Format : Checklist, conseils pratiques
   
📄 "Légumes de Saison au Cameroun : Calendrier Complet"
   Mot-clé : "légumes saison cameroun"
   Format : Calendrier visuel, fiches produits
Mois 3-4 : Approfondissement
📄 "Cas Client : Comment [Entreprise X] a Économisé 40%"
📄 "Aquaponie : Notre Retour d'Expérience Après 1 An"
📄 "IA et RGPD : Guide Pratique pour PME"
📄 "Recettes Exclusives viTEDia : 10 Plats Signature"
📄 "Interview : Chef [Nom] Parle de Cuisine Durable"
📄 "IoT au Jardin : Technologies que Nous Utilisons"
Mois 5-6 : Expansion
📄 "Livre Blanc : Transformation Digitale PME 2025" (Lead magnet)
📄 "Webinar Replay : Automatiser Sans Licencier"
📄 "Comparatif : TEDSAI vs [Concurrents]" (Transparent)
📄 "Behind The Scenes : Une Journée chez viTEDia" (Vidéo)
📄 "Partenariat : Comment Travailler avec SelecTED Garden"
8.2 Optimisation On-Page
Structure Optimale Article Blog
<!-- Template Article SEO-Optimisé -->
<article class="blog-post">
    <!-- Breadcrumb -->
    <nav aria-label="breadcrumb">
        <ol>
            <li><a href="/">Accueil</a></li>
            <li><a href="/blog">Le Lab</a></li>
            <li><a href="/blog/intelligence-artificielle">IA</a></li>
            <li aria-current="page">Guide IA pour PME</li>
        </ol>
    </nav>
    
    <!-- Header Article -->
    <header>
        <!-- H1 avec mot-clé principal -->
        <h1>Guide Complet : L'Intelligence Artificielle pour les PME Camerounaises en 2025</h1>
        
        <!-- Méta informations -->
        <div class="post-meta">
            <span class="author">
                <img src="author.jpg" alt="Photo auteur">
                Par <a href="/auteur/john">John Doe</a>
            </span>
            <time datetime="2024-12-13">13 décembre 2024</time>
            <span class="reading-time">⏱️ 12 min lecture</span>
            <span class="category">
                <a href="/blog/intelligence-artificielle">Intelligence Artificielle</a>
            </span>
        </div>
        
        <!-- Image featured optimisée -->
        <figure>
            <img 
                src="ia-pme-cameroun.jpg" 
                alt="Intelligence artificielle pour PME au Cameroun - Dashboard IA"
                width="1200" 
                height="630"
                loading="eager"
            >
            <figcaption>L'IA transforme les PME camerounaises - Photo : TEDSAI</figcaption>
        </figure>
    </header>
    
    <!-- Table des matières -->
    <nav class="table-of-contents">
        <h2>Table des Matières</h2>
        <ul>
            <li><a href="#section-1">1. Pourquoi l'IA pour les PME ?</a></li>
            <li><a href="#section-2">2. Les 5 Domaines d'Application</a></li>
            <li><a href="#section-3">3. ROI et Budget</a></li>
            <li><a href="#section-4">4. Cas Pratiques</a></li>
            <li><a href="#section-5">5. Par Où Commencer ?</a></li>
        </ul>
    </nav>
    
    <!-- Contenu structuré avec H2/H3 -->
    <div class="post-content">
        <!-- Introduction avec mot-clé -->
        <p class="lead">
            L'<strong>intelligence artificielle pour les PME camerounaises</strong> 
            n'est plus un luxe réservé aux grandes entreprises. En 2025, les 
            solutions IA deviennent accessibles et rentables pour toutes les 
            tailles d'entreprise...
        </p>
        
        <!-- Section 1 -->
        <h2 id="section-1">1. Pourquoi l'IA pour les PME Camerounaises ?</h2>
        <p>Les petites et moyennes entreprises au Cameroun font face à...</p>
        
        <!-- Variation mot-clé naturelle -->
        <h3>Les Défis des PME Locales</h3>
        <ul>
            <li><strong>Gestion manuelle chronophage</strong> : 60% du temps...</li>
            <li><strong>Erreurs coûteuses</strong> : Les erreurs de saisie...</li>
            <li><strong>Manque de visibilité</strong> : Impossible d'avoir...</li>
        </ul>
        
        <!-- Image inline avec alt descriptif -->
        <figure>
            <img 
                src="defis-pme-cameroun.jpg" 
                alt="Graphique montrant les défis des PME au Cameroun"
                width="800" 
                height="500"
                loading="lazy"
            >
            <figcaption>Principaux défis des PME (Étude TEDSAI 2024)</figcaption>
        </figure>
        
        <!-- Encadré clé -->
        <aside class="highlight-box">
            <h4>💡 Le Saviez-Vous ?</h4>
            <p>Les PME qui automatisent 3+ processus augmentent leur 
            productivité de 40% en moyenne (Source : Étude TEDSAI 2024)</p>
        </aside>
        
        <!-- Section 2 avec liste structurée -->
        <h2 id="section-2">2. Les 5 Domaines d'Application de l'IA</h2>
        
        <h3>2.1 Automatisation de la Facturation</h3>
        <p>L'automatisation de la facturation est le point d'entrée idéal...</p>
        
        <!-- Lien interne vers page pilier -->
        <p>→ <a href="/solutions-ia/automatisation-facturation">
           Découvrez notre solution d'automatisation facturation
        </a></p>
        
        <!-- Video embed -->
        <figure>
            <iframe 
                width="560" 
                height="315" 
                src="https://www.youtube.com/embed/VIDEO_ID" 
                title="Démo automatisation facturation"
                loading="lazy"
            ></iframe>
            <figcaption>Démonstration : Automatiser une facture en 30 secondes</figcaption>
        </figure>
        
        <!-- Section 3 avec tableau -->
        <h2 id="section-3">3. ROI et Budget : Combien Ça Coûte Vraiment ?</h2>
        
        <table class="comparison-table">
            <thead>
                <tr>
                    <th>Solution</th>
                    <th>Coût Mensuel</th>
                    <th>Économie/Mois</th>
                    <th>ROI</th>
                </tr>
            </thead>
            <tbody>
                <tr>
                    <td>Automatisation Facturation</td>
                    <td>349€</td>
                    <td>~1 200€</td>
                    <td>344%</td>
                </tr>
                <!-- ... -->
            </tbody>
        </table>
        
        <!-- CTA mid-article -->
        <div class="cta-box">
            <h3>Calculez Votre ROI Personnalisé</h3>
            <p>Découvrez combien vous pourriez économiser avec l'IA</p>
            <a href="/calculateur-roi" class="btn-primary">
                Calculer Mon ROI Gratuit →
            </a>
        </div>
        
        <!-- Section 4 avec cas client -->
        <h2 id="section-4">4. Cas Pratiques : Des PME Qui Ont Réussi</h2>
        
        <div class="case-study">
            <h3>📊 Cas #1 : Supermarché "Chez Nous"</h3>
            <div class="case-metrics">
                <div class="metric">
                    <strong>Secteur</strong>
                    <span>Commerce de détail</span>
                </div>
                <div class="metric">
                    <strong>Défi</strong>
                    <span>180 factures/mois en manuel</span>
                </div>
                <div class="metric">
                    <strong>Résultat</strong>
                    <span>-87% temps de traitement</span>
                </div>
            </div>
            <p>Témoignage complet...</p>
            <a href="/cas-clients/supermarche-chez-nous">Lire le cas complet →</a>
        </div>
        
        <!-- Section 5 avec CTA -->
        <h2 id="section-5">5. Par Où Commencer ?</h2>
        
        <div class="action-steps">
            <div class="step">
                <span class="step-number">1</span>
                <h4>Identifiez Votre Processus le Plus Chronophage</h4>
                <p>Utilisez notre checklist gratuite pour identifier...</p>
            </div>
            <div class="step">
                <span class="step-number">2</span>
                <h4>Testez Gratuitement</h4>
                <p>Essayez notre IA Playground sans engagement...</p>
            </div>
            <div class="step">
                <span class="step-number">3</span>
                <h4>Planifiez un Audit</h4>
                <p>Notre équipe analyse gratuitement vos besoins...</p>
            </div>
        </div>
    </div>
    
    <!-- Conclusion avec FAQ Schema -->
    <section class="faq-section" itemscope itemtype="https://schema.org/FAQPage">
        <h2>Questions Fréquentes sur l'IA pour PME</h2>
        
        <div class="faq-item" itemscope itemprop="mainEntity" itemtype="https://schema.org/Question">
            <h3 itemprop="name">Combien coûte une solution IA pour PME ?</h3>
            <div itemscope itemprop="acceptedAnswer" itemtype="https://schema.org/Answer">
                <p itemprop="text">
                    Les solutions IA pour PME commencent à partir de 349€/mois 
                    pour un module d'automatisation simple...
                </p>
            </div>
        </div>
        
        <!-- Plus de FAQ items -->
    </section>
    
    <!-- Footer article -->
    <footer class="post-footer">
        <!-- Tags -->
        <div class="tags">
            <a href="/tag/ia">#IA</a>
            <a href="/tag/pme">#PME</a>
            <a href="/tag/automatisation">#Automatisation</a>
            <a href="/tag/cameroun">#Cameroun</a>
        </div>
        
        <!-- Partage social -->
        <div class="social-share">
            <h4>Partager cet article</h4>
            <button onclick="shareLinkedIn()">LinkedIn</button>
            <button onclick="shareTwitter()">Twitter</button>
            <button onclick="shareFacebook()">Facebook</button>
            <button onclick="shareWhatsApp()">WhatsApp</button>
        </div>
        
        <!-- Bio auteur détaillée -->
        <div class="author-bio">
            <img src="author.jpg" alt="John Doe">
            <div>
                <h4>À Propos de l'Auteur</h4>
                <p><strong>John Doe</strong> - Expert IA chez TEDSAI. 
                10 ans d'expérience en transformation digitale des PME...</p>
                <a href="/auteur/john">Voir tous les articles →</a>
            </div>
        </div>
        
        <!-- Articles similaires -->
        <div class="related-posts">
            <h3>Articles Similaires</h3>
            <div class="related-grid">
                <!-- 3 articles similaires -->
            </div>
        </div>
        
        <!-- CTA final fort -->
        <div class="final-cta">
            <h3>Prêt à Transformer Votre PME Avec l'IA ?</h3>
            <p>Démarrez avec un audit gratuit de 30 minutes</p>
            <a href="/contact?source=blog-ia-pme" class="btn-large">
                Planifier Mon Audit Gratuit
            </a>
            <small>✓ Sans engagement • ✓ Réponse sous 24h • ✓ Conseils personnalisés</small>
        </div>
    </footer>
</article>
8.3 Stratégie de Backlinks
Acquisition de Liens de Qualité
1. Partenariats Médias Locaux
TARGET : Journaux/Sites Camerounais
├─ CameroonWeb.com (DA 52)
├─ Journal du Cameroun (DA 48)
├─ Investir au Cameroun (DA 55)
├─ Business in Cameroon (DA 50)
└─ 237online.com (DA 45)

STRATÉGIE :
• Communiqués de presse mensuels
• Interviews fondateurs/chef
• Articles invités sur innovation/tech
• Couverture événements TEDSAI
2. Guest Blogging Ciblé
TARGET : Blogs Tech/Business Afrique
├─ Afrique IT News (DA 42)
├─ Tech Afrique (DA 38)
├─ StartupBRICS (DA 46)
└─ African Business Central (DA 40)

PITCH ARTICLES :
• "Comment l'IA Transforme les PME en Afrique"
• "Agriculture Urbaine : Solution pour Villes Africaines"
• "Traçabilité Alimentaire : Enjeu Majeur en Afrique"
3. Listings & Directories
PRIORITÉ HAUTE (Gratuit + Do-Follow)
├─ Google My Business (Essential!)
├─ Bing Places
├─ Trip Advisor (viTEDia)
├─ TheFork / LaFourchette
├─ Yelp Cameroun
├─ Pages Jaunes Afrique
└─ Annuaire entreprises Cameroun

PRIORITÉ MOYENNE (Payant mais ROI élevé)
├─ Startups.cm (Profil startup)
├─ Invest in Africa directories
└─ Chamber of Commerce listings
4. Linkable Assets (Contenu Attirant Liens)
CRÉER :
📊 "Baromètre IA PME Cameroun 2025" (Étude annuelle)
   → 50+ liens potentiels de médias/blogs

📘 "Le Grand Guide Agriculture Urbaine Tropicale"
   → Resource référence pour ONG, universités

🧮 "Calculateur ROI Automatisation" (Outil gratuit)
   → Liens de sites business/compta

📹 "Mini-Documentaire : Du Jardin à l'Assiette"
   → Partages médias sociaux/food bloggers

📄 "Infographie : Traçabilité Alimentaire"
   → Téléchargeable, partageable
5. Relations Publiques Digitales
ACTIONS :
• Communiqués succès clients (avec permission)
• Participation événements tech (Cameroun Digital Week)
• Interviews podcasts business/tech
• Webinars gratuits publics (génèrent mentions)
• Partenariats ONG/institutions (liens .org, .edu)
 
9. ASPECTS LÉGAUX & CONFORMITÉ
9.1 Mentions Légales
Page /mentions-legales
MENTIONS LÉGALES

1. ÉDITEUR DU SITE
Raison sociale : TEDSAI Complex SARL
Siège social : [Adresse complète]
Capital social : [Montant] FCFA
RCCM : [Numéro]
NIF : [Numéro]
Téléphone : +237 XXX XXX XXX
Email : contact@tedsai.cm

Directeur de la publication : [Nom Fondateur]

2. HÉBERGEMENT
Hébergeur : Vercel Inc.
Adresse : 340 S Lemon Ave #4133, Walnut, CA 91789
Téléphone : +1 (559) 288-7060

3. PROPRIÉTÉ INTELLECTUELLE
L'ensemble du site (structure, textes, images, logos, 
vidéos) est protégé par le droit d'auteur.

Toute reproduction sans autorisation est interdite.

Marques déposées :
• TEDSAI™
• viTEDia™
• SelecTED Garden™

4. DONNÉES PERSONNELLES
Voir notre Politique de Confidentialité détaillée.

5. COOKIES
Voir notre Politique Cookies.

6. CRÉDITS
Design & Développement : [Agence/Équipe]
Photographies : [Photographe] sauf mention contraire
Illustrations : [sources]

7. CONTACT
Pour toute question juridique :
legal@tedsai.cm
9.2 Politique de Confidentialité (RGPD)
Page /privacy ou /confidentialite
POLITIQUE DE CONFIDENTIALITÉ

Dernière mise à jour : 13 décembre 2024

1. QUI SOMMES-NOUS ?
TEDSAI Complex SARL, responsable du traitement de vos données.
DPO (si applicable) : privacy@tedsai.cm

2. QUELLES DONNÉES COLLECTONS-NOUS ?

2.1 Données fournies directement par vous :
• Nom, prénom
• Email
• Numéro de téléphone
• Entreprise (B2B)
• Données de réservation (B2C)
• Messages via formulaires/TED

2.2 Données collectées automatiquement :
• Adresse IP
• Type de navigateur
• Pages visitées
• Durée de visite
• Cookies (voir section dédiée)

3. POURQUOI COLLECTONS-NOUS CES DONNÉES ?

Base légale : Consentement / Contrat / Intérêt légitime

Finalités :
✓ Traiter vos réservations restaurant
✓ Répondre à vos demandes de contact
✓ Vous envoyer notre newsletter (avec consentement)
✓ Améliorer notre site web
✓ Analyser le trafic (anonymisé)
✓ Détecter fraude/abus

4. AVEC QUI PARTAGEONS-NOUS VOS DONNÉES ?

Nous ne vendons JAMAIS vos données.

Partage limité à :
• Prestataires techniques (hébergement, email)
• Processeurs de paiement (Stripe - si achat)
• Outils analytics (Google Analytics - anonymisé)

Tous nos partenaires sont contractuellement tenus au RGPD.

5. COMBIEN DE TEMPS CONSERVONS-NOUS VOS DONNÉES ?

• Données compte actif : Tant que compte existe
• Newsletter : Jusqu'à désinscription
• Réservations : 3 ans (obligations légales)
• Analytics : 26 mois maximum
• Logs serveur : 12 mois

6. VOS DROITS (RGPD)

Vous disposez des droits suivants :

🔍 DROIT D'ACCÈS
   Obtenir copie de vos données
   → privacy@tedsai.cm

✏️ DROIT DE RECTIFICATION
   Corriger données inexactes
   → Votre compte ou contact

🗑️ DROIT À L'EFFACEMENT ("Droit à l'oubli")
   Supprimer vos données
   → Nous contacter

⏸️ DROIT À LA LIMITATION
   Geler temporairement traitement
   → Nous contacter

📦 DROIT À LA PORTABILITÉ
   Récupérer vos données (format JSON)
   → privacy@tedsai.cm

🚫 DROIT D'OPPOSITION
   S'opposer à certains traitements
   → Désinscription newsletter / contact

⚖️ RÉCLAMATION AUPRÈS AUTORITÉ
   CNIL (France) ou autorité locale

Délai de réponse : 30 jours maximum

7. SÉCURITÉ

Mesures techniques :
• Chiffrement HTTPS (SSL/TLS)
• Mots de passe hashés (bcrypt)
• Sauvegardes quotidiennes chiffrées
• Accès restreint aux données
• Audits sécurité réguliers

8. TRANSFERTS INTERNATIONAUX

Certaines données peuvent être transférées hors Cameroun 
(hébergement US/EU). Tous nos prestataires respectent 
standards RGPD ou clauses contractuelles types.

9. MINEURS

Notre site n'est pas destiné aux moins de 16 ans.
Si vous êtes mineur, demandez autorisation parentale.

10. MODIFICATIONS

Cette politique peut évoluer. Date de dernière MAJ indiquée.
Nous vous notifierons des changements majeurs.

11. CONTACT

Questions sur vos données :
📧 privacy@tedsai.cm
📞 +237 XXX XXX XXX
✉️ TEDSAI Complex, [Adresse]
9.3 Politique Cookies
Page /cookies
POLITIQUE COOKIES

1. QU'EST-CE QU'UN COOKIE ?

Un cookie est un petit fichier texte stocké sur votre 
appareil lors de votre visite.

2. QUELS COOKIES UTILISONS-NOUS ?

🟢 COOKIES STRICTEMENT NÉCESSAIRES (Toujours actifs)
   Essentiels au fonctionnement du site
   
   • tedsai_session : Maintient votre session
   • csrf_token : Protection contre attaques
   • cookie_consent : Mémorise vos choix cookies
   
   Durée : Session ou 1 an
   Désactivation : Impossible (site ne fonctionnerait pas)

🟡 COOKIES FONCTIONNELS (Optionnels - Besoin consentement)
   Améliorent votre expérience
   
   • language_preference : Mémorise langue choisie
   • ted_context : Améliore assistant TED
   • theme_preference : Mode clair/sombre
   
   Durée : 1 an
   Désactivation : Via panneau préférences

🟠 COOKIES ANALYTIQUES (Optionnels)
   Nous aident à comprendre l'usage du site
   
   • _ga, _gid (Google Analytics) : Mesures trafic
   • _hjid (Hotjar) : Heatmaps, enregistrements
   
   Données anonymisées
   Durée : 2 ans (Google), 1 an (Hotjar)
   Désactivation : Via panneau OU extension navigateur

🔴 COOKIES MARKETING (Optionnels)
   Publicité personnalisée (si applicable)
   
   • facebook_pixel : Remarketing Facebook
   • linkedin_insight : Suivi conversions LinkedIn
   
   Durée : 90 jours
   Désactivation : Via panneau préférences

3. COOKIES TIERS

Nous utilisons des services tiers qui déposent cookies :
• Google Analytics (analytics)
• Google Maps (localisation)
• YouTube (vidéos embed)
• Stripe (paiements)

Chacun a sa propre politique cookies.

4. GÉRER VOS PRÉFÉRENCES

🎛️ VIA NOTRE PANNEAU
   [Bouton : Gérer Mes Préférences]
   
🌐 VIA VOTRE NAVIGATEUR
   Chrome : Paramètres > Confidentialité > Cookies
   Firefox : Préférences > Vie privée > Cookies
   Safari : Préférences > Confidentialité
   
🚫 SUPPRIMER LES COOKIES
   Dans paramètres navigateur, effacer historique/cookies
   
⚠️ Impact : Certaines fonctionnalités peuvent ne plus marcher

5. DO NOT TRACK

Nous respectons le signal "Do Not Track" du navigateur.
Si activé, cookies marketing/analytics désactivés.

6. CONTACT

Questions sur les cookies :
cookies@tedsai.cm
9.4 Conditions Générales d'Utilisation (CGU)
Page /cgu
CONDITIONS GÉNÉRALES D'UTILISATION

1. OBJET
Les présentes CGU régissent l'utilisation du site tedsai.com

2. ACCEPTATION
L'accès au site implique acceptation pleine et entière des CGU.

3. SERVICES PROPOSÉS

Le site propose :
• Présentation de nos activités (IA, Restaurant, Jardin)
• Réservation en ligne restaurant viTEDia
• Demandes de contact/devis solutions IA
• Assistant conversationnel TED
• Contenus éducatifs (blog)

4. ACCÈS AU SITE

4.1 Accès libre et gratuit (sauf services payants)

4.2 Nous ne garantissons pas :
• Disponibilité 100% (maintenances possibles)
• Absence totale de bugs
• Compatibilité tous navigateurs (optimisé Chrome, Firefox, Safari récents)

4.3 Vous êtes responsable de :
• Votre connexion Internet
• Sécurité de vos identifiants
• Matériel conforme

5. COMPTE UTILISATEUR

5.1 Création facultative pour :
• Réservations récurrentes
• Programme fidélité
• Historique conversations TED

5.2 Vos obligations :
• Fournir informations exactes
• Maintenir mot de passe confidentiel
• Signaler utilisation non autorisée

5.3 Nous nous réservons le droit de :
• Suspendre compte en cas d'abus
• Supprimer compte inactif (2 ans)

6. UTILISATION ACCEPTABLE

🚫 IL EST INTERDIT DE :
• Utiliser le site à des fins illégales
• Tenter de pirater/compromettre la sécurité
• Envoyer spam/contenu malveillant
• Usurper identité
• Scraper/extraire données massivement
• Nuire à la réputation de TEDSAI

Violations → Suspension immédiate + poursuites si nécessaire

7. PROPRIÉTÉ INTELLECTUELLE

Tous contenus (textes, images, logos, code) sont protégés.

Vous pouvez :
✓ Consulter pour usage personnel
✓ Partager liens vers le site
✓ Citer courts extraits (avec attribution)

Vous ne pouvez pas :
✗ Copier/reproduire contenus sans autorisation
✗ Utiliser nos marques commercialement
✗ Revendre/redistribuer nos contenus

8. RÉSERVATIONS viTEDia (Voir aussi CGV)

8.1 Confirmation :
Réservation ferme après confirmation email/SMS

8.2 Modification :
Possible jusqu'à 24h avant (contact restaurant)

8.3 Annulation :
• Gratuite jusqu'à 24h avant
• Moins de 24h : des frais peuvent s'appliquer

8.4 No-show :
3 no-shows consécutifs = blocage temporaire réservations

9. ASSISTANT TED

9.1 TED est un outil d'assistance, pas de conseil juridique/médical

9.2 Conversations peuvent être enregistrées pour :
• Améliorer le service
• Former les modèles IA
• Support client

9.3 Limites :
• Disponibilité non garantie 100%
• Réponses pas toujours parfaites
• Pas de responsabilité en cas d'erreur

10. LIENS EXTERNES

Le site peut contenir liens vers sites tiers.
Nous ne sommes pas responsables de leur contenu/politique.

11. LIMITATION DE RESPONSABILITÉ

Dans les limites légales :

Nous ne sommes PAS responsables de :
• Dommages indirects
• Perte de données
• Interruption temporaire du service
• Erreurs dans contenus informatifs

Nous SOMMES responsables de :
• Dommages directs causés par notre négligence grave
• Obligations légales impératives

12. FORCE MAJEURE

Non-responsabilité en cas d'événements hors contrôle :
guerre, catastrophe naturelle, panne Internet généralisée, etc.

13. MODIFICATIONS DES CGU

Nous pouvons modifier ces CGU.
Date de dernière MAJ indiquée en haut.
Modifications importantes notifiées par email.

14. DROIT APPLICABLE & JURIDICTION

Droit applicable : Droit camerounais
Juridiction compétente : Tribunaux de Yaoundé

Tentative de résolution amiable préalable obligatoire.

15. CONTACT

Questions sur les CGU :
legal@tedsai.cm
9.5 Conditions Générales de Vente (CGV) - viTEDia
Page /cgv
CONDITIONS GÉNÉRALES DE VENTE - viTEDia Restaurant

1. CHAMP D'APPLICATION

Régissent toutes ventes de :
• Repas sur place (réservation)
• Commandes Click & Collect
• Services traiteur/événements privés

2. RÉSERVATIONS

2.1 Modalités :
• En ligne via tedsai.com/vitedia
• Par téléphone : +237 XXX XXX XXX
• Sur place (selon disponibilités)

2.2 Confirmation :
Email + SMS de confirmation = Réservation ferme

2.3 Arrivée :
• Merci d'arriver 10 min avant
• Au-delà de 15 min de retard : table libérée

2.4 Modification :
Gratuite jusqu'à 24h avant via :
• Lien email de confirmation
• Téléphone
• TED assistant

2.5 Annulation :
• > 24h avant : Gratuite
• 24h-12h avant : Frais 5 000 FCFA/personne
• < 12h avant : Frais 10 000 FCFA/personne
• No-show : Facturation 50% menu moyen

2.6 Groupes (> 8 personnes) :
• Arrhes de 30% exigées
• Non remboursables si annulation < 48h

3. COMMANDES CLICK & COLLECT

3.1 Commande :
• En ligne via site web
• Délai minimum : 2h
• Paiement en ligne obligatoire

3.2 Retrait :
• Créneau choisi lors commande
• Se présenter avec confirmation email
• Si retard > 30 min : Commande annulée sans remboursement

3.3 Modification :
Possible jusqu'à 1h avant retrait prévu

3.4 Annulation :
• > 2h avant retrait : Remboursement intégral
• < 2h : Pas de remboursement (préparation lancée)

4. PRIX

4.1 Tarifs :
• Affichés en FCFA TTC
• Peuvent varier selon saison/disponibilité produits
• Prix confirmé lors de réservation/commande

4.2 Inclus :
• TVA (si applicable)
• Service

4.3 Non inclus :
• Boissons (sauf menus spécifiques)
• Suppléments optionnels (champagne, fleurs, etc.)

5. PAIEMENT

5.1 Modes acceptés :
• Carte bancaire (Visa, Mastercard)
• Mobile Money (Orange, MTN)
• Espèces sur place
• PayPal

5.2 Sécurité :
Paiements en ligne via Stripe (certifié PCI-DSS)

5.3 Facture :
Disponible sur demande (email ou imprimée)

6. ALLERGIES & RÉGIMES SPÉCIAUX

6.1 Information obligatoire :
Mentionnez allergies lors réservation/commande

6.2 Notre engagement :
Adaptation des plats dans la mesure du possible

6.3 Limitations :
Traces possibles malgré précautions
→ Responsabilité client si info non communiquée

7. HYGIÈNE & SÉCURITÉ ALIMENTAIRE

7.1 Normes :
Respect strict normes camerounaises + standards internationaux

7.2 Traçabilité :
100% produits traçables (voir QR codes)

7.3 Conservation :
Plats Click & Collect à consommer dans 2h (sauf indication)

8. RÉCLAMATIONS

8.1 Délai :
Réclamation à formuler :
• Sur place : Immédiatement au personnel
• À distance : 24h max après prestation

8.2 Procédure :
• Email : reclamation@vitedia.cm
• Téléphone : +237 XXX XXX XXX
• Objet, date, détails, photos si applicable

8.3 Traitement :
Réponse sous 48h ouvrées
Solution amiable privilégiée

9. RESPONSABILITÉ

9.1 viTEDia s'engage à :
• Qualité des produits et préparations
• Respect des normes d'hygiène
• Service professionnel

9.2 Limites de responsabilité :
• Vol/dommages objets personnels clients
• Incidents entre clients
• Réactions allergiques si info non fournie

10. DROIT DE RÉTRACTATION

10.1 Prestations de service (repas) :
Pas de droit de rétractation (exécution immédiate)

10.2 Produits Click & Collect :
Produits frais périssables → Pas de rétractation

11. FORCE MAJEURE

Non-responsabilité en cas de :
• Coupure électrique/eau prolongée
• Pénurie approvisionnement majeure
• Catastrophe naturelle
• Restrictions sanitaires gouvernementales

→ Remboursement intégral ou report réservation

12. DONNÉES PERSONNELLES

Voir Politique de Confidentialité complète
Données réservation conservées 3 ans

13. PROGRAMME FIDÉLITÉ

13.1 Points :
Cumulés sur chaque visite (100 points = 1 repas)

13.2 Conditions détaillées :
Voir règlement programme SelecTED Rewards

14. LITIGES

14.1 Résolution amiable :
Tentative obligatoire avant action judiciaire

14.2 Médiation :
Possibilité de recourir à médiateur de consommation

14.3 Juridiction :
Tribunaux de Yaoundé compétents

15. CONTACT

Questions CGV :
📧 cgv@vitedia.cm
📞 +237 XXX XXX XXX

Dernière mise à jour : 13 décembre 2024
 
10. PLAN DE DÉPLOIEMENT
10.1 Phases de Développement
Phase 1 : MVP (Minimum Viable Product) - 6 semaines
Semaines 1-2 : Foundation
LIVRABLE 1 : Infrastructure & Design System
✓ Configuration environnement (Next.js, Tailwind, DB)
✓ Design System complet (composants réutilisables)
✓ Palette couleurs, typographie, grille
✓ Architecture base de données
✓ CI/CD pipeline (GitHub → Vercel)

ÉQUIPE : 1 Dev Frontend, 1 Dev Backend, 1 Designer
Semaines 3-4 : Pages Core
LIVRABLE 2 : Pages Essentielles
✓ Homepage (version complète)
✓ Page Solutions IA (vue d'ensemble + 1 problématique)
✓ Page viTEDia (sans réservation encore)
✓ Page SelecTED Garden (informative)
✓ Page Contact (formulaire basique)
✓ Footer complet
✓ Navigation responsive

ÉQUIPE : 2 Dev Frontend, 1 Dev Backend
Semaines 5-6 : Fonctionnalités Clés
LIVRABLE 3 : MVP Fonctionnel
✓ Système de réservation viTEDia (basique)
✓ Formulaires de contact intelligents
✓ TED Assistant (version simple, règles prédéfinies)
✓ Blog (CMS Strapi + 5 articles)
✓ SEO de base (meta tags, sitemap)
✓ Analytics (Google Analytics 4)

ÉQUIPE : 2 Dev Fullstack, 1 Content Manager

🚀 LANCEMENT MVP : Site en ligne, testing utilisateurs
Phase 2 : Enrichissement - 6 semaines
Semaines 7-9 : Fonctionnalités Avancées
LIVRABLE 4 : Amélioration Expérience
✓ TED Assistant IA avancé (GPT-4 integration)
✓ IA Playground (demo interactive)
✓ Système traçabilité SelecTED Garden
✓ QR codes traçabilité
✓ Pages Solutions IA complètes (5 problématiques)
✓ Pages Secteurs (5 secteurs)
✓ Calculateur ROI

ÉQUIPE : 2 Dev Fullstack, 1 AI Engineer, 1 Designer
Semaines 10-12 : Optimisation & Contenu
LIVRABLE 5 : Contenu & Performance
✓ 20+ articles blog (calendrier editorial)
✓ Cas clients détaillés (3 études de cas)
✓ Vidéos (3 vidéos produits)
✓ Optimisation performance (Lighthouse > 90)
✓ SEO avancé (structured data, backlinks)
✓ Système multi-langue (FR + EN)

ÉQUIPE : 1 Dev Frontend, 2 Content Creators, 1 SEO Specialist
Phase 3 : Gamification & Automation - 4 semaines
Semaines 13-16 : Features Premium
LIVRABLE 6 : Engagement Utilisateur
✓ Programme fidélité SelecTED Rewards
✓ Système de badges & achievements
✓ Dashboard utilisateur personnalisé
✓ Email automation (nurturing campaigns)
✓ Notifications push web
✓ Click & Collect viTEDia
✓ Intégrations paiement (Stripe, Mobile Money)

ÉQUIPE : 2 Dev Fullstack, 1 Marketing Automation Specialist

🎉 LANCEMENT PUBLIC COMPLET : V1.0
10.2 Checklist Pré-Lancement
2 Semaines Avant Lancement :
☐ TECHNIQUE
  ☐ Tests cross-browser (Chrome, Firefox, Safari, Edge)
  ☐ Tests responsive (iPhone, iPad, Android, Desktop)
  ☐ Tests de charge (1000+ utilisateurs simultanés)
  ☐ Backup automatique configuré
  ☐ Monitoring (Sentry, Uptime Robot)
  ☐ SSL certificat valide
  ☐ CDN configuré (Cloudflare)
  ☐ Compression GZIP/Brotli active
  
☐ CONTENU
  ☐ Tous les textes relus (zéro faute)
  ☐ Toutes images optimisées (WebP + fallback)
  ☐ Toutes vidéos uploadées et testées
  ☐ Blog : 10 articles minimum publiés
  ☐ Cas clients : 3 études détaillées
  ☐ FAQ : 30 questions/réponses
  
☐ SEO
  ☐ Meta titles/descriptions toutes pages
  ☐ Alt texts toutes images
  ☐ Sitemap.xml généré et soumis
  ☐ Robots.txt configuré
  ☐ Google Search Console configuré
  ☐ Google Analytics 4 installé et testé
  ☐ Structured data (Schema.org) implémenté
  ☐ Canonical URLs configurées
  
☐ LÉGAL
  ☐ Mentions légales complètes
  ☐ CGU rédigées et validées
  ☐ CGV viTEDia validées
  ☐ Politique de confidentialité (RGPD compliant)
  ☐ Politique cookies
  ☐ Bannière cookies fonctionnelle
  ☐ Procédure RGPD (export/suppression données)
  
☐ SÉCURITÉ
  ☐ Injection SQL : Tests passés
  ☐ XSS : Protection active
  ☐ CSRF : Tokens implémentés
  ☐ Rate limiting actif
  ☐ Mots de passe : Bcrypt hash
  ☐ HTTPS forcé
  ☐ Headers sécurité (Helmet.js)
  
☐ FONCTIONNEL
  ☐ TED Assistant : 50 scénarios testés
  ☐ Réservations : Process complet testé
  ☐ Formulaires : Validation + emails confirmés
  ☐ Paiements : Transactions test OK
  ☐ Traçabilité : QR codes fonctionnels
  ☐ Multi-langue : FR + EN complets
  
☐ MARKETING
  ☐ Google My Business configuré
  ☐ Réseaux sociaux créés (FB, IG, LinkedIn)
  ☐ Newsletter : Template + premier email prêt
  ☐ Communiqué de presse rédigé
  ☐ Liste influenceurs/médias à contacter
  ☐ Visuels social media (20 posts prêts)
Jour J - Lancement :
MATIN (08h00)
☐ 08:00 - Vérification finale tous systèmes
☐ 08:30 - Mise en production (domaine principal)
☐ 09:00 - Tests post-déploiement
☐ 09:30 - Monitoring actif (équipe en alerte)

JOURNÉE (10h00-18h00)
☐ 10:00 - Communiqué de presse envoyé (50 médias)
☐ 10:30 - Posts réseaux sociaux (FB, IG, LinkedIn)
☐ 11:00 - Email VIP clients/partenaires
☐ 12:00 - Newsletter générale
☐ 14:00 - Suivi analytics temps réel
☐ 16:00 - Réponse commentaires/questions
☐ 18:00 - Debriefing équipe

SOIRÉE (19h00+)
☐ 19:00 - Analyse métriques J1
☐ 20:00 - Ajustements urgents si nécessaire
☐ 22:00 - Rapport quotidien
10.3 KPIs à Suivre Post-Lancement
Semaine 1 (Objectifs Initiaux) :
TRAFIC
• Visiteurs uniques : 1 000+
• Pages vues : 5 000+
• Taux de rebond : < 60%
• Durée session : > 2 min

ENGAGEMENT
• Conversations TED : 200+
• Réservations viTEDia : 20+
• Formulaires contact B2B : 30+
• Scans QR traçabilité : 50+

TECHNIQUE
• Uptime : > 99.5%
• Page load : < 3s (moyenne)
• Lighthouse score : > 85
• Zéro bug critique

SOCIAL
• Abonnés réseaux : +200
• Partages : 50+
• Mentions : 10+
Mois 1 (Objectifs Croissance) :
TRAFIC
• Visiteurs uniques : 5 000+
• Trafic organique : 30%
• Trafic direct : 20%
• Taux retour : 15%

CONVERSIONS
• Leads B2B qualifiés : 25+
• Réservations viTEDia : 150+
• Inscriptions newsletter : 500+
• Téléchargements ressources : 100+

CONTENU
• Articles blog : 8 nouveaux
• Backlinks : 20+
• Domaine authority : +5 points

SATISFACTION
• Avis Google : 15+ (moy. 4.5★)
• NPS (Net Promoter Score) : > 50
• Satisfaction TED : > 80%
Mois 3 (Objectifs Maturité) :
TRAFIC
• Visiteurs uniques : 15 000+
• Trafic organique : 50%
• Pages indexées Google : 100+

CONVERSIONS
• Leads B2B : 50+/mois
• Taux conversion leads : 15%
• Réservations viTEDia : 400+/mois
• Clients récurrents : 30%

REVENUS (indicatif)
• CA viTEDia via site : 40%
• Contrats IA signés : 5+
• LTV client moyen : [Objectif]

NOTORIÉTÉ
• Mentions médias : 10+ articles
• DA (Domain Authority) : 35+
• Classement Google (mots-clés cibles) : Top 10
 
CONCLUSION & ROADMAP FUTURE
10.4 Évolutions Futures (Phases Suivantes)
Phase 4 : Expansion (Mois 6-12)
🌍 INTERNATIONAL
• Version Anglais complète
• Version Allemand/Espagnol
• Multi-devises (EUR, USD, FCFA)

🤖 IA AVANCÉE
• TED avec voix (speech-to-text)
• Recommandations produits IA personnalisées
• Chatbot proactif (anticipe besoins)

📱 APPLICATIONS MOBILES
• App iOS viTEDia (réservations + fidélité)
• App Android viTEDia
• PWA (Progressive Web App)

🔗 INTÉGRATIONS
• Uber Eats / Glovo (livraison)
• Booking platforms (OpenTable, TheFork)
• ERP clients B2B (Sage, Odoo, SAP)
• CRM (Salesforce, HubSpot)
Phase 5 : Écosystème Complet (Année 2)
🛒 MARKETPLACE
• Boutique en ligne SelecTED Garden
• Paniers hebdomadaires personnalisés
• Abonnement légumes bio

🎓 ACADEMY
• Cours en ligne IA pour PME
• Certifications TEDSAI
• Webinars mensuels

👥 COMMUNAUTÉ
• Forum utilisateurs
• Réseau ambassadeurs
• Programme affiliation

🔬 R&D
• API publique TEDSAI
• Open source certains modules
• Partenariats recherche universités

